\begin{frame}[fragile]{Definire il database}
I database vengono creati con il comando \textbf{CREATE DATABASE}, seguito dal nome del database:
\begin{lstlisting}
CREATE DATABASE nomeDB;
\end{lstlisting}
\end{frame}
%
\begin{frame}{Definire le tabelle}
Le tabelle vengono definite con il comando \textbf{CREATE TABLE}, seguito dal nome della tabella seguito, tra parentesi, dall'elenco degli attributi.
\begin{itemize}
    \item Per ogni attributo occorre specificare il nome e il tipo di dato;
    \item Gli attributi possono essere qualificati mediante diverse clausole che permettono di definire:
    \begin{itemize}
        \item la chiave primaria;
        \item le chiavi esterne;
        \item l'obbligatoriet\`a e il valore di default di un campo.
    \end{itemize}
\end{itemize}
\end{frame}
%
\begin{frame}[fragile]{DDL: CREATE TABLE}
\begin{lstlisting}
CREATE TABLE nomeTabella(
    campo1 tipoDato [valore di default] [vincoli],
    campo2 tipoDato [valore di default] [vincoli],
    campo3 tipoDato [valore di default] [vincoli],
    |\ldots|
);
\end{lstlisting}
\end{frame}
%
\begin{frame}[fragile]{CREATE TABLE: Impiegati}
\begin{lstlisting}
CREATE TABLE Impiegati(
    ID              smallint PRIMARY KEY,
    Nome            char(20) NOT NULL,
    Cognome         char(30) NOT NULL,
    Residenza       char(30) DEFAULT `*** Manca residenza',
    Stipendio       decimal(9,2),
    Dipartimento    char(5) REFERENCES Dipartimenti(Codice),
    UNIQUE(Cognome, Nome, Dipartimento)
    );
\end{lstlisting}
\end{frame}
%
\begin{frame}[fragile]{Osservazioni CREATE TABLE: Impiegati}
\begin{lstlisting}
CREATE TABLE Impiegati(
    |\colorbox{magenta!40}{ID              smallint PRIMARY KEY}|,
    Nome            char(20) NOT NULL,
    Cognome         char(30) NOT NULL,
    Residenza       char(30) DEFAULT `*** Manca residenza',
    Stipendio       decimal(9,2),
    Dipartimento    char(5) REFERENCES Dipartimenti(Codice),
    UNIQUE(Cognome, Nome, Dipartimento)
);
\end{lstlisting}
\begin{minipage}{0.4\textwidth}
    \begin{block}{ID: chiave primaria}
        ID \`e chiave primaria della tabella.
    \end{block}
    \end{minipage}
\end{frame}
%
\begin{frame}[fragile]{Osservazioni CREATE TABLE: Impiegati}
\begin{lstlisting}
CREATE TABLE Impiegati(
    ID              smallint PRIMARY KEY,
    |\colorbox{magenta!40}{Nome \quad\quad\quad\quad\quad\quad\quad char(20) NOT NULL}|,
    |\colorbox{magenta!40}{Cognome\quad\quad\quad\quad\quad\quad char(30) NOT NULL}|,
    Residenza       char(30) DEFAULT `*** Manca residenza',
    Stipendio       decimal(9,2),
    Dipartimento    char(5) REFERENCES Dipartimenti(Codice),
    UNIQUE(Cognome, Nome, Dipartimento)
);
\end{lstlisting}
\begin{block}{Nome, Cognome}
    Campi obbligatori: non sono ammessi i valori nulli in \textit{Nome} e \textit{Cognome}.
\end{block}
\end{frame}
%
\begin{frame}[fragile]{Osservazioni CREATE TABLE: Impiegati}
\begin{lstlisting}
CREATE TABLE Impiegati(
    ID              smallint PRIMARY KEY,
    Nome            char(20) NOT NULL,
    Cognome         char(30) NOT NULL,
    |\colorbox{magenta!40}{Residenza\quad\quad\quad\quad\quad char(30) DEFAULT `*** Manca residenza'}|,
    Stipendio       decimal(9,2),
    Dipartimento    char(5) REFERENCES Dipartimenti(Codice),
    UNIQUE(Cognome, Nome, Dipartimento)
);
\end{lstlisting}
\begin{minipage}{0.7\textwidth}
\begin{block}{Residenza}
    Il valore da attribuire a \textit{Residenza} in caso di valore nullo.
\end{block}
\end{minipage}
\end{frame}
%
\begin{frame}[fragile]{Osservazioni CREATE TABLE: Impiegati}
\begin{lstlisting}
CREATE TABLE Impiegati(
    ID              smallint PRIMARY KEY,
    Nome            char(20) NOT NULL,
    Cognome         char(30) NOT NULL,
    Residenza       char(30) DEFAULT `*** Manca residenza',
    Stipendio       decimal(9,2),
    |\colorbox{magenta!40}{Dipartimento \quad\quad\quad char(5) REFERENCES Dipartimenti(Codice)}|,
    UNIQUE(Cognome, Nome, Dipartimento)
);
\end{lstlisting}
\begin{block}{Dipartimento: chiave esterna}
    \textit{Dipartimento} \`e chiave esterna associata a \textit{Codice}. Crea un vincolo di integrit\`a referenziale con \textit{Dipartimenti}.
\end{block}
\end{frame}
%
\begin{frame}[fragile]{Osservazioni CREATE TABLE: Impiegati}
\begin{lstlisting}
CREATE TABLE Impiegati(
    ID              smallint PRIMARY KEY,
    Nome            char(20) NOT NULL,
    Cognome         char(30) NOT NULL,
    Residenza       char(30) DEFAULT `*** Manca residenza',
    Stipendio       decimal(9,2),
    Dipartimento    char(5) REFERENCES Dipartimenti(Codice),
    |\colorbox{magenta!40}{UNIQUE(Cognome, Nome, Dipartimento)}|
);
\end{lstlisting}
\begin{block}{Cognome, Nome e Dipartimento: Vincolo di unicit\`a}
    \textbf{UNIQUE} vieta la presenza di valori duplicati in una colonna o un insieme di colonne: non ci possono essere due dipendenti con identico nome e cognome nello stesso dipartimento.
\end{block}
\end{frame}
%
\begin{frame}[fragile]{Osservazioni CREATE TABLE: Impiegati}
\begin{lstlisting}
CREATE TABLE Impiegati(
    ID              smallint PRIMARY KEY,
    Nome            char(20) NOT NULL,
    Cognome         char(30) NOT NULL,
    Residenza       char(30) DEFAULT `*** Manca residenza',
    Stipendio       decimal(9,2),
    Dipartimento    char(5) REFERENCES Dipartimenti(Codice),
    UNIQUE(Cognome, Nome, Dipartimento)
|\colorbox{magenta!40}{);}|
\end{lstlisting}
\begin{minipage}{0.6\textwidth}
    \begin{block}{Fine comando: punto e virgola}
        Ogni comando SQL termina con il punto e virgola.
    \end{block}
\end{minipage}
\end{frame}
%
\begin{frame}{CREATE TABLE: Impiegati}
\onslide<1->{\begin{tabular}{|c|c|c|c|c|c|}
    \hline
    \rowcolor{cyan!30}ID & Nome & Cognome & Residenza & Stipendio & Dipartimento\\
    \hline
    & & & & & \\
    \hline
\end{tabular}}
\vspace*{0.9cm}

\onslide<2>{Come facciamo a conoscere la struttura della tabella?}
\end{frame}
%
\begin{frame}[fragile]{Descrivere le tabelle}
Il comando \textbf{DESCRIBE} viene utilizzato per descrivere la struttura di una tabella.
\begin{lstlisting}
DESCRIBE nomeTabella;
\end{lstlisting}
\begin{tabular}{|c|c|c|c|c|c|}
    \hline
    \rowcolor{cyan!30} Field & Type & Null & Key & Default & Extra \\ \hline
    & & & & & \\ \hline
\end{tabular}
\end{frame}
%
\begin{frame}[fragile]{DESCRIBE TABLE: Impiegati}
\begin{lstlisting}
DESCRIBE Impiegati;
\end{lstlisting}
\begin{tabular}{|c|c|c|c|c|c|}
    \hline
    \rowcolor{cyan!30} Field & Type & Null & Key & Default & Extra \\
    \hline
    ID & smallint & NO & PRI & NULL & \\ \hline
    Nome & char(20) & NO & & NULL & \\ \hline
    Cognome & char(30) & NO & MUL & NULL & \\ \hline
    Residenza & char(30) & YES & & *** Manca residenza & \\ \hline
    Stipendio & decimal(9,2) & YES & & NULL & \\ \hline
    Dipartimento & char(5) & YES & & NULL & \\ \hline
\end{tabular}
\end{frame}
%
\begin{frame}[fragile]{CREATE TABLE: Dipartimenti}
\begin{lstlisting}
CREATE TABLE Dipartimenti(
    Codice          char(5),
    Descrizione     char(20) NOT NULL,
    Sede            char(20),
    Manager         smallint,
    PRIMARY KEY (Codice),
    UNIQUE(Descrizione),
    FOREIGN KEY (Manager) REFERENCES Impiegati(ID)
        ON DELETE SET NULL
        ON UPDATE CASCADE
);    
\end{lstlisting}
\end{frame}
%
\begin{frame}[fragile]{Osservazioni CREATE TABLE: Dipartimenti}
\vspace{-.6cm}
\begin{lstlisting}
CREATE TABLE Dipartimenti(
    Codice          char(5),
    Descrizione     char(20) NOT NULL,
    Sede            char(20),
    Manager         smallint,
    |\colorbox{magenta!40}{PRIMARY KEY (Codice)}|,
    UNIQUE(Descrizione),
    FOREIGN KEY (Manager) REFERENCES Impiegati(ID)
        ON DELETE SET NULL
        ON UPDATE CASCADE
);    
\end{lstlisting}
\begin{block}{Codice: chiave primaria}
    \textit{Codice} \`e chiave primaria della tabella. La clausola \textbf{PRIMARY KEY} pu\`o essere inserita come riga isolata per permettere di dichiarare chiavi primarie formate da pi\`u campi.
\end{block}
\end{frame}
%
\begin{frame}[fragile]{Osservazioni CREATE TABLE: Dipartimenti}
\vspace{-.6cm}
\begin{lstlisting}
CREATE TABLE Dipartimenti(
    Codice          char(5),
    Descrizione     char(20) NOT NULL,
    Sede            char(20),
    Manager         smallint,
    PRIMARY KEY (Codice),
    |\colorbox{magenta!40}{UNIQUE(Descrizione)}|,
    FOREIGN KEY (Manager) REFERENCES Impiegati(ID)
        ON DELETE SET NULL
        ON UPDATE CASCADE
);    
\end{lstlisting}
\begin{block}{Descrizione: vincolo di unicit\`a}
    Nella colonna \textit{Descrizione} non ci possono essere valori duplicati. \textit{Descrizione} identifica univocamente una riga della tabella.
\end{block}
\end{frame}
%
\begin{frame}[fragile]{Osservazioni CREATE TABLE: Dipartimenti}
\vspace{-.6cm}
\begin{lstlisting}
CREATE TABLE Dipartimenti(
    Codice          char(5),
    Descrizione     char(20) NOT NULL,
    Sede            char(20),
    Manager         smallint,
    PRIMARY KEY (Codice),
    UNIQUE(Descrizione),
    |\colorbox{magenta!40}{FOREIGN KEY (Manager) REFERENCES Impiegati(ID)}|
        ON DELETE SET NULL
        ON UPDATE CASCADE
);    
\end{lstlisting}
\begin{block}{Manager: chiave esterna}
    \textit{Manager} \`e chiave esterna associata al campo \textit{ID} della tabella \textit{Impiegati}.
\end{block}
\end{frame}
%
\begin{frame}[fragile]{Osservazioni CREATE TABLE: Dipartimenti}
\vspace{-.6cm}
\begin{lstlisting}
CREATE TABLE Dipartimenti(
    Codice          char(5),
    Descrizione     char(20) NOT NULL,
    Sede            char(20),
    Manager         smallint,
    PRIMARY KEY (Codice),
    UNIQUE(Descrizione),
    FOREIGN KEY (Manager) REFERENCES Impiegati(ID)
    |\colorbox{magenta!40}{\quad\quad ON DELETE SET NULL}|
        ON UPDATE CASCADE
);    
\end{lstlisting}
\begin{block}{Cancellazione}
    La cancellazione di una riga di \textit{Impiegati} implica che i valori di \textit{Manager} associati all'\textit{ID} di quella riga assumano valore nullo.
\end{block}
\end{frame}
%
\begin{frame}[fragile]{Osservazioni CREATE TABLE: Dipartimenti}
\vspace{-.6cm}
\begin{lstlisting}
CREATE TABLE Dipartimenti(
    Codice          char(5),
    Descrizione     char(20) NOT NULL,
    Sede            char(20),
    Manager         smallint,
    PRIMARY KEY (Codice),
    UNIQUE(Descrizione),
    FOREIGN KEY (Manager) REFERENCES Impiegati(ID)
        ON DELETE SET NULL
        |\colorbox{magenta!40}{ON UPDATE CASCADE}|
);    
\end{lstlisting}
\begin{block}{Aggiornamento}
    L'aggiornamento di un \textit{ID} associato a \textit{Manager} si riflette a catena sui valori di \textit{Manager}.
\end{block}
\end{frame}
%
\begin{frame}[fragile]{DESCRIBE TABLE: Dipartimenti}
\begin{lstlisting}
DESCRIBE Dipartimenti;
\end{lstlisting}
\begin{tabular}{|c|c|c|c|c|c|}
    \hline
    \rowcolor{cyan!30} Field & Type & Null & Key & Default & Extra \\
    \hline
    Codice & char(5) & NO & PRI & NULL & \\ \hline
    Descrizione & char(20) & NO & UNI & NULL & \\ \hline
    Sede & char(20) & YES & & NULL & \\ \hline
    Manager & smallint & YES & MUL & NULL & \\ \hline
\end{tabular}
\end{frame}
%
\begin{frame}[fragile]{Modificare le tabelle}
La struttura di una tabella pu\`o essere modificata in un secondo momento con il comando \textbf{ALTER TABLE} per:
\begin{itemize}
    \item aggiungere una nuova colonna (\textbf{ADD}) a quelle gi\`a esistenti;
    \begin{lstlisting}
        ALTER TABLE nomeTabella
        ADD nomeColonna tipoDato;
    \end{lstlisting}
    \item oppure per togliere una colonna (\textbf{DROP}).
    \begin{lstlisting}
        ALTER TABLE nomeTabella
        DROP nomeColonna;
    \end{lstlisting}
\end{itemize}
\end{frame}
%
\begin{frame}[fragile]{DDL: ALTER TABLE - ADD Colonna}
Il seguente comando inserisce in \textit{Impiegati} una nuova colonna di nome \textit{Nascita} per registrare la data di nascita del dipendente:
\begin{lstlisting}
ALTER TABLE Impiegati
ADD Nascita date;
\end{lstlisting}
\end{frame}
%
\begin{frame}[fragile]{DDL: ALTER TABLE - DROP Colonna}
Il seguente comando elimina da \textit{Impiegati} la colonna di nome \textit{Residenza}.
\begin{lstlisting}
ALTER TABLE Impiegati
DROP Residenza;
\end{lstlisting}
\begin{minipage}{0.7\textwidth}
    \begin{alertblock}{Attenzione}
        Il comando \textbf{DROP} elimina anche i dati contenuti nella colonna!
    \end{alertblock}
\end{minipage}
\end{frame}
%
\begin{frame}[fragile]{DDL: ALTER TABLE - RENAME Colonna}
Il seguente comando permette di rinominare una colonna di una tabella:
\begin{lstlisting}
ALTER TABLE nomeTabella
RENAME COLUMN vecchioNomeColonna to nuovoNomeColonna;
\end{lstlisting}
\end{frame}
%
\begin{frame}[fragile]{DDL: ALTER TABLE - MODIFY tipoDato Colonna}
Il seguente comando permette di modificare il tipo di dato di una colonna di una tabella:
\begin{lstlisting}
ALTER TABLE nomeTabella
MODIFY COLUMN nomeColonna tipoDato;
\end{lstlisting}
\end{frame}
%
\begin{frame}[fragile]{Eliminare il database}
I database vengono eliminati con il comando comando \textbf{DROP DATABASE}, seguito dal nome del database:
\begin{lstlisting}
DROP DATABASE nomeDB;
\end{lstlisting}
\end{frame}
%
\begin{frame}[fragile]{Eliminare le tabelle}
Il comando \textbf{DROP TABLE} elimina una tabella dal database:
\begin{lstlisting}
DROP TABLE nomeTabella;
\end{lstlisting}
\end{frame}
%
\begin{frame}[fragile]{DDL: DROP TABLE}
Il seguente comando elimina la tabella \textit{Impiegati} dal database:
\begin{lstlisting}
DROP TABLE Impiegati;
\end{lstlisting}
\end{frame}
%