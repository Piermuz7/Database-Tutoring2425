\begin{frame}{Clausole di SQL}
Sintesi delle diverse clausole che possono apparire nel comando \textbf{SELECT}:
\newline
\\\textbf{SELECT}~~~~~~~Elenco espressioni da mostrare
\\\textbf{FROM}~~~~~~~~~Tabelle da cui estrarre le righe
\\\textbf{WHERE}~~~~~~~Condizioni sulle righe estratte
\\\textbf{GROUP BY}~~Campi da considerare per i raggruppamenti
\\\textbf{HAVING}~~~~~~~Condizioni sui raggruppamenti
\\\textbf{ORDER BY}~~~Ordinamenti sulle espressioni elencate nella clausola SELECT
\end{frame}
%
\begin{frame}{Precisazioni sulle clausole SQL}
\`E opportuno fare le seguenti precisazioni:
\begin{itemize}[<+->]
    \item si possono scrivere comandi del tipo: \textbf{SELECT NOW();} oppure \textbf{SELECT 3*4;} nei quali compare la sola clausola SELECT.
    \item Per estrarre dati da una tabella le uniche clausole che devono comparire \textbf{obbligatoriamente} sono \textbf{SELECT} e \textbf{FROM} mentre le altre sono facoltative.
    \item Le clausole utilizzate devono essere elencate rispettando l'ordine (slide precedente):
    SELECT deve precedere FROM, FROM deve precedere WHERE, \ldots, HAVING deve precedere ORDER BY.
\end{itemize}
\end{frame}
%
\begin{frame}{Precisazioni sulle clausole SQL}
\begin{itemize}
    \item Il DBMS esegue il comando SQL elaborando le clausole nel seguente ordine:
\end{itemize}
\begin{center}
\begin{tabular}{|c|c|c|}
    \hline
    \rowcolor{cyan!30} \textbf{Ordine} & \textbf{Clausola} & \textbf{Funzione} \\
    \hline
    1 & FROM & Tabelle da cui estrarre le righe\\ \hline
    2 & WHERE & Condizioni sulle righe estratte\\ \hline
    3 & GROUP BY & Campi da considerare per i raggruppamenti\\ \hline
    4 & HAVING & Condizioni sui raggruppamenti\\ \hline
    5 & SELECT & Elenco espressioni da mostrare\\ \hline
    6 & ORDER BY & Ordinamenti sulle espressioni elencate nella clausola SELECT\\ \hline
    7 & LIMIT & I dati restituiti sono limitati al conteggio delle righe\\ \hline
\end{tabular}
\end{center}
\end{frame}