\begin{frame}{Ripasso sui concetti base di SQL}
    \onslide<1->{\begin{block}{Structured Query Language (SQL)}
        SQL \`e un linguaggio nato con lo scopo di poter eseguire con facilit\`a:
        \begin{itemize}
            \item la \textbf{definizione} e la \textbf{creazione} di un database relazionale;
            \item le diverse operazioni di \textbf{gestione} dei dati (inserimento, cancellazione, variazione) di un archivio;
            \item l'\textbf{interrogazione} del database a scopo informativo.
        \end{itemize}
    \end{block}}
    
    \onslide<2->\noindent Il linguaggio \`e basato su costrutti semplici e facili da imparare.
    \newline
    \onslide<3->\\ Le sue caratteristiche sono standardizzate in modo che l'utente, cambiando DBMS, non debba apprendere un nuovo linguaggio per usare la base di dati.
\end{frame}
%
\begin{frame}{Ripasso sui concetti base di SQL}
    Il linguaggio SQL consente all'utente:
    \begin{itemize}[<+->]
        \item \textbf{Data Definition Language (DDL)}: definire il database, la struttura delle tabelle che lo compongono, gli indici, le associazioni tra le tabelle e le viste logiche;
        \item \textbf{Data Manipulation Language (DML)}: modificare i dati contenuti nel database, con le operazioni di inserimento, variazione e cancellazione ed effettuare le interrogazioni;
        \item \textbf{Data Control Language (DCL)}: definire gli utenti e controllare gli accessi al database;
        \item \textbf{Transaction Control Language (TCL)}: gestire e controllare le transazioni.
    \end{itemize}
\end{frame}