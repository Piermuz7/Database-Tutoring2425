\begin{frame}[fragile]{Le transazioni}
Le transazioni sono un insieme di operazioni di interrogazione o di modifica dei dati del database che devono essere eseguite unitariamente e che godono delle propriet\`a \textbf{ACID}:
\begin{itemize}[<+->]
    \item \textbf{Atomicity}: una transazione \`e un' entit\`a atomica indivisibile. \`E compito del sistema di controllo della sicurezza riuscire a ripristinare la situazione preesistente;
    \item \textbf{Consistency}: le transazioni non devono violare i vincoli di integrit\`a dei dati. La consistenza \`e gestita dal DBMS con procedure opportune;
    \item \textbf{Isolation}: gli effetti di una transazione devono essere indipendenti da quello di tutte le altre transazioni eseguite correntemente. L'isolamento \`e gestito dal controllore della concorrenza;
    \item \textbf{Durability}: le informazioni in un database devono essere memorizzate in modo persistente. Gli effetti di una transazione eseguita con esito positivo devono essere memorizzati permanentemente nel database a cura del sottosistema di gestione della sicurezza e del ripristino dei dati.
\end{itemize}
\end{frame}
%
\begin{frame}{Transazioni in SQL}
Una transazione consiste in un insieme di comandi SQL, inseriti tra apposte dichiarazioni che definiscono l'inizio e la fine di una transazione.
\newline
\\I comandi SQL usati per definire e controllare le transazioni sono:
\begin{itemize}[<+->]
    \item \textbf{START TRANSACTION}: per iniziare una nuova transazione;
    \item \textbf{COMMIT}: per rendere definitivi i cambiamenti apportati al database nella transazione e terminarla;
    \item \textbf{ROLLBACK}: per annullare i cambiamenti eseguiti dall'inizio della transazione e terminarla;
    \item \textbf{SAVEPOINT}: per definire situazioni intermedie alle quali ritornare con un comando \textbf{ROLLBACK TO SAVEPOINT} che annulla solo parzialmente la transazione.
\end{itemize}
\end{frame}
%
\begin{frame}[fragile]{TCL: Esempio ROLLBACK}
Alla fine della transazione con \textbf{ROLLBACK}, le variazioni apportate al database dall'inizio delle transazione sono annullate.
\begin{lstlisting}
START TRANSACTION;

INSERT INTO bollette
VALUES( NULL, `2015-11-21', 100, 102.45, 1);

ROLLBACK;
\end{lstlisting}
\end{frame}
%
\begin{frame}[fragile]{TCL: Esempio COMMIT}
Con \textbf{COMMIT} si conferma l'inserimento della riga e le variazioni sono eseguite.
\begin{lstlisting}
START TRANSACTION;

INSERT INTO bollette
VALUES( NULL, `2015-11-21', 100, 102.45, 1);

COMMIT;
\end{lstlisting}
\end{frame}
%
\begin{frame}[fragile]{TCL: Esempio Transazioni concorrenti}
\begin{minipage}[t]{0.48\linewidth}
\textbf{T1:}
\begin{lstlisting}
START TRANSACTION;

UPDATE bollette
SET Importo=Importo-100
WHERE Numero=12;
\end{lstlisting}
\end{minipage}%
\hfill%
\begin{minipage}[t]{0.48\linewidth}
\textbf{T2:}
\begin{lstlisting}
START TRANSACTION;

UPDATE bollette
SET Importo=Importo-100
WHERE Numero=12;
\end{lstlisting}
\end{minipage}
\vspace{.9cm}

Quale sar\`a il risultato della tabella \textit{bollette} dopo che T1 e T2 vengono eseguite?
\end{frame}
%
\begin{frame}[fragile]{TCL: Esempio Transazioni concorrenti}
\begin{minipage}[t]{0.48\linewidth}
\textbf{T1:}
\begin{lstlisting}
START TRANSACTION;

UPDATE bollette
SET Importo=Importo-100
WHERE Numero=12;
\end{lstlisting}
\end{minipage}%
\hfill%
\begin{minipage}[t]{0.48\linewidth}
\textbf{T2:}
\begin{lstlisting}
START TRANSACTION;

UPDATE bollette
SET Importo=Importo-100
WHERE Numero=12;
\end{lstlisting}
\end{minipage}
\vspace{.5cm}

\begin{itemize}[<+->]
    \item T1 inizia e riduce di 100 l'importo della bolletta 12;
    \item T2 inizia. Non ``vede'' la variazione eseguita da T1;
    \item Per evitare perdite di aggiornamento, il record 12 non pu\`o essere modificato da T2 sino al termine di T1. MySQL blocca T2 in attesa per 50 secondi, poi abortisce il comando e va in errore in timeout.
\end{itemize}
\end{frame}
%
\begin{frame}[fragile]{TCL: Esempio Transazioni concorrenti}
\begin{minipage}[t]{0.48\linewidth}
\textbf{T1:}
\begin{lstlisting}
COMMIT;
\end{lstlisting}
\end{minipage}%
\hfill%
\begin{minipage}[t]{0.48\linewidth}
\textbf{T2:}
\begin{lstlisting}
UPDATE bollette
SET Importo=Importo-100
WHERE Numero=12;

COMMIT;
\end{lstlisting}
\end{minipage}
\vspace{.5cm}

\begin{itemize}[<+->]
    \item Completata la transazione T1, l'aggiornamento di T2 pu\`o essere eseguito;
    \item I due aggiornamenti sono stati eseguiti in sequenza e il valore dell'importo al termine dei due aggiornamenti \`e quello corretto.
\end{itemize}
\end{frame}