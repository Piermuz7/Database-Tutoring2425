\documentclass[11pt,aspectratio=169]{beamer}
\usepackage[italian]{babel}
\usepackage[latin1]{inputenc}
\usepackage{graphicx}
\usepackage{listings}
\usepackage[export]{adjustbox}
% Custom bullets
\usepackage{pifont}
\usepackage[useregional]{datetime2}
\usepackage{tikz}
\usepackage[table]{colortbl}
\usepackage{venndiagram}
\usepackage{amsmath}
\usepackage{fancyvrb} % indent in verbatim


% custom legend
\newcommand{\cbox}[1]{\raisebox{\depth}{\fcolorbox{black}{#1}{\null}}}

\usetheme{Madrid}
\usecolortheme{spruce}

% Redefine labels
\deftranslation[to=Italian]{Section}{Sezione}
\deftranslation[to=Italian]{Subsection}{Sottosezione}

% Join commands
\def\ojoin{\setbox0=\hbox{$\bowtie$}%
  \rule[-.02ex]{.25em}{.4pt}\llap{\rule[\ht0]{.25em}{.4pt}}}
\def\leftouterjoin{\mathbin{\ojoin\mkern-5.8mu\bowtie}}
\def\rightouterjoin{\mathbin{\bowtie\mkern-5.8mu\ojoin}}
\def\fullouterjoin{\mathbin{\ojoin\mkern-5.8mu\bowtie\mkern-5.8mu\ojoin}}

% highlight text
\lstset{
  basicstyle=\ttfamily,
  escapeinside=||
}

\author[Gianluca Lanchini \and Piermichele Rosati]{Gianluca Lanchini \and Piermichele Rosati}

\institute[]{\large Universit\`a di Camerino\\ \footnotesize Tutorato - Basi di Dati}

\title[Caratteristiche di SQL]{5. Caratteristiche di SQL}
\subtitle{Caratteristiche, Tipi di Dati, DDL, DML, DCL, TCL}
\setbeamertemplate{navigation symbols}{}
\setbeamertemplate{section in toc}[sections numbered]
\setbeamertemplate{subsection in toc}[subsections numbered]
%\titlegraphic{\includegraphics[width=6cm]{img/unicam-logo.jpg}}
\makeatletter
\setbeamertemplate{footline}{
    \leavevmode%
    \hbox{%
        \begin{beamercolorbox}[wd=.333333\paperwidth,ht=2.25ex,dp=1ex,center]{author in head/foot}%
            \usebeamerfont{author in head/foot}\insertshortauthor
        \end{beamercolorbox}%
        \begin{beamercolorbox}[wd=.333333\paperwidth,ht=2.25ex,dp=1ex,center]{title in head/foot}%
            \usebeamerfont{title in head/foot}\insertshorttitle
        \end{beamercolorbox}%
        \begin{beamercolorbox}[wd=.333333\paperwidth,ht=2.25ex,dp=1ex,right]{date in head/foot}%
            \usebeamerfont{date in head/foot}\today\hspace*{5 em}
            \insertframenumber{} / \inserttotalframenumber\hspace*{2ex} 
        \end{beamercolorbox}%
    }%
    \vskip0pt%
}
\makeatother


\AtBeginSection[]
{
\begin{frame}<beamer>
\frametitle{Indice}
\tableofcontents[currentsection]
\end{frame}
}

\begin{document}
\begin{frame}
\centering
\includegraphics[width=5.5cm]{../img/unicam-logo.jpg}
\date{\today}
\titlepage
\end{frame}
\addtobeamertemplate{frametitle}{}{%
\begin{tikzpicture}[remember picture,overlay]
\node[anchor=north east,yshift=2pt] at (current page.north east) {\includegraphics[height=2cm]{../img/unicam-logo-notext.png}};
\end{tikzpicture}}

%\begin{frame}
%\titlepage
%\end{frame}

\section{Caratteristiche generali di SQL}
\begin{frame}{Ripasso sui concetti base di SQL}
    \onslide<1->{\begin{block}{Structured Query Language (SQL)}
        SQL \`e un linguaggio nato con lo scopo di poter eseguire con facilit\`a:
        \begin{itemize}
            \item la \textbf{definizione} e la \textbf{creazione} di un database relazionale;
            \item le diverse operazioni di \textbf{gestione} dei dati (inserimento, cancellazione, variazione) di un archivio;
            \item l'\textbf{interrogazione} del database a scopo informativo.
        \end{itemize}
    \end{block}}
    
    \onslide<2->\noindent Il linguaggio \`e basato su costrutti semplici e facili da imparare.
    \newline
    \onslide<3->\\ Le sue caratteristiche sono standardizzate in modo che l'utente, cambiando DBMS, non debba apprendere un nuovo linguaggio per usare la base di dati.
\end{frame}
%
\begin{frame}{Ripasso sui concetti base di SQL}
    Il linguaggio SQL consente all'utente:
    \begin{itemize}[<+->]
        \item \textbf{Data Definition Language (DDL)}: definire il database, la struttura delle tabelle che lo compongono, gli indici, le associazioni tra le tabelle e le viste logiche;
        \item \textbf{Data Manipulation Language (DML)}: modificare i dati contenuti nel database, con le operazioni di inserimento, variazione e cancellazione ed effettuare le interrogazioni;
        \item \textbf{Data Control Language (DCL)}: definire gli utenti e controllare gli accessi al database;
        \item \textbf{Transaction Control Language (TCL)}: gestire e controllare le transazioni.
    \end{itemize}
\end{frame}
%
\section{Identificatori e tipi di dati}
\begin{frame}{Identificatori}
    \onslide<1->Gli \textbf{Identificatori} (nomi di tabelle e di attributi) sono costituiti da sequenze di caratteri: devono iniziare con una lettera e possono anche contenere il carattere \_ .
    \newline
    \onslide<2->\\ Il nome di un attributo (colonna di una tabella) \`e identificato per mezzo della notazione:
    \[ NomeTabella.NomeAttributo \]
    \onslide<2->Il nome della tabella pu\`o essere omesso se non ci sono ambiguit\`a nell'identificazione dell'attributo.
\end{frame}
%
\begin{frame}[allowframebreaks]{Tipi di Dati}
    Ogni colonna di una tabella di database deve avere un nome e un tipo di dati.
    \centering
    \begin{tabular}{|c|c|}
        \hline
        \rowcolor{cyan!30}Tipo di dato & Descrizione \\
        \hline
        CHARACTER(n) & Stringa di caratteri. Lunghezza fissa n \\ \hline
        VARCHAR(n) & Stringa di caratteri. Lunghezza variabile. Lunghezza massima n \\ \hline
        BINARY(n) & Stringa binaria. Lunghezza fissa n \\ \hline
        BOOLEAN & Memorizza i valori TRUE o FALSE \\ \hline
        VARBINARY(n) & Stringa binaria. Lunghezza variabile. Lunghezza massima n \\ \hline
        INTEGER(p) & Numero intero (senza decimali). Precisione p \\ \hline
        SMALLINT & Numero intero (senza decimali). Precisione 5 \\ \hline
        INTEGER	& Numero intero (senza decimali). Precisione 10 \\ \hline
        BIGINT & Numero intero (senza decimali). Precisione 19 \\ \hline
        DECIMAL(p,s) & Numero decimale con precisione p e s cifre decimali \\ \hline
        NUMERIC(p,s) & Numero decimale con precisione p e s cifre decimali \\ \hline
        \end{tabular}

        \centering
    \begin{tabular}{|c|c|}
        \hline
        \rowcolor{cyan!30}Tipo di dato & Descrizione \\
        \hline
        FLOAT(p) & Numero reale con mantissa di precisione p \\ \hline
        REAL & Numero reale con mantissa di precisione 7 \\ \hline
        FLOAT & Numero reale con mantissa di precisione 16 \\ \hline
        DOUBLE PRECISION & Numero reale con mantissa di precisione 16 \\ \hline
        DATE & Memorizza i valori di anno, mese e giorno \\ \hline
        TIME & Memorizza i valori di ore, minuti e secondi \\ \hline
        TIMESTAMP & Memorizza i valori di anno, mese, giorno, ora, minuto e secondo \\ \hline
        ARRAY & un insieme ordinato di n elementi \\ \hline
        XML & Memorizza dati XML \\ \hline
        \end{tabular}
\end{frame}
%
\section{Comandi per la Definizione di DB e Tabelle}
\begin{frame}[fragile]{Definire il database}
I database vengono creati con il comando \textbf{CREATE DATABASE}, seguito dal nome del database:
\begin{lstlisting}
CREATE DATABASE nomeDB;
\end{lstlisting}
\end{frame}
%
\begin{frame}{Definire le tabelle}
Le tabelle vengono definite con il comando \textbf{CREATE TABLE}, seguito dal nome della tabella seguito, tra parentesi, dall'elenco degli attributi.
\begin{itemize}
    \item Per ogni attributo occorre specificare il nome e il tipo di dato;
    \item Gli attributi possono essere qualificati mediante diverse clausole che permettono di definire:
    \begin{itemize}
        \item la chiave primaria;
        \item le chiavi esterne;
        \item l'obbligatoriet\`a e il valore di default di un campo.
    \end{itemize}
\end{itemize}
\end{frame}
%
\begin{frame}[fragile]{DDL: CREATE TABLE}
\begin{lstlisting}
CREATE TABLE nomeTabella(
    campo1 tipoDato [valore di default] [vincoli],
    campo2 tipoDato [valore di default] [vincoli],
    campo3 tipoDato [valore di default] [vincoli],
    |\ldots|
);
\end{lstlisting}
\end{frame}
%
\begin{frame}[fragile]{CREATE TABLE: Impiegati}
\begin{lstlisting}
CREATE TABLE Impiegati(
    ID              smallint PRIMARY KEY,
    Nome            char(20) NOT NULL,
    Cognome         char(30) NOT NULL,
    Residenza       char(30) DEFAULT `*** Manca residenza',
    Stipendio       decimal(9,2),
    Dipartimento    char(5) REFERENCES Dipartimenti(Codice),
    UNIQUE(Cognome, Nome, Dipartimento)
    );
\end{lstlisting}
\end{frame}
%
\begin{frame}[fragile]{Osservazioni CREATE TABLE: Impiegati}
\begin{lstlisting}
CREATE TABLE Impiegati(
    |\colorbox{magenta!40}{ID              smallint PRIMARY KEY}|,
    Nome            char(20) NOT NULL,
    Cognome         char(30) NOT NULL,
    Residenza       char(30) DEFAULT `*** Manca residenza',
    Stipendio       decimal(9,2),
    Dipartimento    char(5) REFERENCES Dipartimenti(Codice),
    UNIQUE(Cognome, Nome, Dipartimento)
);
\end{lstlisting}
\begin{minipage}{0.4\textwidth}
    \begin{block}{ID: chiave primaria}
        ID \`e chiave primaria della tabella.
    \end{block}
    \end{minipage}
\end{frame}
%
\begin{frame}[fragile]{Osservazioni CREATE TABLE: Impiegati}
\begin{lstlisting}
CREATE TABLE Impiegati(
    ID              smallint PRIMARY KEY,
    |\colorbox{magenta!40}{Nome \quad\quad\quad\quad\quad\quad\quad char(20) NOT NULL}|,
    |\colorbox{magenta!40}{Cognome\quad\quad\quad\quad\quad\quad char(30) NOT NULL}|,
    Residenza       char(30) DEFAULT `*** Manca residenza',
    Stipendio       decimal(9,2),
    Dipartimento    char(5) REFERENCES Dipartimenti(Codice),
    UNIQUE(Cognome, Nome, Dipartimento)
);
\end{lstlisting}
\begin{block}{Nome, Cognome}
    Campi obbligatori: non sono ammessi i valori nulli in \textit{Nome} e \textit{Cognome}.
\end{block}
\end{frame}
%
\begin{frame}[fragile]{Osservazioni CREATE TABLE: Impiegati}
\begin{lstlisting}
CREATE TABLE Impiegati(
    ID              smallint PRIMARY KEY,
    Nome            char(20) NOT NULL,
    Cognome         char(30) NOT NULL,
    |\colorbox{magenta!40}{Residenza\quad\quad\quad\quad\quad char(30) DEFAULT `*** Manca residenza'}|,
    Stipendio       decimal(9,2),
    Dipartimento    char(5) REFERENCES Dipartimenti(Codice),
    UNIQUE(Cognome, Nome, Dipartimento)
);
\end{lstlisting}
\begin{minipage}{0.7\textwidth}
\begin{block}{Residenza}
    Il valore da attribuire a \textit{Residenza} in caso di valore nullo.
\end{block}
\end{minipage}
\end{frame}
%
\begin{frame}[fragile]{Osservazioni CREATE TABLE: Impiegati}
\begin{lstlisting}
CREATE TABLE Impiegati(
    ID              smallint PRIMARY KEY,
    Nome            char(20) NOT NULL,
    Cognome         char(30) NOT NULL,
    Residenza       char(30) DEFAULT `*** Manca residenza',
    Stipendio       decimal(9,2),
    |\colorbox{magenta!40}{Dipartimento \quad\quad\quad char(5) REFERENCES Dipartimenti(Codice)}|,
    UNIQUE(Cognome, Nome, Dipartimento)
);
\end{lstlisting}
\begin{block}{Dipartimento: chiave esterna}
    \textit{Dipartimento} \`e chiave esterna associata a \textit{Codice}. Crea un vincolo di integrit\`a referenziale con \textit{Dipartimenti}.
\end{block}
\end{frame}
%
\begin{frame}[fragile]{Osservazioni CREATE TABLE: Impiegati}
\begin{lstlisting}
CREATE TABLE Impiegati(
    ID              smallint PRIMARY KEY,
    Nome            char(20) NOT NULL,
    Cognome         char(30) NOT NULL,
    Residenza       char(30) DEFAULT `*** Manca residenza',
    Stipendio       decimal(9,2),
    Dipartimento    char(5) REFERENCES Dipartimenti(Codice),
    |\colorbox{magenta!40}{UNIQUE(Cognome, Nome, Dipartimento)}|
);
\end{lstlisting}
\begin{block}{Cognome, Nome e Dipartimento: Vincolo di unicit\`a}
    \textbf{UNIQUE} vieta la presenza di valori duplicati in una colonna o un insieme di colonne: non ci possono essere due dipendenti con identico nome e cognome nello stesso dipartimento.
\end{block}
\end{frame}
%
\begin{frame}[fragile]{Osservazioni CREATE TABLE: Impiegati}
\begin{lstlisting}
CREATE TABLE Impiegati(
    ID              smallint PRIMARY KEY,
    Nome            char(20) NOT NULL,
    Cognome         char(30) NOT NULL,
    Residenza       char(30) DEFAULT `*** Manca residenza',
    Stipendio       decimal(9,2),
    Dipartimento    char(5) REFERENCES Dipartimenti(Codice),
    UNIQUE(Cognome, Nome, Dipartimento)
|\colorbox{magenta!40}{);}|
\end{lstlisting}
\begin{minipage}{0.6\textwidth}
    \begin{block}{Fine comando: punto e virgola}
        Ogni comando SQL termina con il punto e virgola.
    \end{block}
\end{minipage}
\end{frame}
%
\begin{frame}{CREATE TABLE: Impiegati}
\onslide<1->{\begin{tabular}{|c|c|c|c|c|c|}
    \hline
    \rowcolor{cyan!30}ID & Nome & Cognome & Residenza & Stipendio & Dipartimento\\
    \hline
    & & & & & \\
    \hline
\end{tabular}}
\vspace*{0.9cm}

\onslide<2>{Come facciamo a conoscere la struttura della tabella?}
\end{frame}
%
\begin{frame}[fragile]{Descrivere le tabelle}
Il comando \textbf{DESCRIBE} viene utilizzato per descrivere la struttura di una tabella.
\begin{lstlisting}
DESCRIBE nomeTabella;
\end{lstlisting}
\begin{tabular}{|c|c|c|c|c|c|}
    \hline
    \rowcolor{cyan!30} Field & Type & Null & Key & Default & Extra \\ \hline
    & & & & & \\ \hline
\end{tabular}
\end{frame}
%
\begin{frame}[fragile]{DESCRIBE TABLE: Impiegati}
\begin{lstlisting}
DESCRIBE Impiegati;
\end{lstlisting}
\begin{tabular}{|c|c|c|c|c|c|}
    \hline
    \rowcolor{cyan!30} Field & Type & Null & Key & Default & Extra \\
    \hline
    ID & smallint & NO & PRI & NULL & \\ \hline
    Nome & char(20) & NO & & NULL & \\ \hline
    Cognome & char(30) & NO & MUL & NULL & \\ \hline
    Residenza & char(30) & YES & & *** Manca residenza & \\ \hline
    Stipendio & decimal(9,2) & YES & & NULL & \\ \hline
    Dipartimento & char(5) & YES & & NULL & \\ \hline
\end{tabular}
\end{frame}
%
\begin{frame}[fragile]{CREATE TABLE: Dipartimenti}
\begin{lstlisting}
CREATE TABLE Dipartimenti(
    Codice          char(5),
    Descrizione     char(20) NOT NULL,
    Sede            char(20),
    Manager         smallint,
    PRIMARY KEY (Codice),
    UNIQUE(Descrizione),
    FOREIGN KEY (Manager) REFERENCES Impiegati(ID)
        ON DELETE SET NULL
        ON UPDATE CASCADE
);    
\end{lstlisting}
\end{frame}
%
\begin{frame}[fragile]{Osservazioni CREATE TABLE: Dipartimenti}
\vspace{-.6cm}
\begin{lstlisting}
CREATE TABLE Dipartimenti(
    Codice          char(5),
    Descrizione     char(20) NOT NULL,
    Sede            char(20),
    Manager         smallint,
    |\colorbox{magenta!40}{PRIMARY KEY (Codice)}|,
    UNIQUE(Descrizione),
    FOREIGN KEY (Manager) REFERENCES Impiegati(ID)
        ON DELETE SET NULL
        ON UPDATE CASCADE
);    
\end{lstlisting}
\begin{block}{Codice: chiave primaria}
    \textit{Codice} \`e chiave primaria della tabella. La clausola \textbf{PRIMARY KEY} pu\`o essere inserita come riga isolata per permettere di dichiarare chiavi primarie formate da pi\`u campi.
\end{block}
\end{frame}
%
\begin{frame}[fragile]{Osservazioni CREATE TABLE: Dipartimenti}
\vspace{-.6cm}
\begin{lstlisting}
CREATE TABLE Dipartimenti(
    Codice          char(5),
    Descrizione     char(20) NOT NULL,
    Sede            char(20),
    Manager         smallint,
    PRIMARY KEY (Codice),
    |\colorbox{magenta!40}{UNIQUE(Descrizione)}|,
    FOREIGN KEY (Manager) REFERENCES Impiegati(ID)
        ON DELETE SET NULL
        ON UPDATE CASCADE
);    
\end{lstlisting}
\begin{block}{Descrizione: vincolo di unicit\`a}
    Nella colonna \textit{Descrizione} non ci possono essere valori duplicati. \textit{Descrizione} identifica univocamente una riga della tabella.
\end{block}
\end{frame}
%
\begin{frame}[fragile]{Osservazioni CREATE TABLE: Dipartimenti}
\vspace{-.6cm}
\begin{lstlisting}
CREATE TABLE Dipartimenti(
    Codice          char(5),
    Descrizione     char(20) NOT NULL,
    Sede            char(20),
    Manager         smallint,
    PRIMARY KEY (Codice),
    UNIQUE(Descrizione),
    |\colorbox{magenta!40}{FOREIGN KEY (Manager) REFERENCES Impiegati(ID)}|
        ON DELETE SET NULL
        ON UPDATE CASCADE
);    
\end{lstlisting}
\begin{block}{Manager: chiave esterna}
    \textit{Manager} \`e chiave esterna associata al campo \textit{ID} della tabella \textit{Impiegati}.
\end{block}
\end{frame}
%
\begin{frame}[fragile]{Osservazioni CREATE TABLE: Dipartimenti}
\vspace{-.6cm}
\begin{lstlisting}
CREATE TABLE Dipartimenti(
    Codice          char(5),
    Descrizione     char(20) NOT NULL,
    Sede            char(20),
    Manager         smallint,
    PRIMARY KEY (Codice),
    UNIQUE(Descrizione),
    FOREIGN KEY (Manager) REFERENCES Impiegati(ID)
    |\colorbox{magenta!40}{\quad\quad ON DELETE SET NULL}|
        ON UPDATE CASCADE
);    
\end{lstlisting}
\begin{block}{Cancellazione}
    La cancellazione di una riga di \textit{Impiegati} implica che i valori di \textit{Manager} associati all'\textit{ID} di quella riga assumano valore nullo.
\end{block}
\end{frame}
%
\begin{frame}[fragile]{Osservazioni CREATE TABLE: Dipartimenti}
\vspace{-.6cm}
\begin{lstlisting}
CREATE TABLE Dipartimenti(
    Codice          char(5),
    Descrizione     char(20) NOT NULL,
    Sede            char(20),
    Manager         smallint,
    PRIMARY KEY (Codice),
    UNIQUE(Descrizione),
    FOREIGN KEY (Manager) REFERENCES Impiegati(ID)
        ON DELETE SET NULL
        |\colorbox{magenta!40}{ON UPDATE CASCADE}|
);    
\end{lstlisting}
\begin{block}{Aggiornamento}
    L'aggiornamento di un \textit{ID} associato a \textit{Manager} si riflette a catena sui valori di \textit{Manager}.
\end{block}
\end{frame}
%
\begin{frame}[fragile]{DESCRIBE TABLE: Dipartimenti}
\begin{lstlisting}
DESCRIBE Dipartimenti;
\end{lstlisting}
\begin{tabular}{|c|c|c|c|c|c|}
    \hline
    \rowcolor{cyan!30} Field & Type & Null & Key & Default & Extra \\
    \hline
    Codice & char(5) & NO & PRI & NULL & \\ \hline
    Descrizione & char(20) & NO & UNI & NULL & \\ \hline
    Sede & char(20) & YES & & NULL & \\ \hline
    Manager & smallint & YES & MUL & NULL & \\ \hline
\end{tabular}
\end{frame}
%
\begin{frame}[fragile]{Modificare le tabelle}
La struttura di una tabella pu\`o essere modificata in un secondo momento con il comando \textbf{ALTER TABLE} per:
\begin{itemize}
    \item aggiungere una nuova colonna (\textbf{ADD}) a quelle gi\`a esistenti;
    \begin{lstlisting}
        ALTER TABLE nomeTabella
        ADD nomeColonna tipoDato;
    \end{lstlisting}
    \item oppure per togliere una colonna (\textbf{DROP}).
    \begin{lstlisting}
        ALTER TABLE nomeTabella
        DROP nomeColonna;
    \end{lstlisting}
\end{itemize}
\end{frame}
%
\begin{frame}[fragile]{DDL: ALTER TABLE - ADD Colonna}
Il seguente comando inserisce in \textit{Impiegati} una nuova colonna di nome \textit{Nascita} per registrare la data di nascita del dipendente:
\begin{lstlisting}
ALTER TABLE Impiegati
ADD Nascita date;
\end{lstlisting}
\end{frame}
%
\begin{frame}[fragile]{DDL: ALTER TABLE - DROP Colonna}
Il seguente comando elimina da \textit{Impiegati} la colonna di nome \textit{Residenza}.
\begin{lstlisting}
ALTER TABLE Impiegati
DROP Residenza;
\end{lstlisting}
\begin{minipage}{0.7\textwidth}
    \begin{alertblock}{Attenzione}
        Il comando \textbf{DROP} elimina anche i dati contenuti nella colonna!
    \end{alertblock}
\end{minipage}
\end{frame}
%
\begin{frame}[fragile]{DDL: ALTER TABLE - RENAME Colonna}
Il seguente comando permette di rinominare una colonna di una tabella:
\begin{lstlisting}
ALTER TABLE nomeTabella
RENAME COLUMN vecchioNomeColonna to nuovoNomeColonna;
\end{lstlisting}
\end{frame}
%
\begin{frame}[fragile]{DDL: ALTER TABLE - MODIFY tipoDato Colonna}
Il seguente comando permette di modificare il tipo di dato di una colonna di una tabella:
\begin{lstlisting}
ALTER TABLE nomeTabella
MODIFY COLUMN nomeColonna tipoDato;
\end{lstlisting}
\end{frame}
%
\begin{frame}[fragile]{Eliminare il database}
I database vengono eliminati con il comando comando \textbf{DROP DATABASE}, seguito dal nome del database:
\begin{lstlisting}
DROP DATABASE nomeDB;
\end{lstlisting}
\end{frame}
%
\begin{frame}[fragile]{Eliminare le tabelle}
Il comando \textbf{DROP TABLE} elimina una tabella dal database:
\begin{lstlisting}
DROP TABLE nomeTabella;
\end{lstlisting}
\end{frame}
%
\begin{frame}[fragile]{DDL: DROP TABLE}
Il seguente comando elimina la tabella \textit{Impiegati} dal database:
\begin{lstlisting}
DROP TABLE Impiegati;
\end{lstlisting}
\end{frame}
%
%
\section{Comandi per la Manipolazione dei Dati}
\begin{frame}{Manipolare i dati}
    \onslide<1->{I valori degli attributi nelle righe della tabella possono essere:
    \begin{itemize}
        \item inseriti (\textbf{INSERT})
        \item aggiornati (\textbf{UPDATE})
        \item cancellati (\textbf{DELETE})
    \end{itemize}}
    \onslide<2->{
    \begin{alertblock}{La clausola WHERE nel DML}
        \`E importante notare che nei comandi \textbf{UPDATE} e \textbf{DELETE} compare la clausola \textbf{WHERE} per effetto della quale \`e possibile operare su molte righe, anzich\'e su una sola riga per volta.
    \end{alertblock}
    }
    \onslide<3>{
    \begin{alertblock}{ATTENZIONE al WHERE}
        La clausola \textbf{WHERE} specifica quali records dovrebbero essere aggiornati. Se essa viene omessa, verranno aggiornati tutti i record presenti nella tabella!
    \end{alertblock}
    }
\end{frame}
%
\begin{frame}[fragile]{DML: INSERT}
Il comando \textbf{INSERT INTO} viene usato per inserire nuovi records nella tabella.

\`E possibile utilizzarlo in 2 modi:
\begin{enumerate}
\item Specificando i nomi delle colonne e dei valori da inserire per tali:
\begin{lstlisting}
INSERT INTO nomeTabella (colonna1, colonna2, colonna3, ...)
VALUES (valore1, valore2, valore3, ...);
\end{lstlisting}
\item Senza specificare i nomi delle colonne quando si vogliono aggiungere i valori per tutte le colonne della tabella.
\begin{lstlisting}
INSERT INTO nomeTabella
VALUES (valore1, valore2, valore3, ...);
\end{lstlisting}
\end{enumerate}
\end{frame}
%
\begin{frame}[fragile]{INSERT: Impiegati}
\begin{itemize}
\item Inserimento di \textbf{una riga} nella tabella Impiegati:
\begin{lstlisting}
INSERT INTO Impiegati
(ID, Nome, Cognome, Residenza, Stipendio, Dipartimento)
VALUES(20, `Mario', `Rossi', `Camerino', 99999, `R&S');
\end{lstlisting}
\item Inserimento di \textbf{pi\`u righe} nella tabella Impiegati:
\begin{lstlisting}
INSERT INTO Impiegati
(ID, Nome, Cognome, Residenza, Stipendio, Dipartimento)
VALUES
(20, `Mario', `Rossi', `Camerino', 99999, `R&S'),
(77, `Luigi', `Verdi', `Castelraimondo', 1234, `Magazzino');
\end{lstlisting}
\end{itemize}
\end{frame}
%
\begin{frame}[fragile]{DML: UPDATE}
Il comando \textbf{UPDATE} viene usato per modificare records esistenti nella tabella.

\begin{lstlisting}
UPDATE nomeTabella
SET colonna1 = valore1, colonna2 = valore2, ...
WHERE condizione;
\end{lstlisting}
\end{frame}
%
\begin{frame}[fragile]{UPDATE: Impiegati}
Assegnamento del dipendente con ID=20 al dipartimento \textit{Produzione}:
\pause
\begin{lstlisting}
UPDATE Impiegati
SET Dipartimento = `Prod'
WHERE ID = 20;
\end{lstlisting}
\end{frame}
%
\begin{frame}[fragile]{UPDATE: Impiegati}
Aumento stipendio del 5\% ai dipendenti del dipartimento \textit{Produzione}:
\pause
\begin{lstlisting}
UPDATE Impiegati
SET Stipendio = Stipendio * 1.05
WHERE Dipartimento = `Prod';
\end{lstlisting}
\end{frame}
%
\begin{frame}[fragile]{DML: DELETE}
Il comando \textbf{DELETE} viene usato per cancellare records nella tabella.

\begin{lstlisting}
DELETE FROM nomeTabella WHERE condizione;
\end{lstlisting}
\end{frame}
%
\begin{frame}[fragile]{DELETE: Impiegati}
Cancellazione del dipendente con ID = 20 dalla tabella \textit{Impiegati}:
\pause
\begin{lstlisting}
DELETE FROM Impiegati
WHERE ID = 20;
\end{lstlisting}
\end{frame}
%
\begin{frame}[fragile]{DELETE: Impiegati}
Cancellazione dei dipendenti del dipartimento di Ricerca \& Sviluppo:
\pause
\begin{lstlisting}
DELETE FROM Impiegati
WHERE Dipartimento = `R&S';
\end{lstlisting}
\end{frame}
%
\begin{frame}[fragile]{DELETE: Impiegati}
Cosa produce il seguente comando?
\begin{lstlisting}
DELETE FROM Impiegati;
\end{lstlisting}
\pause
\begin{alertblock}{DELETE senza WHERE}
Manca la clausola \textbf{WHERE}: il comando svuota la tabella \textit{Impiegati} dai dati, ma rimane la sua struttura.
\end{alertblock}
\pause
{\begin{block}{Svuotare le tabelle}
Per svuotare una tabella \`e possibile utilizzare DELETE FROM nomeTabella. Tuttavia, la maniera pi\`u corretta \`e utilizzare il comando \textbf{TRUNCATE TABLE} nomeTabella, il quale elimina i dati ma mantiene la struttura della tabella.
\end{block}
\begin{lstlisting}
TRUNCATE TABLE Impiegati;
\end{lstlisting}
}
\end{frame}
%
%
\section*{Clausole SQL}
\begin{frame}{Clausole di SQL}
Sintesi delle diverse clausole che possono apparire nel comando \textbf{SELECT}:
\newline
\\\textbf{SELECT}~~~~~~~Elenco espressioni da mostrare
\\\textbf{FROM}~~~~~~~~~Tabelle da cui estrarre le righe
\\\textbf{WHERE}~~~~~~~Condizioni sulle righe estratte
\\\textbf{GROUP BY}~~Campi da considerare per i raggruppamenti
\\\textbf{HAVING}~~~~~~~Condizioni sui raggruppamenti
\\\textbf{ORDER BY}~~~Ordinamenti sulle espressioni elencate nella clausola SELECT
\end{frame}
%
\begin{frame}{Precisazioni sulle clausole SQL}
\`E opportuno fare le seguenti precisazioni:
\begin{itemize}[<+->]
    \item si possono scrivere comandi del tipo: \textbf{SELECT NOW();} oppure \textbf{SELECT 3*4;} nei quali compare la sola clausola SELECT.
    \item Per estrarre dati da una tabella le uniche clausole che devono comparire \textbf{obbligatoriamente} sono \textbf{SELECT} e \textbf{FROM} mentre le altre sono facoltative.
    \item Le clausole utilizzate devono essere elencate rispettando l'ordine (slide precedente):
    SELECT deve precedere FROM, FROM deve precedere WHERE, \ldots, HAVING deve precedere ORDER BY.
\end{itemize}
\end{frame}
%
\begin{frame}{Precisazioni sulle clausole SQL}
\begin{itemize}
    \item Il DBMS esegue il comando SQL elaborando le clausole nel seguente ordine:
\end{itemize}
\begin{center}
\begin{tabular}{|c|c|c|}
    \hline
    \rowcolor{cyan!30} \textbf{Ordine} & \textbf{Clausola} & \textbf{Funzione} \\
    \hline
    1 & FROM & Tabelle da cui estrarre le righe\\ \hline
    2 & WHERE & Condizioni sulle righe estratte\\ \hline
    3 & GROUP BY & Campi da considerare per i raggruppamenti\\ \hline
    4 & HAVING & Condizioni sui raggruppamenti\\ \hline
    5 & SELECT & Elenco espressioni da mostrare\\ \hline
    6 & ORDER BY & Ordinamenti sulle espressioni elencate nella clausola SELECT\\ \hline
    7 & LIMIT & I dati restituiti sono limitati al conteggio delle righe\\ \hline
\end{tabular}
\end{center}
\end{frame}
%
\section{Comandi per il Controllo dei Dati}
\begin{frame}[fragile]{Definire i permessi nel database}
\onslide<1->{I comandi di tipo \textbf{Data Control Language (DCL)} hanno lo scopo di garantire la sicurezza agli oggetti del database.}
\newline
\\\onslide<2->{Nei DBMS SQL ogni operazione deve essere autorizzata, ovvero l'utente che esegue l'operazione deve avere i privilegi necessari.}
\newline
\\\onslide<3->{Il database administrator (DBA) pu\`o:
\begin{itemize}
    \item concedere, con il comando \textbf{GRANT}
    \item o revocare, con il comando \textbf{REVOKE},
\end{itemize}
il diritto di eseguire certe azioni su determinati oggetti del database a specifici (o tutti gli) utenti.}
\end{frame}
%
\begin{frame}[fragile]{DCL: GRANT}
Il comando \textbf{GRANT} concede i permessi, specificando il tipo di accesso, per le tabelle sulle quali \`e consentito l'accesso e l'elenco degli utenti ai quali \`e permesso di accedere.
\newline
\\ Il tipo di accesso pu\`o riguardare, per esempio:
\begin{itemize}[<+->]
    \item il diritto di modifica della struttura della tabella con l'aggiunta di nuove colonne;
    \item oppure di modifica dei dati contenuti nella tabella;
    \item oppure l'uso del comando SELECT
    \item \ldots
\end{itemize}
\end{frame}
%
\begin{frame}[fragile]{DCL: GRANT - privilegi per Schemi}
\begin{itemize}
\item Sintassi di GRANT per concedere privilegi su schemi:
\begin{lstlisting}
GRANT < lista_privilegi >
ON SCHEMA < nome_schema >
TO { < lista_utenti e gruppi |$\vert$| PUBLIC > }
[ WITH GRANT OPTION ];
\end{lstlisting}
\item Possibili privilegi:
\\\textbf{CREATEIN}: per creare oggetti (tabelle, viste) nello schema;

\textbf{ALTERIN}:~~~per modificare la struttura di tabelle dello schema;

\textbf{DROPIN}:~~~~per eliminare oggetti dallo schema.
\end{itemize}
\end{frame}
%
\begin{frame}[fragile]{Esempio DCL: GRANT - privilegi per Schemi}
Il seguente comando DCL consente agli utenti USR7 e USR42 di creare e modificare oggetti nello schema DB12345:
\begin{lstlisting}
GRANT CREATEIN, ALTERIN
ON SCHEMA DB12345
TO USR7, USR42;
\end{lstlisting}
\end{frame}
%
\begin{frame}[fragile]{DCL: GRANT - privilegi per Tabelle e Viste}
Sintassi di GRANT per concedere privilegi su tabelle e viste:
\begin{lstlisting}
GRANT < ALL |$\vert$| lista_privilegi >
ON [ TABLE ] < nome_tabella >
TO { < lista_utenti e gruppi |$\vert$| PUBLIC > }
[ WITH GRANT OPTION ];
\end{lstlisting}
\end{frame}
%
\begin{frame}[fragile]{Esempio DCL: GRANT - privilegi per Tabelle}
Per concedere il diritto di modifica sulla tabella \textit{Impiegati} agli utenti User1 e User2:
\begin{lstlisting}
GRANT UPDATE
ON Impiegati
TO User1, User2;
\end{lstlisting}
\end{frame}
%
\begin{frame}[fragile]{DCL: REVOKE}
La revoca dei permessi con annullamento dei diritti di accesso viene effettuata con il comando \textbf{REVOKE} che ha una sintassi analoga al GRANT.
\end{frame}
\begin{frame}[fragile]{DCL: REVOKE - privilegi da Tabelle e Viste}
Sintassi di REVOKE per rimuovere privilegi da tabella e viste:
\begin{lstlisting}
REVOKE < ALL |$\vert$| lista_privilegi >
ON [ TABLE ] < nome_tabella >
FROM { < lista_utenti e gruppi |$\vert$| PUBLIC > };
\end{lstlisting}
\end{frame}
%
\begin{frame}[fragile]{Esempio DCL: REVOKE - privilegi da Tabelle}
Per revocare il permesso di modifica sulla tabella \textit{Impiegati} agli utenti User1 e User2:
\begin{lstlisting}
REVOKE UPDATE
ON Impiegati
FROM User1, User2;
\end{lstlisting}
\end{frame}
%
\begin{frame}[fragile]{DCL: Privilegi SQL}
I permessi che possono essere concessi (o revocati) agli utenti sono indicati con le seguenti parole chiave, le quali vanno specificate dopo GRANT o REVOKE:
\newline
\\\textbf{ALTER}:~~~~~ per aggiungere o eliminare colonne oppure per modificare tipi di dati;
\\\textbf{DELETE}:~~~ per eliminare righe dalle tabelle;
\\\textbf{INDEX}:~~~~~ per creare indici;
\\\textbf{INSERT}:~~~~per inserire nuove righe nelle tabelle;
\\\textbf{SELECT}:~~~ per ritrovare i dati  dalle tabelle;
\\\textbf{UPDATE}:~~ per cambiare i valori contenuti nelle tabelle;
\\\textbf{ALL}:~~~~~~~~~ per tutti i permessi precedenti.
\end{frame}
%
\section{Comandi per il Controllo delle Transazioni}
\begin{frame}[fragile]{Le transazioni}
Le transazioni sono un insieme di operazioni di interrogazione o di modifica dei dati del database che devono essere eseguite unitariamente e che godono delle propriet\`a \textbf{ACID}:
\begin{itemize}[<+->]
    \item \textbf{Atomicity}: una transazione \`e un' entit\`a atomica indivisibile. \`E compito del sistema di controllo della sicurezza riuscire a ripristinare la situazione preesistente;
    \item \textbf{Consistency}: le transazioni non devono violare i vincoli di integrit\`a dei dati. La consistenza \`e gestita dal DBMS con procedure opportune;
    \item \textbf{Isolation}: gli effetti di una transazione devono essere indipendenti da quello di tutte le altre transazioni eseguite correntemente. L'isolamento \`e gestito dal controllore della concorrenza;
    \item \textbf{Durability}: le informazioni in un database devono essere memorizzate in modo persistente. Gli effetti di una transazione eseguita con esito positivo devono essere memorizzati permanentemente nel database a cura del sottosistema di gestione della sicurezza e del ripristino dei dati.
\end{itemize}
\end{frame}
%
\begin{frame}{Transazioni in SQL}
Una transazione consiste in un insieme di comandi SQL, inseriti tra apposte dichiarazioni che definiscono l'inizio e la fine di una transazione.
\newline
\\I comandi SQL usati per definire e controllare le transazioni sono:
\begin{itemize}[<+->]
    \item \textbf{START TRANSACTION}: per iniziare una nuova transazione;
    \item \textbf{COMMIT}: per rendere definitivi i cambiamenti apportati al database nella transazione e terminarla;
    \item \textbf{ROLLBACK}: per annullare i cambiamenti eseguiti dall'inizio della transazione e terminarla;
    \item \textbf{SAVEPOINT}: per definire situazioni intermedie alle quali ritornare con un comando \textbf{ROLLBACK TO SAVEPOINT} che annulla solo parzialmente la transazione.
\end{itemize}
\end{frame}
%
\begin{frame}[fragile]{TCL: Esempio ROLLBACK}
Alla fine della transazione con \textbf{ROLLBACK}, le variazioni apportate al database dall'inizio delle transazione sono annullate.
\begin{lstlisting}
START TRANSACTION;

INSERT INTO bollette
VALUES( NULL, `2015-11-21', 100, 102.45, 1);

ROLLBACK;
\end{lstlisting}
\end{frame}
%
\begin{frame}[fragile]{TCL: Esempio COMMIT}
Con \textbf{COMMIT} si conferma l'inserimento della riga e le variazioni sono eseguite.
\begin{lstlisting}
START TRANSACTION;

INSERT INTO bollette
VALUES( NULL, `2015-11-21', 100, 102.45, 1);

COMMIT;
\end{lstlisting}
\end{frame}
%
\begin{frame}[fragile]{TCL: Esempio Transazioni concorrenti}
\begin{minipage}[t]{0.48\linewidth}
\textbf{T1:}
\begin{lstlisting}
START TRANSACTION;

UPDATE bollette
SET Importo=Importo-100
WHERE Numero=12;
\end{lstlisting}
\end{minipage}%
\hfill%
\begin{minipage}[t]{0.48\linewidth}
\textbf{T2:}
\begin{lstlisting}
START TRANSACTION;

UPDATE bollette
SET Importo=Importo-100
WHERE Numero=12;
\end{lstlisting}
\end{minipage}
\vspace{.9cm}

Quale sar\`a il risultato della tabella \textit{bollette} dopo che T1 e T2 vengono eseguite?
\end{frame}
%
\begin{frame}[fragile]{TCL: Esempio Transazioni concorrenti}
\begin{minipage}[t]{0.48\linewidth}
\textbf{T1:}
\begin{lstlisting}
START TRANSACTION;

UPDATE bollette
SET Importo=Importo-100
WHERE Numero=12;
\end{lstlisting}
\end{minipage}%
\hfill%
\begin{minipage}[t]{0.48\linewidth}
\textbf{T2:}
\begin{lstlisting}
START TRANSACTION;

UPDATE bollette
SET Importo=Importo-100
WHERE Numero=12;
\end{lstlisting}
\end{minipage}
\vspace{.5cm}

\begin{itemize}[<+->]
    \item T1 inizia e riduce di 100 l'importo della bolletta 12;
    \item T2 inizia. Non ``vede'' la variazione eseguita da T1;
    \item Per evitare perdite di aggiornamento, il record 12 non pu\`o essere modificato da T2 sino al termine di T1. MySQL blocca T2 in attesa per 50 secondi, poi abortisce il comando e va in errore in timeout.
\end{itemize}
\end{frame}
%
\begin{frame}[fragile]{TCL: Esempio Transazioni concorrenti}
\begin{minipage}[t]{0.48\linewidth}
\textbf{T1:}
\begin{lstlisting}
COMMIT;
\end{lstlisting}
\end{minipage}%
\hfill%
\begin{minipage}[t]{0.48\linewidth}
\textbf{T2:}
\begin{lstlisting}
UPDATE bollette
SET Importo=Importo-100
WHERE Numero=12;

COMMIT;
\end{lstlisting}
\end{minipage}
\vspace{.5cm}

\begin{itemize}[<+->]
    \item Completata la transazione T1, l'aggiornamento di T2 pu\`o essere eseguito;
    \item I due aggiornamenti sono stati eseguiti in sequenza e il valore dell'importo al termine dei due aggiornamenti \`e quello corretto.
\end{itemize}
\end{frame}
\section{Conclusioni}

\begin{frame}{Domande?}
    \begin{figure}
\centering
    \includegraphics[width=0.75\textwidth]{../img/questions.jpg}
\end{figure}
\end{frame}

\begin{frame}{Fine}
    \centering
    \huge Grazie dell'attenzione!
\end{frame}

\end{document}