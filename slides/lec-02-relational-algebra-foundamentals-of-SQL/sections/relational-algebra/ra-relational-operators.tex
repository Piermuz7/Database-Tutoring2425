\begin{frame}{Operazioni Relazionali}
    \begin{itemize}
        \item Agiscono su una o pi\`u relazioni per ottenere una nuova relazione;
        \item Consentono di effettuare le interrogazioni al database per ottenere le informazioni desiderate: \begin{itemize}
            \item estraendo da una tabella una sottotabella;
            \item combinando tra loro 2 o pi\`u tabelle;
            \item generando nuove relazioni. 
        \end{itemize}
        \item Dato che le relazioni sono insiemi:
        \begin{itemize}
            \item $X$ insieme di attributi su cui le relazioni $A$ e $B$ sono definite;
            \item \textbf{unione} $A~\mathbb{U}~B$: relazione su $X$ contente le tuple appartenenti a $A$ o $B$ o entrambe;
            \item \textbf{intersezione} $A~\cap~B$: relazione su $X$ contente le tuple appartenenti a $A$ e $B$;
            \item \textbf{differenza} $A~-~B$: relazione su $X$ contente le tuple appartenenti a $A$ ma non a $B$.
        \end{itemize}
    \end{itemize}
    \end{frame}
    %
    \begin{frame}{Operazioni Relazionali}
        \framesubtitle{Unione}
        \vspace{-1.4cm}
        \begin{align*}
        X &= \{Nome, Cognome\}\\
        Docenti(X) &= \{(Barbara,Re),(Roberto,Gagliardi),(Fausto,Marcantoni),\\&(Mario,Rossi)\}\\
            Studenti(X) &= \{(Mario,Rossi),(Luigi,Verdi),(Pinco,Pallino)\}
        \end{align*}
        \pause
        \onslide<1->{
            \begin{columns}
                \begin{column}{0.48\textwidth}
                    \centering
                    {\small Docenti}
                    \newline
            \begin{tabular}{|c|c|}
                \hline
                \rowcolor{cyan!30}Nome & Cognome \\
                \hline
                Barbara & Re \\ \hline
                Roberto & Gagliardi \\ \hline
                Fausto & Marcantoni \\ \hline
                Mario & Rossi \\ \hline
                \end{tabular}
                \end{column}
                \begin{column}{0.48\textwidth}
                    \centering
                    {\small Studenti}
                    \newline
                    \begin{tabular}{|c|c|}
                        \hline
                        \rowcolor{cyan!30} Nome & Cognome \\ \hline
                        Mario & Rossi \\
                        \hline
                        Luigi & Verdi \\ \hline
                        Pinco & Pallino \\ \hline
                        \end{tabular}
                \end{column}
            \end{columns}
        }
        \vspace*{0.2cm}
        \onslide<2> \centering Docenti~$\mathbb{U}~Studenti = ?$
    \end{frame}
    %
    \begin{frame}{Operazioni Relazionali}
        \framesubtitle{Unione}
        \vspace{-1cm}
        \begin{align*}Docenti(X) &= \{(Barbara,Re),(Roberto,Gagliardi),(Fausto,Marcantoni),\\&(Mario,Rossi)\}\\
        Studenti(X) &= \{(Mario,Rossi),(Luigi,Verdi),(Pinco,Pallino)\}\\
        Docenti~\mathbb{U}~Studenti &= \{(Barbara,Re),(Roberto,Gagliardi),(Fausto,Marcantoni),\\&(Mario,Rossi),(Luigi,Verdi),(Pinco,Pallino)\}
        \end{align*}
        \centering
        $Docenti~\mathbb{U}~Studenti$\\
        \begin{tabular}{|c|c|}
            \hline
            \rowcolor{cyan!30}Nome & Cognome \\
            \hline
            Barbara & Re \\ \hline
            Roberto & Gagliardi \\ \hline
            Fausto & Marcantoni \\ \hline
            \rowcolor{red!30}Mario & Rossi \\ \hline
            Luigi & Verdi \\ \hline
            Pinco & Pallino \\ \hline
        \end{tabular}
    \end{frame}
    %
    \begin{frame}{Operazioni Relazionali}
        \framesubtitle{Intersezione}
        \vspace{-1.4cm}
        \begin{align*}
        X &= \{Nome, Cognome\}\\
        Docenti(X) &= \{(Barbara,Re),(Roberto,Gagliardi),(Fausto,Marcantoni),\\&(Mario,Rossi)\}\\
            Studenti(X) &= \{(Mario,Rossi),(Luigi,Verdi),(Pinco,Pallino)\}
            \end{align*}
        \pause
        \onslide<1->{
            \begin{columns}
                \begin{column}{0.48\textwidth}
                    \centering
                    {\small Docenti}
                    \newline
            \begin{tabular}{|c|c|}
                \hline
                \rowcolor{cyan!30}Nome & Cognome \\
                \hline
                Barbara & Re \\ \hline
                Roberto & Gagliardi \\ \hline
                Fausto & Marcantoni \\ \hline
                Mario & Rossi \\ \hline
                \end{tabular}
                \end{column}
                \begin{column}{0.48\textwidth}
                    \centering
                    {\small Studenti}
                    \newline
                    \begin{tabular}{|c|c|}
                        \hline
                        \rowcolor{cyan!30} Nome & Cognome \\ \hline
                        Mario & Rossi \\
                        \hline
                        Luigi & Verdi \\ \hline
                        Pinco & Pallino \\ \hline
                        \end{tabular}
                \end{column}
            \end{columns}
        }
        \vspace*{0.2cm}
        \onslide<2> \centering Docenti~$\cap~Studenti = ?$
    \end{frame}
    %
    \begin{frame}{Operazioni Relazionali}
        \framesubtitle{Intersezione}
        \vspace{-1cm}
        \begin{align*}Docenti(X) &= \{(Barbara,Re),(Roberto,Gagliardi),(Fausto,Marcantoni),\\&(Mario,Rossi)\}\\
        Studenti(X) &= \{(Mario,Rossi),(Luigi,Verdi),(Pinco,Pallino)\}\\
        Docenti~\cap~Studenti &= \{(Mario,Rossi)\}
        \end{align*}
        \centering
        $Docenti~\cap~Studenti$\\
        \begin{tabular}{|c|c|}
            \hline
            \rowcolor{cyan!30}Nome & Cognome \\
            \hline
            Mario & Rossi \\ \hline
        \end{tabular}
    \end{frame}
    %
    \begin{frame}{Operazioni Relazionali}
        \framesubtitle{Differenza}
        \vspace{-1.4cm}
        \begin{align*}
        X &= \{Nome, Cognome\}\\
        Docenti(X) &= \{(Barbara,Re),(Roberto,Gagliardi),(Fausto,Marcantoni),\\&(Mario,Rossi)\}\\
            Studenti(X) &= \{(Mario,Rossi),(Luigi,Verdi),(Pinco,Pallino)\}
            \end{align*}
        \pause
        \onslide<1->{
            \begin{columns}
                \begin{column}{0.48\textwidth}
                    \centering
                    {\small Docenti}
                    \newline
            \begin{tabular}{|c|c|}
                \hline
                \rowcolor{cyan!30}Nome & Cognome \\
                \hline
                Barbara & Re \\ \hline
                Roberto & Gagliardi \\ \hline
                Fausto & Marcantoni \\ \hline
                Mario & Rossi \\ \hline
                \end{tabular}
                \end{column}
                \begin{column}{0.48\textwidth}
                    \centering
                    {\small Studenti}
                    \newline
                    \begin{tabular}{|c|c|}
                        \hline
                        \rowcolor{cyan!30} Nome & Cognome \\ \hline
                        Mario & Rossi \\
                        \hline
                        Luigi & Verdi \\ \hline
                        Pinco & Pallino \\ \hline
                        \end{tabular}
                \end{column}
            \end{columns}
        }
        \vspace*{0.2cm}
        \onslide<2> \centering $Docenti~-~Studenti = ?$
    \end{frame}
    %
    \begin{frame}{Operazioni Relazionali}
        \framesubtitle{Differenza}
        \vspace{-1cm}
        \begin{align*}Docenti(X) &= \{(Barbara,Re),(Roberto,Gagliardi),(Fausto,Marcantoni),(Mario,Rossi)\}\\
        Studenti(X) &= \{(Mario,Rossi),(Luigi,Verdi),(Pinco,Pallino)\}\\
        Docenti~-~Studenti &= \{(Mario,Rossi)\}
        \end{align*}
        \centering
        $Docenti~-~Studenti$\\
        \begin{tabular}{|c|c|}
            \hline
            \rowcolor{cyan!30}Nome & Cognome \\
            \hline
            Barbara & Re \\ \hline
            Roberto & Gagliardi \\ \hline
            Fausto & Marcantoni \\ \hline
        \end{tabular}
    \end{frame}
    %
    \begin{frame}{Operazioni Relazionali}
        \framesubtitle{Ridenominazione}
        L'operatore $\rho$ \textbf{Ridenominazione} viene utilizzato per rinominare uno o pi\`u attributi $A_1,\dotsc,A_n$ in $A'_1,\dotsc,A'_n$ di una relazione r.
        \[\rho_{A'_1,\dotsc,A'_n \leftarrow A_1,\dotsc,A_n} (r)\]
        \onslide<1->{
            \begin{columns}
                \begin{column}{0.48\textwidth}
                    \centering
                    {\small Cani}
                    \newline
            \begin{tabular}{|c|c|}
                \hline
                \rowcolor{cyan!30}NomeCane & RazzaCane \\
                \hline
                Alfred & Beagle \\ \hline
                Bruce & Chihuahua \\ \hline
                Tom & Bassotto \\ \hline
                \end{tabular}
                \end{column}
                \begin{column}{0.48\textwidth}
                    \centering
                    {\small Gatti}
                    \newline
                    \begin{tabular}{|c|c|}
                        \hline
                        \rowcolor{cyan!30} NomeGatto & RazzaGatto \\ \hline
                        Virgola & Siamese \\ \hline
                Joker & Persiano \\ \hline
                Elvis & Siberiano \\ \hline
                        \end{tabular}
                \end{column}
            \end{columns}
        }
        \vspace*{0.2cm}
        \onslide<2-> \centering $Cani~\mathbb{U}~Gatti = ?$
    
        \onslide<3> \centering Non si pu\`o fare.\\ \`E necessario usare $\rho$ per ridenominare NomeCane e NomeGatto! \\\`E necessario usare $\rho$ per ridenominare RazzaCane e RazzaGatto!
    \end{frame}
    %
    \begin{frame}{Operazioni Relazionali}
        \framesubtitle{Ridenominazione}
        \vspace{-1.2cm}
        \[\rho_{Nome,Razza \leftarrow NomeCane, RazzaCane} (Cani)\]
        \[\rho_{Nome,Razza \leftarrow NomeGatto, RazzaGatto} (Gatti)\]
        \vspace{-.4cm}
        \begin{columns}
            \begin{column}{0.48\textwidth}
                \centering
                {\small Cani}
                \newline
        \begin{tabular}{|c|c|}
            \hline
            \cellcolor{red!30}Nome & \cellcolor{green!30}Razza \\
            \hline
            Alfred & Beagle \\ \hline
            Bruce & Chihuahua \\ \hline
            Tom & Bassotto \\ \hline
            \end{tabular}
            \end{column}
            \begin{column}{0.48\textwidth}
                \centering
                {\small Gatti}
                \newline
                \begin{tabular}{|c|c|}
                    \hline
                    \cellcolor{red!30}Nome & \cellcolor{green!30}Razza \\ \hline
                    Virgola & Siamese \\ \hline
            Joker & Persiano \\ \hline
            Elvis & Siberiano \\ \hline
                    \end{tabular}
            \end{column}
        \end{columns}
        \centering
        $Cani~\mathbb{U}~Gatti$\\
        \begin{tabular}{|c|c|}
            \hline
            \rowcolor{cyan!30}Nome & Razza \\
            \hline
            Alfred & Beagle \\ \hline
            Bruce & Chihuahua \\ \hline
            Tom & Bassotto \\ \hline
            Virgola & Siamese \\ \hline
            Joker & Persiano \\ \hline
            Elvis & Siberiano \\ \hline
        \end{tabular}
    \end{frame}
    %
    \begin{frame}{Operazioni Relazionali}
        \framesubtitle{Selezione (Taglio orizzontale)}
        L'operatore $\sigma$ \textbf{Selezione} viene utilizzato su una relazione $r$ per produrre una nuova relazione in base a una condizione booleana $C$.
        \\La nuova relazione conterr\`a tutte le $n$-uple che soddisfano $C$.
        \[\sigma_C (r)\]
        \onslide<1->{
                    \begin{center}
                        {\small Studenti}
                        
                        \begin{tabular}{|c|c|c|}
                            \hline
                            \rowcolor{cyan!30}Nome & Cognome & Eta \\
                            \hline
                            Mario & Rossi & 19 \\ \hline
                            Luigi & Verdi & 22 \\ \hline
                            Pinco & Pallino & 21 \\ \hline
                            Mario & Verdini & 20 \\ \hline
                            \end{tabular}
                    
                    \end{center}
            }
        \vspace*{0.2cm}
        \onslide<2-> \centering Studenti con pi\`u di 20 anni?
    \end{frame}
    %
    \begin{frame}{Operazioni Relazionali}
        \framesubtitle{Selezione (Taglio orizzontale)}
        Studenti con pi\`u di 20 anni:
        \[\sigma_{Eta~>~20} (Studenti)\]
                    \begin{center}
                        {\small Studenti}
                        
                        \begin{tabular}{|c|c|c|}
                            \hline
                            \rowcolor{cyan!30}Nome & Cognome & Eta \\
                            \hline
                            Mario & Rossi & 19 \\ \hline
                            \rowcolor{red}Luigi & Verdi & 22 \\ \hline
                            \rowcolor{red}Pinco & Pallino & 21 \\ \hline
                            Mario & Verdini & 20 \\ \hline
                            \end{tabular}
                    
                    \end{center}
    \end{frame}
    %
    \begin{frame}{Operazioni Relazionali}
        \framesubtitle{Selezione (Taglio orizzontale)}
        \onslide<1->Studenti di nome ``Mario'' con pi\`u di 20 anni?
        \onslide<2->\[\sigma_{Nome=`Mario'~AND~ Eta~>~20} (Studenti)\]
        \onslide<3>{            \begin{center}
                        {\small Studenti}
                        
                        \begin{tabular}{|c|c|c|}
                            \hline
                            \rowcolor{cyan!30}Nome & Cognome & Eta \\
                            \hline
                            & & \\
                            \hline
                            \end{tabular}
    
                    \end{center}}
    \end{frame}
    %
    \begin{frame}{Operazioni Relazionali}
        \framesubtitle{Proiezione (Taglio verticale)}
        L'operatore $\pi$ \textbf{Proiezione} viene utilizzato su una relazione $r$ per produrre una nuova relazione in base a una lista di attributi $L$ specificata.
        \\La nuova relazione conterr\`a tutte le $n$-uple di $r$ con gli attributi presenti in $L$.
        \[\pi_L (r)\]
        \onslide<1->{
                    \begin{center}
                        {\small Studenti}
                        
                        \begin{tabular}{|c|c|c|}
                            \hline
                            \rowcolor{cyan!30}Nome & Cognome & Eta \\
                            \hline
                            Mario & Rossi & 19 \\ \hline
                            Luigi & Verdi & 22 \\ \hline
                            Pinco & Pallino & 21 \\ \hline
                            Mario & Verdini & 20 \\ \hline
                            \end{tabular}
                    
                    \end{center}
            }
        \vspace*{0.2cm}
        \onslide<2-> \centering Nome e Cognome degli studenti?
    \end{frame}
    %
    \begin{frame}{Operazioni Relazionali}
        \framesubtitle{Proiezione (Taglio verticale)}
        Nome e cognome degli studenti:
        \[\pi_{Nome,Cognome} (Studenti)\]
                    \begin{center}
                        {\small Studenti}
                        
                        \begin{tabular}{|c|c|c|}
                            \hline
                            \rowcolor{cyan!30}Nome & Cognome & Eta \\
                            \hline
                            \cellcolor{red}Mario & \cellcolor{red}Rossi & 19 \\ \hline
                            \cellcolor{red}Luigi & \cellcolor{red}Verdi & 22 \\ \hline
                            \cellcolor{red}Pinco & \cellcolor{red}Pallino & 21 \\ \hline
                            \cellcolor{red}Mario & \cellcolor{red}Verdini & 20 \\ \hline
                            \end{tabular}
                    
                    \end{center}
    \end{frame}
    %
    \begin{frame}{Operazioni Relazionali}
        \framesubtitle{Selezione (Taglio orizzontale) e Proiezione (Taglio verticale)}
        \onslide<1->Nome ed et\`a degli studenti con meno di 21 anni?
        \onslide<2->\[\pi_{Nome,Eta}\color{red}({\sigma_{Eta~<=~20} (Studenti)})\]
        \onslide<3>{            \begin{center}
                        {\small Studenti}
                        
                        \begin{tabular}{|c|c|c|}
                            \hline
                            \rowcolor{cyan!30}Nome & Cognome & Eta \\
                            \hline
                            \cellcolor{red}Mario & \cellcolor{red}Rossi & \cellcolor{red}19 \\ \hline
                            Luigi & Verdi & 22 \\ \hline
                            Pinco & Pallino & 21 \\ \hline
                            \cellcolor{red}Mario & \cellcolor{red}Verdini & \cellcolor{red}20 \\ \hline
                            \end{tabular}
    
                    \end{center}}
    \end{frame}
    %
    \begin{frame}{Operazioni Relazionali}
        \framesubtitle{Selezione (Taglio orizzontale) e Proiezione (Taglio verticale)}
        Nome ed et\`a degli studenti con meno di 21 anni?
        \[{\color{red}{\pi Nome,Eta}}(\sigma_{Eta~<=~20} (Studenti))\]
                    \begin{center}
                        {\small Studenti}
                        
                        \begin{tabular}{|c|c|c|}
                            \hline
                            \rowcolor{cyan!30}Nome & Eta \\
                            \hline
                            \cellcolor{red}Mario &  \cellcolor{red}19 \\ \hline
                            Luigi & 22 \\ \hline
                            Pinco & 21 \\ \hline
                            \cellcolor{red}Mario & \cellcolor{red}20 \\ \hline
                            \end{tabular}
    
                    \end{center}
    \end{frame}
    %
    \begin{frame}{Operazioni Relazionali}
        \framesubtitle{Join}
        L'operatore $\bowtie$ \textbf{Join} viene utilizzato per correlare dati in relazioni diverse basandosi sui valori degli attributi.
        \pause
        \\Esistono diversi tipi di join:
        \begin{itemize}
            \item \textbf{Natural Join};
            \item \textbf{Join completo, incompleto, vuoto};
            \item \textbf{Outer Join};
            \begin{itemize}
                \item Left outer Join;
                \item Right outer Join;
                \item Full outer Join.
            \end{itemize}
            \item \textbf{SemiJoin};
            \item \textbf{Theta Join};
            \item \textbf{Equi-Join}.
        \end{itemize}
    \end{frame}
    %
    \begin{frame}{Operazioni Relazionali}
        \framesubtitle{Natural Join}
        Il \textbf{natural join} correla i dati in 2 relazioni diverse sulla base di valori uguali negli attributi con lo stesso nome.
        \[ R_1 \bowtie R_2~=~\{ t \in~X_1~ \mathbb{U}~X_2~|~t[X_1] \in R_1 \land~t[X_2]~\in~R_2 \} \]
        \`E una relazione definita sull'unione degli insiemi degli attributi delle due relazioni $R_1(X_1)$ e $R_2(X_2)$ ossia $X_1~\mathbb{U}~X_2$.
        \begin{columns}
            \begin{column}{0.48\textwidth}
                \centering
                {\small $R_1$}
                \newline
        \begin{tabular}{|c|c|c|}
            \hline
            \rowcolor{cyan!30}Col1 & Col2 & Col3 \\
            \hline
            1 & a & b \\ \hline
            2 & c & d \\ \hline
            3 & e & f \\ \hline
            4 & g & h \\ \hline
            5 & g & j \\ \hline
            \end{tabular}
            \end{column}
            \begin{column}{0.48\textwidth}
                \centering
                {\small $R_2$}
                \newline
                \begin{tabular}{|c|c|}
                    \hline
                    \rowcolor{cyan!30} Col2 & Col4 \\ \hline
                    a & 50 \\ \hline
                    g & 120 \\ \hline
                    m & 345 \\ \hline
                    \end{tabular}
            \end{column}
        \end{columns}
    \end{frame}
    %
    \begin{frame}{Operazioni Relazionali}
        \framesubtitle{Natural Join}
        \begin{columns}
            \begin{column}{0.48\textwidth}
                \centering
                {\small $R_1$}
                \newline
        \begin{tabular}{|c|c|c|}
            \hline
            \rowcolor{cyan!30}Col1 & Col2 & Col3 \\
            \hline
            1 & \cellcolor{green!30}a & b \\ \hline
            2 & c & d \\ \hline
            3 & e & f \\ \hline
            4 & \cellcolor{red!30}g & h \\ \hline
            5 & \cellcolor{red!30}g & j \\ \hline
            \end{tabular}
            \end{column}
            \begin{column}{0.48\textwidth}
                \centering
                {\small $R_2$}
                \newline
                \begin{tabular}{|c|c|}
                    \hline
                    \rowcolor{cyan!30} Col2 & Col4 \\ \hline
                    \cellcolor{green!30}a & 50 \\ \hline
                    \cellcolor{red!30}g & 120 \\ \hline
                    m & 345 \\ \hline
                    \end{tabular}
            \end{column}
        \end{columns}
        \vspace{.5cm}
        \centering
        $R_1 \bowtie R_2$
        \begin{tabular}{|c|c|c|c|}
            \hline
            \rowcolor{cyan!30} Col1 & Col2 & Col3 & Col4 \\ \hline
            1 & a & b & 50 \\ \hline
            4 & g & h & 50 \\ \hline
            5 & g & j & 120 \\ \hline
            \end{tabular}
    \end{frame}
    %
    \begin{frame}{Operazioni Relazionali}
        \framesubtitle{Join Completo}
        Il Join \`e \textbf{completo} se ogni tupla di ciascuna relazione degli operandi contribuisce almeno a una tupla del risultato.
        \begin{columns}
            \begin{column}{0.48\textwidth}
                \centering
                {\small $R_1$}
                \newline
        \begin{tabular}{|c|c|}
            \hline
            \rowcolor{cyan!30}ID & Carattere \\
            \hline
            1 & \cellcolor{green!30}a \\ \hline
            2 & \cellcolor{red!30}b \\ \hline
            3 & \cellcolor{red!30}b \\ \hline
            \end{tabular}
            \end{column}
            \begin{column}{0.48\textwidth}
                \centering
                {\small $R_2$}
                \newline
                \begin{tabular}{|c|c|}
                    \hline
                    \rowcolor{cyan!30} Carattere & Numero \\ \hline
                    \cellcolor{green!30}a & Uno \\ \hline
                    \cellcolor{red!30}b & Due \\ \hline
                    \end{tabular}
            \end{column}
        \end{columns}
        \vspace{.5cm}
        \centering
        $R_1 \bowtie R_2$
        \begin{tabular}{|c|c|c|c|}
            \hline
            \rowcolor{cyan!30} ID & Carattere & Numero \\ \hline
            1 & a & Uno \\ \hline
            2 & b & Due \\ \hline
            3 & b & Due \\ \hline
            \end{tabular}
    \end{frame}
    %
    \begin{frame}{Operazioni Relazionali}
        \framesubtitle{Join Incompleto}
        Il Join \`e \textbf{incompleto} se almeno una tupla delle relazioni degli operandi non contribuisce al risultato.
        \begin{columns}
            \begin{column}{0.48\textwidth}
                \centering
                {\small $R_1$}
                \newline
        \begin{tabular}{|c|c|}
            \hline
            \rowcolor{cyan!30}ID & Carattere \\
            \hline
            1 & \cellcolor{green!30}a \\ \hline
            2 & b \\ \hline
            3 & b \\ \hline
            \end{tabular}
            \end{column}
            \begin{column}{0.48\textwidth}
                \centering
                {\small $R_2$}
                \newline
                \begin{tabular}{|c|c|}
                    \hline
                    \rowcolor{cyan!30} Carattere & Numero \\ \hline
                    \cellcolor{green!30}a & Uno \\ \hline
                    \cellcolor{red}c & Due \\ \hline
                    \end{tabular}
            \end{column}
        \end{columns}
        \vspace{.5cm}
        \centering
        $R_1 \bowtie R_2$
        \begin{tabular}{|c|c|c|c|}
            \hline
            \rowcolor{cyan!30} ID & Carattere & Numero \\ \hline
            1 & a & Uno \\ \hline
            \end{tabular}
    \end{frame}
    %
    \begin{frame}{Operazioni Relazionali}
        \framesubtitle{Join Esterno}
        Il \textbf{join esterno} (outer join) include tutte le tuple di una relazione estese con le tuple dell'altra relazione che rispettano la condizione di join.
        \\Gli attributi delle tuple che non rispettano la condizione di join sono riempite con i valori \textbf{NULL}.
        \newline
        3 tipi di join esterno:
        \begin{itemize}
            \item \textbf{sinistro}: include tutte le tuple della prima relazione
            \[R_1 \leftouterjoin R_2\]
            \item \textbf{destro}: include tutte le tuple della seconda relazione
            \[R_1\rightouterjoin R_2\]
            \item \textbf{completo}: include tutte le tuple della prima e della seconda relazione
            \[R_1\fullouterjoin R_2\]
        \end{itemize}
    \end{frame}
    %
    \begin{frame}{Operazioni Relazionali}
        \framesubtitle{Left Join}
        Il \textbf{Left join} include tutte le tuple della prima relazione.
        \begin{columns}
            \begin{column}{0.48\textwidth}
                \centering
                {\small $R_1$}
                \newline
        \begin{tabular}{|c|c|}
            \hline
            \rowcolor{cyan!30}ID & Carattere \\
            \hline
            1 & a \\ \hline
            2 & \cellcolor{green}b \\ \hline
            3 & \cellcolor{green}b \\ \hline
            \end{tabular}
            \end{column}
            \begin{column}{0.48\textwidth}
                \centering
                {\small $R_2$}
                \newline
                \begin{tabular}{|c|c|}
                    \hline
                    \rowcolor{cyan!30} Carattere & Numero \\ \hline
                    a & Uno \\ \hline
                    \cellcolor{red}c & Due \\ \hline
                    \end{tabular}
            \end{column}
        \end{columns}
        \vspace{.5cm}
        \centering
        \pause
        {\small $R_1 \leftouterjoin R_2$}
        \begin{tabular}{|c|c|c|c|}
            \hline
            \rowcolor{cyan!30} ID & Carattere & Numero \\ \hline
            1 & a & Uno \\ \hline
            2 & b & NULL \\ \hline
            3 & b & NULL \\ \hline
            \end{tabular}
    \end{frame}
    %
    \begin{frame}{Operazioni Relazionali}
        \framesubtitle{Right Join}
        Il \textbf{Right join} include tutte le tuple della seconda relazione.
        \begin{columns}
            \begin{column}{0.48\textwidth}
                \centering
                {\small $R_1$}
                \newline
        \begin{tabular}{|c|c|}
            \hline
            \rowcolor{cyan!30}ID & Carattere \\
            \hline
            1 & a \\ \hline
            2 & \cellcolor{red}b \\ \hline
            3 & \cellcolor{red}b \\ \hline
            \end{tabular}
            \end{column}
            \begin{column}{0.48\textwidth}
                \centering
                {\small $R_2$}
                \newline
                \begin{tabular}{|c|c|}
                    \hline
                    \rowcolor{cyan!30} Carattere & Numero \\ \hline
                    a & Uno \\ \hline
                    \cellcolor{green}c & Due \\ \hline
                    \end{tabular}
            \end{column}
        \end{columns}
        \vspace{.5cm}
        \centering
        \pause
        {\small $R_1 \rightouterjoin R_2$}
        \begin{tabular}{|c|c|c|c|}
            \hline
            \rowcolor{cyan!30} ID & Carattere & Numero \\ \hline
            1 & a & Uno \\ \hline
            NULL & c & Due \\ \hline
            \end{tabular}
    \end{frame}
    %
    \begin{frame}{Operazioni Relazionali}
        \framesubtitle{Full Join}
        Il \textbf{Full join} include tutte le tuple della prima e della seconda relazione.
        \begin{columns}
            \begin{column}{0.48\textwidth}
                \centering
                {\small $R_1$}
                \newline
        \begin{tabular}{|c|c|}
            \hline
            \rowcolor{cyan!30}ID & Carattere \\
            \hline
            1 & a \\ \hline
            2 & \cellcolor{red}b \\ \hline
            3 & \cellcolor{red}b \\ \hline
            \end{tabular}
            \end{column}
            \begin{column}{0.48\textwidth}
                \centering
                {\small $R_2$}
                \newline
                \begin{tabular}{|c|c|}
                    \hline
                    \rowcolor{cyan!30} Carattere & Numero \\ \hline
                    a & Uno \\ \hline
                    \cellcolor{red}c & Due \\ \hline
                    \end{tabular}
            \end{column}
        \end{columns}
        \vspace{.5cm}
        \centering
        \pause
        {\small $R_1 \fullouterjoin R_2$}
        \begin{tabular}{|c|c|c|c|}
            \hline
            \rowcolor{cyan!30} ID & Carattere & Numero \\ \hline
            1 & a & Uno \\ \hline
            \rowcolor{red}2 & b & NULL \\ \hline
            \rowcolor{red}3 & b & NULL \\ \hline
            \rowcolor{red}NULL & c & Due \\ \hline
            \end{tabular}
    \end{frame}
    %
    \begin{frame}{Operazioni Relazionali}
        \framesubtitle{Join Vuoto}
        Il Join \`e \textbf{vuoto} se nessuna tupla delle relazioni degli operandi contribuisce al risultato finale.
        \begin{columns}
            \begin{column}{0.48\textwidth}
                \centering
                {\small $R_1$}
                \newline
        \begin{tabular}{|c|c|}
            \hline
            \rowcolor{cyan!30}ID & Carattere \\
            \hline
            1 & a \\ \hline
            2 & b \\ \hline
            3 & b \\ \hline
            \end{tabular}
            \end{column}
            \begin{column}{0.48\textwidth}
                \centering
                {\small $R_2$}
                \newline
                \begin{tabular}{|c|c|}
                    \hline
                    \rowcolor{cyan!30} Carattere & Numero \\ \hline
                    \cellcolor{red}c & Uno \\ \hline
                    \cellcolor{red}d & Due \\ \hline
                    \end{tabular}
            \end{column}
        \end{columns}
        \vspace{.5cm}
        \centering
        \pause
        $R_1 \bowtie R_2~=~\emptyset$
        \begin{tabular}{|c|c|c|c|}
            \hline
            \rowcolor{cyan!30} ID & Carattere & Numero \\ \hline
            & & \\ \hline
            \end{tabular}
    \end{frame}
    %
    \begin{frame}{Operazioni Relazionali}
        \framesubtitle{Natural Join senza attributi in comune}
        Se nel natural join non ci sono attributi in comune, il risultato \`e il \textbf{prodotto cartesiano} delle relazioni $R_1~\times~R_2$.
        \begin{columns}
            \begin{column}{0.48\textwidth}
                \centering
                {\small $R_1$}
                \newline
        \begin{tabular}{|c|c|}
            \hline
            \rowcolor{cyan!30}ID & Carattere$R_1$ \\
            \hline
            1 & a \\ \hline
            2 & b \\ \hline
            3 & b \\ \hline
            \end{tabular}
            \end{column}
            \begin{column}{0.48\textwidth}
                \centering
                {\small $R_2$}
                \newline
                \begin{tabular}{|c|c|}
                    \hline
                    \rowcolor{cyan!30} Carattere$R_2$ & Numero \\ \hline
                    a & Uno \\ \hline
                    c & Due \\ \hline
                    \end{tabular}
            \end{column}
        \end{columns}
        \vspace{.1cm}
        \centering
        \pause
        {\small $R_1 \times R_2$
        \begin{tabular}{|c|c|c|c|}
            \hline
            \rowcolor{cyan!30} ID & Carattere$R_1$ & Carattere$R_2$ & Numero \\ \hline
            1 & a & a & Uno \\ \hline
            1 & a & c & Due \\ \hline
            2 & b & a & Uno \\ \hline
            2 & b & c & Due \\ \hline
            3 & b & a & Uno \\ \hline
            3 & b & c & Due \\ \hline
            \end{tabular}}
    \end{frame}
    %
    \begin{frame}{Operazioni Relazionali}
        \framesubtitle{Natural Join con attributi in comune (SemiJoin)}
        Il natural join con tutti gli attributi in comune viene detto \textbf{SemiJoin} ed il risultato \`e l'intersezione delle relazioni $R_1~\cap~R_2$ su $X_1$.
        \begin{columns}
            \begin{column}{0.48\textwidth}
                \centering
                {\small $R_1$}
                \newline
        \begin{tabular}{|c|c|}
            \hline
            \rowcolor{cyan!30}ID & Carattere \\
            \hline
            1 & \cellcolor{red!30}a \\ \hline
            2 & b \\ \hline
            3 & b \\ \hline
            \end{tabular}
            \end{column}
            \begin{column}{0.48\textwidth}
                \centering
                {\small $R_2$}
                \newline
                \begin{tabular}{|c|c|}
                    \hline
                    \rowcolor{cyan!30} Carattere & Numero \\ \hline
                    \cellcolor{red!30}a & Uno \\ \hline
                    c & Due \\ \hline
                    \end{tabular}
            \end{column}
        \end{columns}
        \vspace{.1cm}
        \centering
        \pause
        {\small $(R_1 \cap R_2)(X_1)$
        \begin{tabular}{|c|c|}
            \hline
            \rowcolor{cyan!30} ID & Carattere \\ \hline
            1 & a \\ \hline
            \end{tabular}}
    \end{frame}
    %
    \begin{frame}{Operazioni Relazionali}
        \framesubtitle{Theta Join (Join condizionale)}
        Il \textbf{theta join} correla i dati in due relazioni diverse sulla base di una condizione booleana $C$\\(AND, OR, NOT, $>,~<,~<=,~>=$).
        \[ R_1 \bowtie_C R_2\]
        \begin{columns}
            \begin{column}{0.48\textwidth}
                \centering
                {\small $Supereroi$}
                \newline
        \begin{tabular}{|c|c|}
            \hline
            \rowcolor{cyan!30}Nome & Film \\
            \hline
            Bruce Wayne & The Batman \\ \hline
            Bruce Wayne & Batman Begins \\ \hline
            Peter Parker & Spiderman \\ \hline
            Clark Kent & Superman\\ \hline
            \end{tabular}
            \end{column}
            \begin{column}{0.48\textwidth}
                \centering
                {\small $Film$}
                \newline
                \begin{tabular}{|c|c|c|}
                    \hline
                    \rowcolor{cyan!30} Titolo & AnnoUscita & Durata \\ \hline
                    The Batman & 2022 & 176 \\ \hline
                    Spiderman & 2002 & 121 \\ \hline
                    Superman & 1978 & 151 \\ \hline
                    \end{tabular}
            \end{column}
        \end{columns}
    \end{frame}
    %
    \begin{frame}{Operazioni Relazionali}
        \framesubtitle{Theta Join (Join condizionale)}
        \vspace{-1.2cm}
        \[ Supereroi \bowtie_{Film~=~Titolo~AND~Durata~<~160} Film\]
        \vspace{-.5cm}
        \begin{columns}
            \begin{column}{0.48\textwidth}
                \centering
                {\small $Supereroi$}
                \newline
        \begin{tabular}{|c|c|}
            \hline
            \rowcolor{cyan!30}Nome & Film \\
            \hline
            Bruce Wayne & The Batman \\ \hline
            Bruce Wayne & Batman Begins \\ \hline
            Peter Parker & Spiderman \\ \hline
            Clark Kent & Superman\\ \hline
            \end{tabular}
            \end{column}
            \begin{column}{0.48\textwidth}
                \centering
                {\small $Film$}
                \newline
                \begin{tabular}{|c|c|c|}
                    \hline
                    \rowcolor{cyan!30} Titolo & AnnoUscita & Durata \\ \hline
                    The Batman & 2022 & 176 \\ \hline
                    Spiderman & 2002 & 121 \\ \hline
                    Superman & 1978 & 151 \\ \hline
                    \end{tabular}
            \end{column}
        \end{columns}
        \vspace{.5cm}
        \centering
        \pause
        {\small $Supereroi \bowtie_{Film~=~Titolo~AND~Durata~<~160} Film$
        \begin{tabular}{|c|c|c|c|c|}
            \hline
            \rowcolor{cyan!30} Nome & Film & Titolo & AnnoUscita & Durata\\ \hline
            Peter Parker & Spiderman & Spiderman & 2002 & 121 \\ \hline
            Clark Kent & Superman & Superman & 1978 & 151\\ \hline
            \end{tabular}}
    \end{frame}
    %
    \begin{frame}{Operazioni Relazionali}
        \framesubtitle{Equi-Join}
        Se la condizione \`e composta da operatori di uguaglianza (=), eventualmente in congiunzione (AND) tra loro, il theta join \`e detto \textbf{equi-join}.
        \begin{columns}
            \begin{column}{0.48\textwidth}
                \centering
                {\small $Supereroi$}
                \newline
        \begin{tabular}{|c|c|}
            \hline
            \rowcolor{cyan!30}Nome & Film \\
            \hline
            Bruce Wayne & The Batman \\ \hline
            Bruce Wayne & Batman Begins \\ \hline
            Peter Parker & Spiderman \\ \hline
            Clark Kent & Superman\\ \hline
            \end{tabular}
            \end{column}
            \begin{column}{0.48\textwidth}
                \centering
                {\small $Film$}
                \newline
                \begin{tabular}{|c|c|c|}
                    \hline
                    \rowcolor{cyan!30} Titolo & AnnoUscita & Durata \\ \hline
                    The Batman & 2022 & 176 \\ \hline
                    Spiderman & 2002 & 121 \\ \hline
                    Superman & 1978 & 151 \\ \hline
                    \end{tabular}
            \end{column}
        \end{columns}
    \end{frame}
    %
    \begin{frame}{Operazioni Relazionali}
        \framesubtitle{Equi-Join}
        \vspace{-1.2cm}
        \[ Supereroi \bowtie_{Film~=~Titolo} Film\]
        \vspace{-.5cm}
        \begin{columns}
            \begin{column}{0.48\textwidth}
                \centering
                {\small $Supereroi$}
                \newline
        \begin{tabular}{|c|c|}
            \hline
            \rowcolor{cyan!30}Nome & Film \\
            \hline
            Bruce Wayne & The Batman \\ \hline
            Bruce Wayne & Batman Begins \\ \hline
            Peter Parker & Spiderman \\ \hline
            Clark Kent & Superman\\ \hline
            \end{tabular}
            \end{column}
            \begin{column}{0.48\textwidth}
                \centering
                {\small $Film$}
                \newline
                \begin{tabular}{|c|c|c|}
                    \hline
                    \rowcolor{cyan!30} Titolo & AnnoUscita & Durata \\ \hline
                    The Batman & 2022 & 176 \\ \hline
                    Spiderman & 2002 & 121 \\ \hline
                    Superman & 1978 & 151 \\ \hline
                    \end{tabular}
            \end{column}
        \end{columns}
        \vspace{.5cm}
        \centering
        \pause
        {\small $Supereroi \bowtie_{Film=Titolo} Film$
        \begin{tabular}{|c|c|c|c|c|}
            \hline
            \rowcolor{cyan!30} Nome & Film & Titolo & AnnoUscita & Durata\\ \hline
            Bruce Wayne & The Batman & The Batman & 2022 & 176 \\ \hline
            Peter Parker & Spiderman & Spiderman & 2002 & 121 \\ \hline
            Clark Kent & Superman & Superman & 1978 & 151\\ \hline
            \end{tabular}}
    \end{frame}
    %
    \begin{frame}{Operazioni Relazionali}
        \framesubtitle{Equi-Join e Natural Join}
        \vspace{-1.2cm}
        \[ Supereroi \bowtie_{Film~=~Titolo} Film\]
        \vspace{-.5cm}
        \begin{center}
        {\small $Supereroi \bowtie Film\footnote{ovviamente ridenominando prima Titolo in Film...}$
        \newline
                \begin{tabular}{|c|c|c|c|}
                    \hline
                    \rowcolor{cyan!30} Nome & Film & AnnoUscita & Durata\\ \hline
                    Bruce Wayne & The Batman & 2022 & 176 \\ \hline
                    Peter Parker & Spiderman & 2002 & 121 \\ \hline
                    Clark Kent & Superman & 1978 & 151\\ \hline
                    \end{tabular}}
                    \vspace{.5cm}
        {
        \small \\$Supereroi \bowtie_{Film=Titolo} Film$
        
        \begin{tabular}{|c|c|c|c|c|}
            \hline
            \rowcolor{cyan!30} Nome & Film & Titolo & AnnoUscita & Durata\\ \hline
            Bruce Wayne & The Batman & The Batman & 2022 & 176 \\ \hline
            Peter Parker & Spiderman & Spiderman & 2002 & 121 \\ \hline
            Clark Kent & Superman & Superman & 1978 & 151\\ \hline
            \end{tabular}}
        \end{center}
    \end{frame}