\documentclass[11pt,aspectratio=169]{beamer}
\usepackage[italian]{babel}
\usepackage[latin1]{inputenc}
\usepackage{graphicx}
\usepackage{listings}
\usepackage[export]{adjustbox}
% Custom bullets
\usepackage{pifont}
\usepackage[useregional]{datetime2}
\usepackage{tikz}
\usepackage[table]{colortbl}
\usepackage{venndiagram}
\usepackage{amsmath}

% custom legend
\newcommand{\cbox}[1]{\raisebox{\depth}{\fcolorbox{black}{#1}{\null}}}

\usetheme{Madrid}
\usecolortheme{spruce}

% Redefine labels
\deftranslation[to=Italian]{Section}{Sezione}
\deftranslation[to=Italian]{Subsection}{Sottosezione}

% Join commands
\def\ojoin{\setbox0=\hbox{$\bowtie$}%
  \rule[-.02ex]{.25em}{.4pt}\llap{\rule[\ht0]{.25em}{.4pt}}}
\def\leftouterjoin{\mathbin{\ojoin\mkern-5.8mu\bowtie}}
\def\rightouterjoin{\mathbin{\bowtie\mkern-5.8mu\ojoin}}
\def\fullouterjoin{\mathbin{\ojoin\mkern-5.8mu\bowtie\mkern-5.8mu\ojoin}}

\author[Gianluca Lanchini \and Piermichele Rosati]{Gianluca Lanchini \and Piermichele Rosati}

\institute[]{\large Universit\`a di Camerino\\ \footnotesize Tutorato - Basi di Dati}

\title[Esercizi - Algebra Relazionale e Fondamenti SQL]{3. Esercizi - Algebra Relazionale e Fondamenti SQL}
\subtitle{Teoria su GROUP BY, HAVING, ORDER BY, UNION}
\setbeamertemplate{navigation symbols}{}
\setbeamertemplate{section in toc}[sections numbered]
\setbeamertemplate{subsection in toc}[subsections numbered]
%\titlegraphic{\includegraphics[width=6cm]{img/unicam-logo.jpg}}
\makeatletter
\setbeamertemplate{footline}{
    \leavevmode%
    \hbox{%
        \begin{beamercolorbox}[wd=.333333\paperwidth,ht=2.25ex,dp=1ex,center]{author in head/foot}%
            \usebeamerfont{author in head/foot}\insertshortauthor
        \end{beamercolorbox}%
        \begin{beamercolorbox}[wd=.333333\paperwidth,ht=2.25ex,dp=1ex,center]{title in head/foot}%
            \usebeamerfont{title in head/foot}\insertshorttitle
        \end{beamercolorbox}%
        \begin{beamercolorbox}[wd=.333333\paperwidth,ht=2.25ex,dp=1ex,right]{date in head/foot}%
            \usebeamerfont{date in head/foot}\today\hspace*{5 em}
            \insertframenumber{} / \inserttotalframenumber\hspace*{2ex} 
        \end{beamercolorbox}%
    }%
    \vskip0pt%
}
\makeatother


\AtBeginSection[]
{
\begin{frame}<beamer>
\frametitle{Indice}
\tableofcontents[currentsection]
\end{frame}
}

\begin{document}
\begin{frame}
\centering
\includegraphics[width=5.5cm]{../img/unicam-logo.jpg}
\date{\today}
\titlepage
\end{frame}
\addtobeamertemplate{frametitle}{}{%
\begin{tikzpicture}[remember picture,overlay]
\node[anchor=north east,yshift=2pt] at (current page.north east) {\includegraphics[height=2cm]{../img/unicam-logo-notext.png}};
\end{tikzpicture}}

%\begin{frame}
%\titlepage
%\end{frame}

\section{Algebra Relazionale}
\def\schemaEx3.8{\small \begin{align*}
    DEPUTATI &= (\underline{Codice}, Cognome, Nome, Commissione, Provincia, Collegio)\\
    COLLEGI &= (\underline{Provincia}, Numero,Nome)\\
    PROVINCE &= (\underline{Sigla}, Nome, Regione)\\
    REGIONI &= (\underline{Codice}, Nome)\\
    COMMISSIONI &= (\underline{Numero}, Nome, Presidente)\\
    \end{align*}}
\begin{frame}{Esercizio 3.8}
    \framesubtitle{3.8.1}
    \schemaEx3.8
    Trovare nome e cognome dei presidenti di commissioni cui partecipa almeno un deputato eletto in una provincia della Sicilia.
\end{frame}
%
\begin{frame}{Esercizio 3.8}
    \framesubtitle{Soluzione 3.8.1}
    \vspace*{-1.2cm}
    \schemaEx3.8
    {\small Nome e cognome dei presidenti di commissioni cui partecipa almeno un deputato eletto in una provincia della Sicilia:}
    \small
    \begin{gather*}
        \onslide<8->{\pi_{Nom,Cogn}}\\
        \onslide<7->{((\rho_{Nom,Cogn~\leftarrow~Nome,Cognome}}
        \onslide<6->{(DEPUTATI))~\bowtie_{Presidente~=~Codice}}\\
        \onslide<5->{(COMMISSIONI~\bowtie_{Numero~=~Comm}}
        \onslide<4->{(\rho_{Comm~\leftarrow~{Commissione}}}
        \onslide<3->{(DEPUTATI~\bowtie_{Provincia~=~Sigla}}\\
        \onslide<2->{(PROVINCE~\bowtie_{Regione~=~Codice}}\\
        \onslide<1->{(\sigma_{Nome~=~'Sicilia'}(REGIONI))}
        \onslide<2->{)}
        \onslide<3->{)}
        \onslide<4->{)}
        \onslide<5->{)}
        \onslide<6->{)}
    \end{gather*}
\end{frame}
%
\begin{frame}{Esercizio 3.8}
    \framesubtitle{3.8.2}
    \schemaEx3.8
    Trovare nome e cognome dei deputati della commissione Bilancio.
\end{frame}
%
\begin{frame}{Esercizio 3.8}
    \framesubtitle{Soluzione 3.8.2}
    \vspace*{-1.2cm}
    \schemaEx3.8
    {\small Nome e cognome dei deputati della commissione Bilancio:}
    \small
    \begin{gather*}
        \onslide<4->{\pi_{NomeC,Cognome}(}
        \onslide<3->{(\rho_{NomeC~\leftarrow~Nome}}
        \onslide<2->{(DEPUTATI))~\bowtie_{Commissione~=~Numero}}\\
        \onslide<1->{(\sigma_{Nome~=~'Bilancio'}(COMMISSIONI))}
    \end{gather*}
\end{frame}
%
\begin{frame}{Esercizio 3.8}
    \framesubtitle{3.8.3}
    \schemaEx3.8
    Trovare nome, cognome e provincia di elezione dei deputati della commissione Bilancio.
\end{frame}
%
\begin{frame}{Esercizio 3.8}
    \framesubtitle{Soluzione 3.8.3}
    \vspace*{-1.2cm}
    \schemaEx3.8
    {\small Nome, cognome e provincia di elezione dei deputati della commissione Bilancio:}
    \small
    \begin{gather*}
        \onslide<6->{\pi_{NomeC,Cognome,NomeP}(}\\
        \onslide<5->{(\rho_{NomeP~\leftarrow~Nome}}
        \onslide<4->{(PROVINCE))~\bowtie_{Sigla~=~Provincia}}\\
        \onslide<3->{((\rho_{NomeC~\leftarrow~Nome}}
        \onslide<2->{(DEPUTATI))~\bowtie_{Commissione~=~Numero}}\\
        \onslide<1->{(\sigma_{Nome~=~'Bilancio'}(COMMISSIONI))}
        \onslide<3->{)}
    \end{gather*}
\end{frame}
%
\begin{frame}{Esercizio 3.8}
    \framesubtitle{3.8.4}
    \schemaEx3.8
    Trovare nome, cognome, provincia e regione di elezione dei deputati della commissione Bilancio.
\end{frame}
%
\begin{frame}{Esercizio 3.8}
    \framesubtitle{Soluzione 3.8.4}
    \vspace*{-1.2cm}
    \schemaEx3.8
    {\small Nome, cognome, provincia e regione di elezione dei deputati della commissione Bilancio:}
    \small
    \begin{gather*}
        \onslide<8->{\pi_{NomeC,Cognome,NomeP,NomeR}(}\\
        \onslide<7->{(\rho_{NomeR~\leftarrow~Nome}}
        \onslide<6->{(REGIONI))~\bowtie_{Codice~=~Regione}}\\
        \onslide<5->{(\rho_{NomeP~\leftarrow~Nome}}
        \onslide<4->{(PROVINCE))~\bowtie_{Sigla~=~Provincia}}\\
        \onslide<3->{((\rho_{NomeC~\leftarrow~Nome}}
        \onslide<2->{(DEPUTATI))~\bowtie_{Commissione~=~Numero}}\\
        \onslide<1->{(\sigma_{Nome~=~'Bilancio'}(COMMISSIONI))}
        \onslide<3->{)}
    \end{gather*}
\end{frame}
%
\def\schemaEx3.10{\small \begin{align*}
    MATERIE &= (\underline{Codice}, Facolta, Denominazione, Professore)\\
    STUDENTI &= (\underline{Matricola}, Cognome, Nome, Facolta)\\
    PROFESSORI &= (\underline{Matricola}, Cognome, Nome)\\
    ESAMI &= (\underline{Studente}, \underline{Materia}, Voto, Data)\\
    PIANIDISTUDIO &= (\underline{Studente}, \underline{Materia}, Anno)\\
    \end{align*}}
\begin{frame}{Esercizio 3.10}
    \framesubtitle{3.10.1}
    \schemaEx3.10
    Trovare gli studenti che hanno riportato in almeno un esame una votazione pari a 30, mostrando, per ciascuno di essi, nome e cognome e data della prima di tali occasioni.
\end{frame}
%
\begin{frame}{Esercizio 3.10}
    \framesubtitle{Soluzione 3.10.1}
    \vspace*{-1.2cm}
    \schemaEx3.10
    {\small Studenti che hanno riportato in almeno un esame una votazione pari a 30, mostrando, per ciascuno di essi, nome e cognome e data della prima di tali occasioni:}
    \small
    \begin{gather*}
        \pi_{Nome,Cognome,Data}\\
        ((\pi_{Studente,Data}(\sigma_{Voto=30} (ESAMI))-\\
        {\pi_{Studente,Data} ((\sigma_{Voto=30}(ESAMI)~\bowtie_{(Studente=Studente1)~AND~(Data>Data1)}}\\
        {\rho_{Studente1,Corso1,Voto1,Data1~\leftarrow~Studente,Corso,Voto,Data}~\sigma_{Voto=30}(ESAMI)))}\\
        {\bowtie_{Studente~=~Matricola}(STUDENTE))}
    \end{gather*}
\end{frame}
%
% \begin{frame}{Esercizio 3.10}
%     \framesubtitle{3.10.2}
%     \schemaEx3.10
%     Per ogni insegnamento della facolt\`a di ingegneria, gli studenti che hanno superato l'esame nell'ultima seduta svoltaf.
% \end{frame}
% %
% \begin{frame}{Esercizio 3.10}
%     \framesubtitle{Soluzione 3.10.2}
%     \vspace*{-1.2cm}
%     \schemaEx3.10
%     {\small Per ogni insegnamento della facolt\`a di ingegneria, gli studenti che hanno superato l'esame nell'ultima seduta svolta:}
%     \small
%     \begin{gather*}
%         \sigma_{(Voto~\geq~18)~AND~(Facolta='Ingegneria')}\\
%         (\pi_{Cognome,Nome,Materia,Voto,Data}\\
%         (STUDENTI~\bowtie_{Studente=Matricola} (\pi_{Studente,Materia,Voto,Data}~(\sigma_{(D~\geq~Data)})\\
%         (ESAMI~\bowtie_{Materia=M}~(\rho_{S,M,V,D~\leftarrow~Studente,Materia,Voto,Data}(ESAMI))))))
%     \end{gather*}
% \end{frame}
%
% \begin{frame}{Esercizio 3.10}
%     \framesubtitle{3.10.3}
%     \schemaEx3.10
%     Trovare gli studenti che hanno superato tutti gli esami previsti dal rispettivo piano di studio.
% \end{frame}
% %
% \begin{frame}{Esercizio 3.10}
%     \framesubtitle{Soluzione 3.10.3}
%     \vspace*{-1.2cm}
%     \schemaEx3.10
%     {\small Gli studenti che hanno superato tutti gli esami previsti dal rispettivo piano di studio:}
%     \small
%     \begin{gather*}
%         \pi_{Studente}~(PIANIDISTUDIO)-\\
%         (\pi_{Studente}(\pi_{Studente,Materia}(PIANIDISTUDIO)~-~\pi_{Studente,Materia}(ESAMI)))
%     \end{gather*}
% \end{frame}
%
\begin{frame}{Esercizio 3.10}
    \framesubtitle{3.10.6}
    \schemaEx3.10
    Trovare nome e cognome degli studenti che hanno sostenuto almeno un esame con un professore che ha il
    loro stesso nome proprio.
\end{frame}
%
\begin{frame}{Esercizio 3.10}
    \framesubtitle{Soluzione 3.10.6}
    \vspace*{-1.2cm}
    \schemaEx3.10
    {\small Nome e cognome degli studenti che hanno sostenuto almeno un esame con un professore che ha il
    loro stesso nome proprio:}
    \small
    \begin{gather*}
        \pi_{Nome,Cognome}(\sigma_{Nome=NomeP}(\\
        (ESAMI \bowtie_{Studente=Matricola} STUDENTI) \bowtie_{Materia=Codice}\\
        (\rho_{NomeP,CognomeP \leftarrow Nome,Cognome} (PROFESSORI \bowtie_{Matricola=Professore} MATERIE))))        
    \end{gather*}
\end{frame}
%
\def\schemaEx3.11{\small \begin{align*}
    CITTA &= (\underline{Nome}, Regione, Abitanti)\\
    ATTRAVERSAMENTI &= (\underline{Citta}, \underline{Fiume})\\
    FIUMI &= (\underline{Fiume}, Lunghezza)
    \end{align*}}
\begin{frame}{Esercizio 3.11}
    \framesubtitle{3.11.1}
    \schemaEx3.11
    Visualizza nome, regione e abitanti per le citt\`a che hanno pi\`u di 50000 abitanti e sono attraversate dal Po oppure dall'Adige.
\end{frame}
%
\begin{frame}{Esercizio 3.11}
    \framesubtitle{Soluzione 3.11.1}
    \vspace*{-1.2cm}
    \schemaEx3.11
    {\small Nome, regione e abitanti per le citt\`a che hanno pi\`u di 50000 abitanti e sono attraversate dal Po oppure dall'Adige:}
    \small
    \begin{gather*}
        \pi_{Nome,Regione,Abitanti} (\sigma_{(Fiume='Po'~OR~(Fiume='Adige'))} (ATTRAVERSAMENTI)~\bowtie_{Citta=Nome}\\
        \sigma_{Abitanti>50000}~(CITTA))
    \end{gather*}
\end{frame}
%
\def\schemaEx3.17{\small \begin{align*}
    FARMACI &= (\underline{Codice}, NomeFarmaco, PrincipioAttivo, Produttore, Prezzo)\\
    PRODUTTORI &= (\underline{CodProduttore}, Nome, Nazione)\\
    SOSTANZE &= (\underline{ID}, NomeSostanza, Categoria)
    \end{align*}}
\begin{frame}{Esercizio 3.17}
    \framesubtitle{3.17.1}
    \schemaEx3.17
    con vincoli di integrit\`a referenziale fra $Produttore$ e la relazione $PRODUTTORI$, fra $PrincipioAttivo$ e la relazione $SOSTANZE$.
    \newline
    \\Formulare l'interrogazione che fornisce, per i farmaci il cui principio attivo \`e nella categoria ``sulfamidico'', il nome del farmaco e quello del suo produttore.
\end{frame}
%
\begin{frame}{Esercizio 3.17}
    \framesubtitle{Soluzione 3.17.1}
    \vspace*{-1.2cm}
    \schemaEx3.17
    {\small Formulare l'interrogazione che fornisce, per i farmaci il cui principio attivo \`e nella categoria ``sulfamidico'', il nome del farmaco e quello del suo produttore:}
    \small
    \begin{gather*}
        A = FARMACI~\bowtie_{PrincipioAttivo~=~ID}(\sigma_{Categoria='sulfamidico'} SOSTANZE);\\
        \pi_{NomeFarmaco,Nome} (PRODUTTORI~\bowtie_{CodProduttore=Produttore} A)
    \end{gather*}
\end{frame}
%
\begin{frame}{Esercizio 3.17}
    \framesubtitle{3.17.2}
    \schemaEx3.17
    con vincoli di integrit\`a referenziale fra $Produttore$ e la relazione $PRODUTTORI$, fra $PrincipioAttivo$ e la relazione $SOSTANZE$.
    \newline
    \\Formulare l'interrogazione che fornisce, per i farmaci con produttore italiano, il nome del farmaco e quello della sostanza del suo principio attivo.
\end{frame}
%
\begin{frame}{Esercizio 3.17}
    \framesubtitle{Soluzione 3.17.2}
    \vspace*{-1.2cm}
    \schemaEx3.17
    {\small Formulare l'interrogazione che fornisce, per i farmaci con produttore italiano, il nome del farmaco e quello della sostanza del suo principio attivo:}
    \small
    \begin{gather*}
        B = FARMACI~\bowtie_{Produttore~=~CodProduttore}(\sigma_{Nazione='Italia'} PRODUTTORI);\\
        \pi_{NomeFarmaco,NomeSostanza} (SOSTANZE~\bowtie_{ID=PrincipioAttivo} B)
    \end{gather*}
\end{frame}
%
\def\schemaEx3.19{\small \begin{align*}
    CLIENTI &= (\underline{Codice}, Nome, Indirizzo, Citta)\\
    NOLEGGI &= (\underline{Cliente, Auto, DataPrelievo}, DataRestituzione)\\
    AUTOVETTURE &= (\underline{Targa}, Modello, Colore, AnnoImmatricolazione, CostoGiornaliero)
    \end{align*}
    $NOLEGGI$ con vincolo di integrit\`a referenziale fra l'attributo $Auto$ e la relazione $AUTOVETTURE$ e con vincolo di integrit\`a referenziale fra l'attributo $Cliente$ e la relazione $CLIENTI$.}
\begin{frame}{Esercizio 3.19}
    \framesubtitle{3.19.1}
    \vspace{-3.cm}
    \schemaEx3.19
    \vspace{.3cm}

    Formulare l'interrogazione che restituisce i dati dei clienti che hanno noleggiato almeno un'autovettura nell'anno 2006.
\end{frame}
%
\begin{frame}{Esercizio 3.19}
    \framesubtitle{Soluzione 3.19.1}
    \vspace*{-2cm}
    \schemaEx3.19
    \vspace{.3cm}

    {\small Formulare l'interrogazione che restituisce i dati dei clienti che hanno noleggiato almeno un'autovettura nell'anno 2006:}
    \small
    \begin{gather*}
        A = \sigma_{(DataPrelievo~\geq~'01/01/2006')~AND~(DataPrelievo~\leq~'31/12/2006')}~(NOLEGGI);\\
        CLIENTI~\bowtie_{Codice~=~Cliente} A
    \end{gather*}
\end{frame}
%
\begin{frame}{Esercizio 3.19}
    \framesubtitle{3.19.2}
    \vspace{-3.cm}
    \schemaEx3.19
    \vspace{.3cm}

    Formulare l'interrogazione che restituisce i clienti che hanno noleggiato pi\`u di un'autovettura.
\end{frame}
%
\begin{frame}{Esercizio 3.19}
    \framesubtitle{Soluzione 3.19.2}
    \vspace*{-2cm}
    \schemaEx3.19
    \vspace{.3cm}

    {\small Formulare l'interrogazione che restituisce i clienti che hanno noleggiato pi\`u di un'autovettura:}
    \small
    \begin{gather*}
        \pi_{Cliente} (\sigma_{Auto~\neq~Auto'} (NOLEGGI~\bowtie_{Cliente=Cliente'}~(\rho_{X'~\leftarrow~X} NOLEGGI)))
    \end{gather*}
\end{frame}
%
\begin{frame}{Esercizio 3.19}
    \framesubtitle{3.19.3}
    \vspace{-3.cm}
    \schemaEx3.19
    \vspace{.3cm}

    Formulare l'interrogazione che restituisce i clienti che hanno noleggiato autovetture di un solo modello.
\end{frame}
%
\begin{frame}{Esercizio 3.19}
    \framesubtitle{Soluzione 3.19.3}
    \vspace*{-2cm}
    \schemaEx3.19
    \vspace{.3cm}

    {\small Formulare l'interrogazione che restituisce i clienti che hanno noleggiato autovetture di un solo modello:}
    \small
    \begin{gather*}
        V = NOLEGGI~\bowtie_{Auto~=~Targa}~AUTOVETTURE\\
        V_{1} = V~\bowtie_{Cliente~=~Cliente'} (\rho_{X'~\leftarrow~X} V)\\
        \pi_{Cliente}~NOLEGGI - \pi_{Cliente}~(\sigma_{Modello~\neq~Modello'} V_1)
    \end{gather*}
\end{frame}
%
\def\schemaEx3.20{\small \begin{align*}
    FORNITORI &= (\underline{CF}, Nome, Indirizzo, Citta)\\
    PRODOTTI &= (\underline{CP}, Nome, Marca, Modello)\\
    CATALOGO &= (\underline{CF}, \underline{CP}, Costo)
    \end{align*}
    $CATALOGO$ con vincoli di integrit\`a referenziale fra l'attributo $CF$ e la relazione $FORNITORI$ e fra l'attributo $CP$ e la relazione $PRODOTTI$.}
\begin{frame}{Esercizio 3.20}
    \framesubtitle{3.20.1}
    \vspace{-3.cm}
    \schemaEx3.20
    \vspace{.3cm}

    Trovare Nome, Marca e Modello dei prodotti acquistabili con meno di 2000.
\end{frame}
%
\begin{frame}{Esercizio 3.20}
    \framesubtitle{Soluzione 3.20.1}
    \vspace*{-2cm}
    \schemaEx3.20
    \vspace{.3cm}

    {\small Nome, Marca e Modello dei prodotti acquistabili con meno di 2000:}
    \small
    \begin{gather*}
        \pi_{Nome,Marca,Modello} (\sigma_{Costo~<~2000}~PRODOTTI~\bowtie~CATALOGO)
    \end{gather*}
\end{frame}
%
\begin{frame}{Esercizio 3.20}
    \framesubtitle{3.20.2}
    \vspace{-3.cm}
    \schemaEx3.20
    \vspace{.3cm}

    Trovare i nomi dei fornitori che distribuiscono prodotti IBM (IBM \`e la marca di un prodotto).
\end{frame}
%
\begin{frame}{Esercizio 3.20}
    \framesubtitle{Soluzione 3.20.2}
    \vspace*{-2cm}
    \schemaEx3.20
    \vspace{.3cm}

    {\small Nomi dei fornitori che distribuiscono prodotti IBM (IBM \`e la marca di un prodotto):}
    \small
    \begin{gather*}
        R = (FORNITORI~\bowtie~CATALOGO)~\bowtie~(\pi_{CP,Marca}~PRODOTTI)\\
        \pi_{Nome} (\sigma_{Marca='IBM'} (R))
    \end{gather*}
\end{frame}
%
\begin{frame}{Esercizio 3.20}
    \framesubtitle{3.20.3}
    \vspace{-3.cm}
    \schemaEx3.20
    \vspace{.3cm}

    Trovare i codici di tutti i prodotti che sono forniti da almeno due fornitori.
\end{frame}
%
\begin{frame}{Esercizio 3.20}
    \framesubtitle{Soluzione 3.20.3}
    \vspace*{-2cm}
    \schemaEx3.20
    \vspace{.3cm}

    {\small Codici di tutti i prodotti che sono forniti da almeno due fornitori:}
    \small
    \begin{gather*}
        COPIACATALOGO = \rho_{CF'~\leftarrow~CF} (\pi_{CP,CF} CATALOGO)\\
        \pi_{CP} (\sigma_{CF~>~CF'} (CATALOGO~\bowtie~COPIACATALOGO))
    \end{gather*}
\end{frame}
%
\begin{frame}{Esercizio 3.20}
    \framesubtitle{3.20.4}
    \vspace{-3.cm}
    \schemaEx3.20
    \vspace{.3cm}

    Trovare i nomi dei fornitori che distribuiscono tutti i prodotti presenti nel catalogo.
\end{frame}
%
\begin{frame}{Esercizio 3.20}
    \framesubtitle{Soluzione 3.20.4}
    \vspace*{-1cm}
    \schemaEx3.20
    \vspace{.3cm}

    {\small Nomi dei fornitori che distribuiscono tutti i prodotti presenti nel catalogo:}
    \small
    \newline
    {\\Formuliamo innanzitutto l'interrogazione $R$ che trova i fornitori a cui manca almeno un prodotto del catalogo:\\}
    \vspace{-.5cm}
    \begin{gather*}
        R = \pi_{CF}((\pi_{CF} (FORNITORI)~\bowtie~\pi_{CP}(CATALOGO))~-~\pi_{CF,CP}(CATALOGO))\\
        \pi_{Nome}((\pi_{CF}(FORNITORI)~-~R)~\bowtie~FORNITORI)
    \end{gather*}
\end{frame}
%
\begin{frame}{Esercizio 3.20}
    \framesubtitle{3.20.5}
    \vspace{-3.cm}
    \schemaEx3.20
    \vspace{.3cm}

    Trovare i nomi dei fornitori che forniscono tutti i prodotti IBM presenti nel catalogo.
\end{frame}
%
\begin{frame}{Esercizio 3.20}
    \framesubtitle{Soluzione 3.20.5}
    \vspace*{-2cm}
    \schemaEx3.20
    \vspace{.3cm}

    {\small Nomi dei fornitori che forniscono tutti i prodotti IBM presenti nel catalogo:}
    \small
    \begin{gather*}
        CATALOGOIBM = \pi_{CP,CF} (\sigma_{Marca='IBM'} (CATALOGO~\bowtie~PRODOTTI))\\
        R = \pi_{CF}((\pi_{CF} (FORNITORI)~\bowtie~\pi_{CP}(CATALOGOIBM))~-~\pi_{CF,CP}(CATALOGOIBM))\\
        \pi_{Nome}((\pi_{CF}(FORNITORI)~-~R)~\bowtie~FORNITORI)
    \end{gather*}
\end{frame}
%
\section{Fondamenti SQL}
%%%%%%%%%%%%%%%%
\begin{frame}{GROUP BY}
\begin{table}[h]
\centering
\begin{minipage}{.45\textwidth}
\centering
\begin{tabular}{|c|c|c|}
\hline
\rowcolor{cyan!30} \multicolumn{3}{|c|}{Esame} \\
\hline
\rowcolor{cyan!30} matricola  & voto & corso \\
\hline
100  & 30 & 27035 \\
100  & 28 & 27038 \\
100  & 18 & 27010 \\
102  & 21 & 27038 \\
101  & 25 & 27045 \\
102 & 21 & 27010 \\
\hline
\end{tabular}
\end{minipage}%
\begin{minipage}{.45\textwidth}
\centering
\begin{tabular}{|c|c|}
\hline
\rowcolor{cyan!30} \multicolumn{2}{|c|}{Result} \\
\hline
\rowcolor{cyan!30} matricola  & voti \\
\hline
100  & 3 \\
101  & 1  \\
102  & 2 \\
\hline
\end{tabular}
\end{minipage}
\end{table}
\vspace{2em}
\texttt{SELECT matricola, COUNT(voto) AS voti\\FROM Esame\\GROUP BY matricola;}
\end{frame}
%%%%%%%%%%%%%%%%
\begin{frame}{HAVING}
\begin{table}[h]
\centering
\begin{minipage}{.45\textwidth}
\centering
\begin{tabular}{|c|c|c|}
\hline
\rowcolor{cyan!30} \multicolumn{3}{|c|}{Esame} \\
\hline
\rowcolor{cyan!30} matricola  & voto & corso \\
\hline
100  & 30 & 27035 \\
100  & 28 & 27038 \\
100  & 18 & 27010 \\
102  & 21 & 27038 \\
101  & 25 & 27045 \\
102 & 21 & 27010 \\
\hline
\end{tabular}
\end{minipage}%
\begin{minipage}{.45\textwidth}
\centering
\begin{tabular}{|c|c|c|}
\hline

\hline
\end{tabular}
\end{minipage}
\end{table}
\vspace{2em}
\texttt{SELECT matricola, COUNT(voto) AS voti\\FROM Esame\\GROUP BY matricola\\HAVING COUNT(voto)>1;}
\end{frame}
%%%%%%%%%%%%%%%%
\begin{frame}{HAVING}
\begin{table}[h]
\centering
\begin{minipage}{.45\textwidth}
\centering
\begin{tabular}{|c|c|c|}
\hline
\rowcolor{cyan!30} \multicolumn{3}{|c|}{Esame} \\
\hline
\rowcolor{cyan!30} matricola  & voto & corso \\
\hline
100  & 30 & 27035 \\
100  & 28 & 27038 \\
100  & 18 & 27010 \\
102  & 21 & 27038 \\
101  & 25 & 27045 \\
102 & 21 & 27010 \\
\hline
\end{tabular}
\end{minipage}%
\begin{minipage}{.45\textwidth}
\centering
\begin{tabular}{|c|c|}
\hline
\rowcolor{cyan!30} \multicolumn{2}{|c|}{Result} \\
\hline
\rowcolor{cyan!30} matricola  & voti \\
\hline
100  & 3 \\
102  & 2  \\
\hline
\end{tabular}
\end{minipage}
\end{table}
\vspace{2em}
\texttt{SELECT matricola, COUNT(voto) AS voti\\FROM Esame\\GROUP BY matricola\\HAVING COUNT(voto)>1;}
\end{frame}
%%%%%%%%%%%%%%%%
\begin{frame}{ORDER BY}
\begin{table}[h]
\centering
\begin{minipage}{.45\textwidth}
\centering
\begin{tabular}{|c|c|c|}
\hline
\rowcolor{cyan!30} \multicolumn{3}{|c|}{Esame} \\
\hline
\rowcolor{cyan!30} matricola  & voto & corso \\
\hline
100  & 30 & 27035 \\
100  & 28 & 27038 \\
100  & 18 & 27010 \\
102  & 21 & 27038 \\
101  & 25 & 27045 \\
102 & 21 & 27010 \\
\hline
\end{tabular}
\end{minipage}%
\begin{minipage}{.45\textwidth}
\centering
\begin{tabular}{|c|c|c|}
\hline

\hline
\end{tabular}
\end{minipage}
\end{table}
\vspace{2em}
\texttt{SELECT *\\FROM Esame\\ORDER BY matricola ASC;}
\end{frame}
%%%%%%%%%%%%%%%%
\begin{frame}{ORDER BY}
\begin{table}[h]
\centering
\begin{minipage}{.45\textwidth}
\centering
\begin{tabular}{|c|c|c|}
\hline
\rowcolor{cyan!30} \multicolumn{3}{|c|}{Esame} \\
\hline
\rowcolor{cyan!30} matricola  & voto & corso \\
\hline
100  & 30 & 27035 \\
100  & 28 & 27038 \\
100  & 18 & 27010 \\
102  & 21 & 27038 \\
101  & 25 & 27045 \\
102 & 21 & 27010 \\
\hline
\end{tabular}
\end{minipage}%
\begin{minipage}{.45\textwidth}
\centering
\begin{tabular}{|c|c|c|}
\hline
\rowcolor{cyan!30} \multicolumn{3}{|c|}{Esame} \\
\hline
\rowcolor{cyan!30} matricola  & voto & corso \\
\hline
100 & 30 & 27035 \\
100 & 28 & 27038 \\
100 & 18 & 27010 \\
101 & 25 & 27045 \\
102 & 21 & 27038 \\
102 & 21 & 27010 \\
\hline
\end{tabular}
\end{minipage}
\end{table}
\vspace{2em}
\texttt{SELECT *\\FROM Esame\\ORDER BY matricola ASC;}
\end{frame}
%%%%%%%%%%%%%%%%
\begin{frame}{ORDER BY}
\begin{table}[h]
\centering
\begin{minipage}{.45\textwidth}
\centering
\begin{tabular}{|c|c|c|}
\hline
\rowcolor{cyan!30} \multicolumn{3}{|c|}{Esame} \\
\hline
\rowcolor{cyan!30} matricola  & voto & corso \\
\hline
100  & 30 & 27035 \\
100  & 28 & 27038 \\
100  & 18 & 27010 \\
102  & 21 & 27038 \\
101  & 25 & 27045 \\
102 & 21 & 27010 \\
\hline
\end{tabular}
\end{minipage}%
\begin{minipage}{.45\textwidth}
\centering
\begin{tabular}{|c|c|c|}
\hline

\hline
\end{tabular}
\end{minipage}
\end{table}
\vspace{2em}
\texttt{SELECT *\\FROM Esame\\ORDER BY matricola DESC, voto ASC;}
\end{frame}
%%%%%%%%%%%%%%%%
\begin{frame}{ORDER BY}
\begin{table}[h]
\centering
\begin{minipage}{.45\textwidth}
\centering
\begin{tabular}{|c|c|c|}
\hline
\rowcolor{cyan!30} \multicolumn{3}{|c|}{Esame} \\
\hline
\rowcolor{cyan!30} matricola  & voto & corso \\
\hline
100  & 30 & 27035 \\
100  & 28 & 27038 \\
100  & 18 & 27010 \\
102  & 21 & 27038 \\
101  & 25 & 27045 \\
102 & 21 & 27010 \\
\hline
\end{tabular}
\end{minipage}%
\begin{minipage}{.45\textwidth}
\centering
\begin{tabular}{|c|c|c|}
\hline
\rowcolor{cyan!30} \multicolumn{3}{|c|}{Esame} \\
\hline
\rowcolor{cyan!30} matricola  & voto & corso \\
\hline
102 & 21 & 27038 \\
102 & 21 & 27010 \\
101 & 25 & 27045 \\
100 & 18 & 27010 \\
100 &  28 & 27038 \\
100 & 30 & 27035 \\
\hline
\end{tabular}
\end{minipage}
\end{table}
\vspace{2em}
\texttt{SELECT *\\FROM Esame\\ORDER BY matricola DESC, voto ASC;}
\end{frame}
%%%%%%%%%%%%%%%%
\begin{frame}{UNION}
\begin{table}[h]
\centering
\begin{minipage}{.45\textwidth}
\centering
\begin{tabular}{|c|c|c|}
\hline
\rowcolor{cyan!30} \multicolumn{3}{|c|}{Studente} \\
\hline
\rowcolor{cyan!30} Nome  & Cognome & AnnoDiNascita \\
\hline
ciccio  & pasticcio & 1992 \\
pippo  & calippo & 1994 \\

\hline
\end{tabular}
\end{minipage}%
\begin{minipage}{.45\textwidth}
\centering
\begin{tabular}{|c|c|c|}
\hline
\rowcolor{cyan!30} \multicolumn{3}{|c|}{Professore} \\
\hline
\rowcolor{cyan!30} Nome  & Cognome & DataNascita \\
\hline
balema & giuseppina & 1967 \\

\hline
\end{tabular}
\end{minipage}
\end{table}
\vspace{2em}
\texttt{SELECT * FROM Studente\\UNION\\SELECT * FROM Professore;}
\end{frame}
%%%%%%%%%%%%%%%%
\begin{frame}{UNION}
\begin{table}[h]
\centering
\begin{minipage}{.45\textwidth}
\centering
\begin{tabular}{|c|c|c|}
\hline
\rowcolor{cyan!30} \multicolumn{3}{|c|}{Studente} \\
\hline
\rowcolor{cyan!30} Nome  & Cognome & AnnoDiNascita \\
\hline
ciccio  & pasticcio & 1992 \\
pippo  & calippo & 1994 \\

\hline
\end{tabular}
\end{minipage}%
\begin{minipage}{.45\textwidth}
\centering
\begin{tabular}{|c|c|c|}
\hline
\rowcolor{cyan!30} \multicolumn{3}{|c|}{Professore} \\
\hline
\rowcolor{cyan!30} Nome  & Cognome & AnnoDiNascita \\
\hline
balema & giuseppina & 1967 \\

\hline
\end{tabular}
\end{minipage}
\end{table}
\vspace{2em}
\texttt{SELECT * FROM Studente\\UNION\\SELECT * FROM Professore;}
\begin{itemize}
    \item Stesso numero di colonne nelle SELECT
    \item Stesso ordine di colonne nelle SELECT
    \item Stesso tipo dato nelle colonne delle SELECT
\end{itemize}
\end{frame}   
\def\consegnaSPsExOne{
Creare una stored procedure \textbf{get\_customer\_total\_payment} che riceve un \textbf{customer\_id} e restituisce l'importo totale pagato da quel cliente.
}
\def\consegnaSPsExTwo{
Creare una stored procedure \textbf{insert\_rental} che inserisce una nuova riga nella tabella \textbf{rental} dato:
    \begin{itemize}
        \item \textbf{customer\_id}
        \item \textbf{inventory\_id}
        \item \textbf{staff\_id}
    \end{itemize}
}
\def\consegnaSPsExThree{
Creare una stored procedure \textbf{deactivate\_inactive\_customers} che imposta \textbf{active = 0} per tutti i clienti che non hanno effettuato un pagamento negli ultimi 12 mesi.
}
\def\consegnaFunctionsExOne{
Creare una funzione \textbf{get\_customer\_full\_name} che prende un \textbf{customer\_id} e restituisce il nome completo (ad esempio, 'John Smith') come stringa unica.
}
\def\consegnaFunctionsExTwo{
Creare una funzione \textbf{get\_rental\_count} che restituisce il numero di noleggi effettuati da un determinato \textbf{customer\_id}.
}
\def\consegnaFunctionsExThree{
Creare una funzione \textbf{is\_film\_available} che prende un \textbf{film\_id} e restituisce 1 se c'\`e almeno una copia di quel film nell'inventario, altrimenti 0.
}
\def\consegnaTriggersExOne{
Creare un trigger che memorizzi il \textbf{titolo} e il \textbf{timestamp} di cancellazione di qualsiasi film che viene eliminato dalla \textbf{tabella film}.

Utilizzare una nuova tabella chiamata \textbf{deleted\_films\_log}.
}
\def\consegnaTriggersExTwo{
Creare un trigger che aggiorni una tabella \textbf{customer\_payment\_count} ogni volta che viene inserito un pagamento.
Se il cliente esiste gi\`a nella tabella, incrementare il suo conteggio; altrimenti, inserirlo con count = 1.

Prima di tutto, creare una tabella di riepilogo \textbf{customer\_payment\_count(customer\_id, payment\_count)}.
}
\def\consegnaTriggersExThree{
Creare un trigger che, prima di cancellare un cliente, copia i dati del cliente in una tabella \textbf{archived\_customers} per scopi di registrazione.
}
%
\begin{frame}[fragile]{Stored Procedures}
\framesubtitle{Esercizio 1: Calcolo del Totale Pagato da un Cliente}
\consegnaSPsExOne
\end{frame}
%
\begin{frame}[fragile]{Stored Procedures}
\framesubtitle{Soluzione Esercizio 1: Calcolo del Totale Pagato da un Cliente}
\consegnaSPsExOne

\vspace{.2cm}

\textbf{Creazione della stored procedure:}
\begin{lstlisting}
DELIMITER \\
CREATE PROCEDURE get_customer_total_payment(IN cust_id INT, OUT total DECIMAL(10,2))
BEGIN
    SELECT SUM(amount)
    INTO total
    FROM payment
    WHERE customer_id = cust_id;
END\\
DELIMITER ;
\end{lstlisting}
\end{frame}
%
\begin{frame}[fragile]{Stored Procedures}
\framesubtitle{Soluzione Esercizio 1: Calcolo del Totale Pagato da un Cliente}
\consegnaSPsExOne

\vspace{.2cm}

\textbf{Chiamata della stored procedure:}
\begin{lstlisting}
CALL get_customer_total_payment(1, @total);
SELECT @total;
\end{lstlisting}
\end{frame}
%
\begin{frame}[fragile]{Stored Procedures}
\framesubtitle{Esercizio 2: Inserimento di un Noleggio}
\consegnaSPsExTwo
\end{frame}
%
\begin{frame}[fragile]{Stored Procedures}
\framesubtitle{Soluzione Esercizio 2: Inserimento di un Noleggio}

\vspace{-.5cm}

\textbf{Creazione della stored procedure:}
\begin{lstlisting}
DELIMITER \\

CREATE PROCEDURE insert_rental(
    IN p_customer_id INT,
    IN p_inventory_id INT,
    IN p_staff_id INT
)
BEGIN
    INSERT INTO rental (rental_date, inventory_id, customer_id, staff_id)
    VALUES (NOW(), p_inventory_id, p_customer_id, p_staff_id);
END\\

DELIMITER ;
\end{lstlisting}
\end{frame}
%
\begin{frame}[fragile]{Stored Procedures}
\framesubtitle{Soluzione Esercizio 2: Inserimento di un Noleggio}
\consegnaSPsExOne

\vspace{.2cm}

\textbf{Chiamata della stored procedure:}
\begin{lstlisting}
CALL insert_rental(1, 1000, 2);
\end{lstlisting}
\end{frame}
%
\begin{frame}[fragile]{Stored Procedures}
\framesubtitle{Esercizio 3: Disattivazione Clienti Inattivi}
\consegnaSPsExThree
\end{frame}
%
\begin{frame}[fragile]{Stored Procedures}
\framesubtitle{Soluzione Esercizio 3: Disattivazione Clienti Inattivi}

\vspace{-.5cm}
\small
\textbf{Creazione della stored procedure:}
\begin{lstlisting}
DELIMITER \\

CREATE PROCEDURE deactivate_inactive_customers()
BEGIN
    UPDATE customer
    SET active = 0
    WHERE customer_id IN (
        SELECT c.customer_id
        FROM customer c
        LEFT JOIN payment p ON c.customer_id = p.customer_id
        GROUP BY c.customer_id
        HAVING MAX(p.payment_date) < DATE_SUB(NOW(), INTERVAL 12 MONTH)
    );
END\\
DELIMITER ;
\end{lstlisting}
\end{frame}
%
\begin{frame}[fragile]{Stored Procedures}
\framesubtitle{Soluzione Esercizio 3: Disattivazione Clienti Inattivi}
\consegnaSPsExOne

\vspace{.2cm}

\textbf{Chiamata della stored procedure:}
\begin{lstlisting}
CALL deactivate_inactive_customers();
\end{lstlisting}
\end{frame}
%
\begin{frame}[fragile]{Functions}
\framesubtitle{Esercizio 1: Nome Completo del Cliente}
\consegnaFunctionsExOne
\end{frame}
%
\begin{frame}[fragile]{Functions}
\framesubtitle{Soluzione Esercizio 1: Nome Completo del Cliente}

\vspace{-.5cm}
\small
\textbf{Creazione della funzione:}
\begin{lstlisting}
DELIMITER \\

CREATE FUNCTION get_customer_full_name(cust_id INT)
RETURNS VARCHAR(100)
DETERMINISTIC
BEGIN
    DECLARE full_name VARCHAR(100);
    SELECT CONCAT(first_name, ' ', last_name)
    INTO full_name
    FROM customer
    WHERE customer_id = cust_id;
    RETURN full_name;
END\\

DELIMITER ;
\end{lstlisting}
\end{frame}
%
\begin{frame}[fragile]{Functions}
\framesubtitle{Soluzione Esercizio 1: Nome Completo del Cliente}
\consegnaFunctionsExOne

\vspace{.2cm}

\textbf{Chiamata della funzione:}
\begin{lstlisting}
SELECT get_customer_full_name(1);
\end{lstlisting}
\end{frame}
%
\begin{frame}[fragile]{Functions}
\framesubtitle{Esercizio 2: Conteggio dei Noleggi per Cliente}
\consegnaFunctionsExTwo
\end{frame}
%
\begin{frame}[fragile]{Functions}
\framesubtitle{Soluzione Esercizio 2: Conteggio dei Noleggi per Cliente}

\vspace{-.5cm}
\small
\textbf{Creazione della funzione:}
\begin{lstlisting}
DELIMITER \\

CREATE FUNCTION get_rental_count(cust_id INT)
RETURNS INT
DETERMINISTIC
BEGIN
    DECLARE rental_total INT;
    SELECT COUNT(*) INTO rental_total
    FROM rental
    WHERE customer_id = cust_id;
    RETURN rental_total;
END\\

DELIMITER ;
\end{lstlisting}
\end{frame}
%
\begin{frame}[fragile]{Functions}
\framesubtitle{Soluzione Esercizio 2: Conteggio dei Noleggi per Cliente}
\consegnaFunctionsExTwo

\vspace{.2cm}

\textbf{Chiamata della funzione:}
\begin{lstlisting}
SELECT get_rental_count(1);
\end{lstlisting}
\end{frame}
%
\begin{frame}[fragile]{Functions}
\framesubtitle{Esercizio 3: Disponibilit\`a del Film}
\consegnaFunctionsExThree
\end{frame}
%
\begin{frame}[fragile]{Functions}
\framesubtitle{Soluzione Esercizio 3: Disponibilit\`a del Film}

\vspace{-.5cm}
\scriptsize
\textbf{Creazione della funzione:}
\begin{lstlisting}
DELIMITER \\

CREATE FUNCTION is_film_available(film_id_input INT)
RETURNS TINYINT
DETERMINISTIC
BEGIN
    DECLARE count_inventory INT;
    SELECT COUNT(*) INTO count_inventory
    FROM inventory
    WHERE film_id = film_id_input;

    IF count_inventory > 0 THEN
        RETURN 1;
    ELSE
        RETURN 0;
    END IF;
END\\

DELIMITER ;
\end{lstlisting}
\end{frame}
%
\begin{frame}[fragile]{Functions}
\framesubtitle{Soluzione Esercizio 3: Disponibilit\`a del Film}
\consegnaFunctionsExThree

\vspace{.2cm}

\textbf{Chiamata della funzione:}
\begin{lstlisting}
SELECT is_film_available(100);
\end{lstlisting}
\end{frame}
%
\begin{frame}[fragile]{Triggers}
\framesubtitle{Esercizio 1: Log di Cancellazione dei Film}
\consegnaTriggersExOne
\end{frame}
%
\begin{frame}[fragile]{Triggers}
\framesubtitle{Soluzione Esercizio 1: Log di Cancellazione dei Film}
\consegnaTriggersExOne

\vspace{.5cm}

\textbf{Step 1: Creazione della tabella di log}
\begin{lstlisting}
CREATE TABLE IF NOT EXISTS deleted_films_log (
    film_id INT,
    title VARCHAR(255),
    deleted_at DATETIME
);

\end{lstlisting}

\end{frame}
%
\begin{frame}[fragile]{Triggers}
\framesubtitle{Soluzione Esercizio 1: Log di Cancellazione dei Film}
\consegnaTriggersExOne

\vspace{.2cm}

\textbf{Step 2: Creazione del trigger}
\begin{lstlisting}
DELIMITER \\
CREATE TRIGGER after_delete_film_log
AFTER DELETE ON film
FOR EACH ROW
BEGIN
    INSERT INTO deleted_films_log(film_id, title, deleted_at)
    VALUES (OLD.film_id, OLD.title, NOW());
END\\
DELIMITER ;
\end{lstlisting}

\end{frame}
%
\begin{frame}[fragile]{Triggers}
\framesubtitle{Esercizio 2: Conteggio dei Pagamenti per Cliente}
\consegnaTriggersExTwo
\end{frame}
%
\begin{frame}[fragile]{Triggers}
\framesubtitle{Soluzione Esercizio 2: Conteggio dei Pagamenti per Cliente}

\consegnaTriggersExTwo

\vspace{.5cm}

\textbf{Step 1: Creazione della tabella di riepilogo}
\begin{lstlisting}
CREATE TABLE IF NOT EXISTS customer_payment_count (
    customer_id INT PRIMARY KEY,
    payment_count INT DEFAULT 0
);

\end{lstlisting}

\end{frame}
%
\begin{frame}[fragile]{Triggers}
\framesubtitle{Soluzione Esercizio 2: Conteggio dei Pagamenti per Cliente}

\scriptsize

\vspace{-1cm}

\textbf{Step 2: Creazione del trigger}
\begin{lstlisting}
DELIMITER \\

CREATE TRIGGER after_insert_payment_count
AFTER INSERT ON payment
FOR EACH ROW
BEGIN
    IF EXISTS (SELECT * FROM customer_payment_count WHERE customer_id = NEW.customer_id) THEN
        UPDATE customer_payment_count
        SET payment_count = payment_count + 1
        WHERE customer_id = NEW.customer_id;
    ELSE
        INSERT INTO customer_payment_count(customer_id, payment_count)
        VALUES (NEW.customer_id, 1);
    END IF;
END\\

DELIMITER ;
\end{lstlisting}
\end{frame}
%
\begin{frame}[fragile]{Triggers}
\framesubtitle{Esercizio 3: Impedire la Cancellazione di Staff Attivi}
\consegnaTriggersExThree
\end{frame}
%
\begin{frame}[fragile]{Triggers}
\framesubtitle{Soluzione Esercizio 3: Impedire la Cancellazione di Staff Attivi}

\textbf{Step 1: Creazione della tabella di archiviazione}
\begin{lstlisting}
CREATE TABLE IF NOT EXISTS archived_customers (
    customer_id INT,
    store_id TINYINT,
    first_name VARCHAR(45),
    last_name VARCHAR(45),
    email VARCHAR(50),
    address_id SMALLINT,
    active BOOLEAN,
    create_date DATETIME,
    archive_date DATETIME
);

\end{lstlisting}

\end{frame}
%
\begin{frame}[fragile]{Triggers}
\framesubtitle{Soluzione Esercizio 3: Impedire la Cancellazione di Staff Attivi}

\vspace{-.5cm}
\small
\textbf{Step 2: Creazione del trigger}
\begin{lstlisting}
DELIMITER \\
CREATE TRIGGER before_delete_customer_archive
BEFORE DELETE ON customer
FOR EACH ROW
BEGIN
    INSERT INTO archived_customers (
        customer_id, store_id, first_name, last_name, email,
        address_id, active, create_date, archive_date
    )
    VALUES (
        OLD.customer_id, OLD.store_id, OLD.first_name, OLD.last_name, OLD.email,
        OLD.address_id, OLD.active, OLD.create_date, NOW()
    );
END\\
DELIMITER ;
\end{lstlisting}
\end{frame}   
%
\section{Conclusioni}

\begin{frame}{Domande?}
    \begin{figure}
\centering
    \includegraphics[width=0.75\textwidth]{../img/questions.jpg}
\end{figure}
\end{frame}

\begin{frame}{Fine}
    \centering
    \huge Grazie dell'attenzione!
\end{frame}

\end{document}