\begin{frame}{Esercizio 4.1}
         Ordinare i seguenti domini in base al valore massimo rappresentabile, supponendo che \texttt{integer} abbia una rappresentazione a 32 bit e \texttt{smallint} a 16 bit:
        \begin{enumerate}
            \item \texttt{numeric(12,4)}
            \item \texttt{decimal(10)}
            \item \texttt{decimal(9)}
            \item \texttt{integer}
            \item \texttt{smallint}
            \item \texttt{decimal(6,1)}
        \end{enumerate}
\end{frame}

\begin{frame}{Esercizio 4.1}
    \textbf{Soluzione:}
    \begin{table}[h]
        \centering
        \begin{tabular}{|c|c|}
            \hline
            Dominio & Valore Massimo \\
            \hline
            \texttt{decimal(10)} & 9999999999 \\
            \texttt{integer} & 4294967296 \\
            \texttt{decimal(9)} & 999999999 \\
            \texttt{numeric(12,4)} & 99999999.9999 \\
            \texttt{decimal(6,1)} & 99999.9 \\
            \texttt{smallint} & 65536 \\
            \hline
        \end{tabular}
    \end{table}
\end{frame}
%%%%%%%%%%%%%%%%
\begin{frame}{Esercizio 4.7}
    Con riferimento ad una relazione \texttt{PROFESSORI(CF, Nome, Et\`a, Qualifica)}, scrivere le interrogazioni SQL che calcolano l'et\`a media dei professori di ciascuna qualifica, nei due casi seguenti:
    \begin{enumerate}
        \item se l'et\`a non \`e nota si usa per essa il valore nullo
        \item se l'et\`a non \`e nota si usa per essa il valore 0.
    \end{enumerate}
\end{frame}

\begin{frame}{Esercizio 4.7}
    \textbf{Soluzione:}
    Le funzioni aggregative escludono dalla valutazione le ennuple con valori nulli:
    \vspace{1em}
    
    \texttt{SELECT Qualifica, AVG(Et\`a) AS Et\`aMedia\\FROM Professori\\GROUP BY Qualifica;}
\end{frame}

\begin{frame}{Esercizio 4.7}
    \textbf{Soluzione:}
    \`E necessario escludere esplicitamente dal calcolo della media le ennuple con il valore che denota l'informazione incompleta:
    \vspace{1em}
    
    \texttt{SELECT Qualifica, AVG(Et\`a) AS Et\`aMedia\\FROM Professori\\WHERE Et\`a <> 0\\GROUP BY Qualifica;}
\end{frame}
%%%%%%%%%%%%%%%%
\begin{frame}{Esercizio 4.14.1}
    Dato il seguente schema:
    \begin{itemize}
        \item AEROPORTO(Citt\`a, Nazione,NumPiste)
        \item VOLO(IdVolo,GiornoSett,Citt\`aPart,OraPart,
        Citt\`aArr,OraArr,TipoAereo)
        \item AEREO(TipoAereo,NumPasseggeri,QtaMerci)
    \end{itemize}
    \vspace{1em}
    
    \textbf{Le citt\`a con un aeroporto di cui non \`e noto il numero di piste;}
\end{frame}

\begin{frame}{Esercizio 4.14.1}
    \textbf{Soluzione:}
    \vspace{1em}
    
    \texttt{SELECT DISTINCT Citt\`a\\FROM AEROPORTO\\WHERE NumPiste IS NULL;}
    \end{frame}
%%%%%%%%%%%%%%%%%%%
\begin{frame}{Esercizio 4.14.2}
    Dato il seguente schema:
    \begin{itemize}
        \item AEROPORTO(Citt\`a, Nazione,NumPiste)
        \item VOLO(IdVolo,GiornoSett,Citt\`aPart,OraPart,
        Citt\`aArr,OraArr,TipoAereo)
        \item AEREO(TipoAereo,NumPasseggeri,QtaMerci)
    \end{itemize}
    \vspace{1em}
    
    \textbf{Le nazioni da cui parte e arriva il volo con codice AZ274;}
\end{frame}

\begin{frame}{Esercizio 4.14.2}
    \textbf{Soluzione:}
    \vspace{1em}
    
    \texttt{SELECT A1.Nazione, A2.Nazione\\FROM AEROPORTO AS A1\\ JOIN VOLO ON A1.Citt\`a = Citt\`aArr \\JOIN AEROPORTO AS A2 ON Citt\`aPart = A2.Citt\`a\\WHERE IdVolo = `AZ274';
}
    \end{frame}
%%%%%%%%%%%%%%%%%%%
\begin{frame}{Esercizio 4.14.3}
    Dato il seguente schema:
    \begin{itemize}
        \item AEROPORTO(Citt\`a, Nazione,NumPiste)
        \item VOLO(IdVolo,GiornoSett,Citt\`aPart,OraPart,
        Citt\`aArr,OraArr,TipoAereo)
        \item AEREO(TipoAereo,NumPasseggeri,QtaMerci)
    \end{itemize}
    \vspace{1em}
    
    \textbf{I tipi di aereo usati nei voli che partono da Torino;}
\end{frame}

\begin{frame}{Esercizio 4.14.3}
    \textbf{Soluzione:}
    \vspace{1em}
    
    \texttt{SELECT TipoAereo\\FROM VOLO\\WHERE Citt\`aPart='Torino'}
    \end{frame}
%%%%%%%%%%%%%%%%%%%
\begin{frame}{Esercizio 4.14.4}
    Dato il seguente schema:
    \begin{itemize}
        \item AEROPORTO(Citt\`a, Nazione,NumPiste)
        \item VOLO(IdVolo,GiornoSett,Citt\`aPart,OraPart,
        Citt\`aArr,OraArr,TipoAereo)
        \item AEREO(TipoAereo,NumPasseggeri,QtaMerci)
    \end{itemize}
    \vspace{1em}
    
    \textbf{I tipi di aereo e il corrispondente numero di passeggeri per i tipi di aereo usati nei voli che partono da Torino. Se la descrizione dell'aereo non \`e disponibile, visualizzare solamente il tipo;}
\end{frame}

\begin{frame}{Esercizio 4.14.4}
    \textbf{Soluzione:}
    \vspace{1em}
    
    \texttt{SELECT VOLO.TipoAereo, NumPasseggeri\\FROM VOLO LEFT JOIN AEREO ON VOLO.TipoAereo=aereo.TipoAereo\\WHERE Citt\`aPart= `Torino'}
\end{frame}
%%%%%%%%%%%%%%%%%%%
\begin{frame}{Esercizio 4.14.5}
    Dato il seguente schema:
    \begin{itemize}
        \item AEROPORTO(Citt\`a, Nazione,NumPiste)
        \item VOLO(IdVolo,GiornoSett,Citt\`aPart,OraPart,
        Citt\`aArr,OraArr,TipoAereo)
        \item AEREO(TipoAereo,NumPasseggeri,QtaMerci)
    \end{itemize}
    \vspace{1em}
    
    \textbf{Le citt\`a da cui partono voli internazionali;}
\end{frame}

\begin{frame}{Esercizio 4.14.5}
    \textbf{Soluzione:}
    \vspace{1em}
    
    \texttt{SELECT Citt\`aPart \\FROM AEROPORTO AS A1 JOIN VOLO ON Citt\`aPart=A1.Citt\`a\\JOIN AEROPORTO AS A2 ON Citt\`aArr=A2.Citt\`a\\WHERE A1.Nazione <> A2.Nazione}
\end{frame}
%%%%%%%%%%%%%%%%%%
\begin{frame}{Esercizio 4.14.6}
    Dato il seguente schema:
    \begin{itemize}
        \item AEROPORTO(Citt\`a, Nazione,NumPiste)
        \item VOLO(IdVolo,GiornoSett,Citt\`aPart,OraPart,
        Citt\`aArr,OraArr,TipoAereo)
        \item AEREO(TipoAereo,NumPasseggeri,QtaMerci)
    \end{itemize}
    \vspace{1em}
    
    \textbf{Le citt\`a da cui partono voli diretti a Bologna, ordinate alfabeticamente;}
\end{frame}

\begin{frame}{Esercizio 4.14.6}
    \textbf{Soluzione:}
    \vspace{1em}
    
    \texttt{SELECT Citt\`aPart\\FROM VOLO\\WHERE Citt\`aArr= `Bologna'\\ORDER BY Citt\`aPart}
\end{frame}
%%%%%%%%%%%%%%%%%%
\begin{frame}{Esercizio 4.14.7}
    Dato il seguente schema:
    \begin{itemize}
        \item AEROPORTO(Citt\`a, Nazione,NumPiste)
        \item VOLO(IdVolo,GiornoSett,Citt\`aPart,OraPart,
        Citt\`aArr,OraArr,TipoAereo)
        \item AEREO(TipoAereo,NumPasseggeri,QtaMerci)
    \end{itemize}
    \vspace{1em}
    
    \textbf{Il numero di voli internazionali che partono il gioved\`i da Napoli;}
\end{frame}

\begin{frame}{Esercizio 4.14.7}
    \textbf{Soluzione:}
    \vspace{1em}
    
    \texttt{SELECT COUNT(*)\\FROM VOLO JOIN AEROPORTO ON Citt\`aArr=Citt\`a\\WHERE Citt\`aPart = `Napoli' AND Nazione <> `Italia' \\AND GiornoSett= `Gioved\`i'}
\end{frame}
%%%%%%%%%%%%%%%%%%
\begin{frame}{Esercizio 4.14.8}
    Dato il seguente schema:
    \begin{itemize}
        \item AEROPORTO(Citt\`a, Nazione,NumPiste)
        \item VOLO(IdVolo,GiornoSett,Citt\`aPart,OraPart,
        Citt\`aArr,OraArr,TipoAereo)
        \item AEREO(TipoAereo,NumPasseggeri,QtaMerci)
    \end{itemize}
    \vspace{1em}
    
    \textbf{Il numero di voli internazionali che partono ogni settimana da citt\`a italiane;}
\end{frame}

\begin{frame}{Esercizio 4.14.8}
    \textbf{Soluzione :}
    \vspace{1em}
    
    \texttt{SELECT COUNT(*), Citt\`aPart\\FROM AEROPORTO AS A1 JOIN VOLO ON A1.Citt\`a=Citt\`aPart
\\JOIN AEROPORTO AS A2 ON Citt\`aArr=A2.Citt\`a\\WHERE A1.Nazione=`Italia' AND A2.Nazione <> `Italia'\\GROUP BY Citt\`aPart}
\end{frame}
%%%%%%%%%%%%%%%%%%
\begin{frame}{Esercizio 4.14.9}
    Dato il seguente schema:
    \begin{itemize}
        \item AEROPORTO(Citt\`a, Nazione,NumPiste)
        \item VOLO(IdVolo,GiornoSett,Citt\`aPart,OraPart,
        Citt\`aArr,OraArr,TipoAereo)
        \item AEREO(TipoAereo,NumPasseggeri,QtaMerci)
    \end{itemize}
    \vspace{1em}
    
    \textbf{Le citt\`a francesi da cui partono pi\`u di venti voli alla settimana diretti in Italia;}
\end{frame}

\begin{frame}{Esercizio 4.14.9}
    \textbf{Soluzione :}
    \vspace{1em}
    
    \texttt{SELECT Citt\`aPart\\FROM AEROPORTO AS A1 JOIN VOLO ON A1.Citt\`a=Citt\`aPart\\JOIN AEROPORTO AS A2 ON Citt\`aArr=A2.Citt\`a\\WHERE A1.Nazione=`Francia' AND A2.Nazione= `Italia'\\GROUP BY Citt\`aPart\\HAVING COUNT(*) >20}
\end{frame}
%%%%%%%%%%%%%%%%%%
\begin{frame}{Esercizio 4.14.10}
    Dato il seguente schema:
    \begin{itemize}
        \item AEROPORTO(Citt\`a, Nazione,NumPiste)
        \item VOLO(IdVolo,GiornoSett,Citt\`aPart,OraPart,
        Citt\`aArr,OraArr,TipoAereo)
        \item AEREO(TipoAereo,NumPasseggeri,QtaMerci)
    \end{itemize}
    \vspace{1em}
    
    \textbf{Gli aeroporti italiani che hanno solo voli interni. Rappresentare questa interrogazione in 3 modi}
\end{frame}
%
\begin{frame}{Esercizio 4.14.10.a}
    Dato il seguente schema:
    \begin{itemize}
        \item AEROPORTO(Citt\`a, Nazione,NumPiste)
        \item VOLO(IdVolo,GiornoSett,Citt\`aPart,OraPart,
        Citt\`aArr,OraArr,TipoAereo)
        \item AEREO(TipoAereo,NumPasseggeri,QtaMerci)
    \end{itemize}
    \vspace{1em}
    
    \textbf{Gli aeroporti italiani che hanno solo voli interni. Rappresentare questa interrogazione con operatori insiemistici (UNION, INTERSECT, EXCEPT).}
\end{frame}
\begin{frame}{Esercizio 4.14.10.a}
    \textbf{Soluzione:}
    \vspace{1em}
    
    \texttt{SELECT Citt\`aPart
    \\FROM VOLO JOIN AEROPORTO ON Citt\`aPart=Citt\`a WHERE Nazione = `Italia'
    \\EXCEPT
    \\SELECT Citt\`aPart
    \\FROM AEROPORTO AS A1 JOIN VOLO ON A1.Citt\`a=Citt\`aPart \\JOIN AEROPORTO AS A2 ON Citt\`aArr=A2.Citt\`a
    \\WHERE (A1.Nazione=`Italia' AND A2.Nazione <> `Italia' )}
\end{frame}
%%%%%%%%%%%%%%%%%%
\begin{frame}{Esercizio 4.14.10.b}
    Dato il seguente schema:
    \begin{itemize}
        \item AEROPORTO(Citt\`a, Nazione,NumPiste)
        \item VOLO(IdVolo,GiornoSett,Citt\`aPart,OraPart,
        Citt\`aArr,OraArr,TipoAereo)
        \item AEREO(TipoAereo,NumPasseggeri,QtaMerci)
    \end{itemize}
    \vspace{1em}
    
    \textbf{Gli aeroporti italiani che hanno solo voli interni. Rappresentare questa interrogazione con un interrogazione nidificata con l'operatore NOT IN.}
\end{frame}
\begin{frame}{Esercizio 4.14.10.b}
    \textbf{Soluzione:}
    \vspace{1em}
    
    \texttt{SELECT Citt\`aPart
    \\FROM VOLO JOIN AEROPORTO ON Citt\`aPart=Citt\`a
    \\WHERE Nazione= `Italia' and Citt\`aPart NOT IN
    \\( SELECT Citt\`aPart
    \\~~FROM AEROPORTO AS A1 JOIN VOLO ON A1.Citt\`a=Citt\`aPart \\~~JOIN AEROPORTO AS A2 ON Citt\`aArr=A2.Citt\`a
    \\~~WHERE  A1.Nazione=`Italia' AND A2.Nazione <> `Italia'  )}
\end{frame}
%%%%%%%%%%%%%%%%%%
\begin{frame}{Esercizio 4.14.10.c}
    Dato il seguente schema:
    \begin{itemize}
        \item AEROPORTO(Citt\`a, Nazione,NumPiste)
        \item VOLO(IdVolo,GiornoSett,Citt\`aPart,OraPart,
        Citt\`aArr,OraArr,TipoAereo)
        \item AEREO(TipoAereo,NumPasseggeri,QtaMerci)
    \end{itemize}
    \vspace{1em}
    
    \textbf{Gli aeroporti italiani che hanno solo voli interni con un interrogazione nidificata con l'operatore NOT EXISTS.}
\end{frame}
\begin{frame}{Esercizio 4.14.10.c}
    \textbf{Soluzione:}
    \vspace{1em}
    
    \texttt{SELECT Citt\`aPart
    \\FROM VOLO JOIN AEROPORTO AS A1 ON Citt\`aPart=Citt\`a
    \\WHERE Nazione=`Italia' AND
    \\NOT EXISTS 
    \\( SELECT *
    \\FROM VOLO JOIN AEROPORTO AS A2 ON A2.Citt\`a=Citt\`aArr
    \\WHERE A1.Citt\`a=Citt\`aPart AND A2.Nazione<>`Italia'
    \\)}
\end{frame}
%
\begin{frame}{Esercizio 4.14.12}
    Dato il seguente schema:
    \begin{itemize}
        \item AEROPORTO(Citt\`a, Nazione,NumPiste)
        \item VOLO(IdVolo,GiornoSett,Citt\`aPart,OraPart,
        Citt\`aArr,OraArr,TipoAereo)
        \item AEREO(TipoAereo,NumPasseggeri,QtaMerci)
    \end{itemize}
    \vspace{1em}
    
    \textbf{Le citt\`a che sono servite dall'aereo caratterizzato dal massimo numero di passeggeri;}
\end{frame}
\begin{frame}{Esercizio 4.14.12}
    \textbf{Soluzione:}
    \vspace{1em}
    
    \texttt{SELECT Citt\`aPart FROM VOLO JOIN AEREO ON VOLO.TipoAereo=AEREO.TipoAereo WHERE NumPasseggeri=      (SELECT MAX( NumPasseggeri ) FROM AEREO )
    \\ UNION
    \\ SELECT Citt\`aArr FROM VOLO JOIN AEREO ON VOLO.TipoAereo=AEREO.TipoAereo WHERE NumPasseggeri=      (SELECT MAX( NumPasseggeri ) FROM AEREO )}
\end{frame}
%%%%%%%%%%%%%%%%%%
\end{document}