\def\schemaEx3.10{\small \begin{align*}
    MATERIE &= (\underline{Codice}, Facolta, Denominazione, Professore)\\
    STUDENTI &= (\underline{Matricola}, Cognome, Nome, Facolta)\\
    PROFESSORI &= (\underline{Matricola}, Cognome, Nome)\\
    ESAMI &= (\underline{Studente}, \underline{Materia}, Voto, Data)\\
    PIANIDISTUDIO &= (\underline{Studente}, \underline{Materia}, Anno)\\
    \end{align*}}
\begin{frame}{Esercizio 3.10}
    \framesubtitle{3.10.1}
    \schemaEx3.10
    Trovare gli studenti che hanno riportato in almeno un esame una votazione pari a 30, mostrando, per ciascuno di essi, nome e cognome e data della prima di tali occasioni.
\end{frame}
%
\begin{frame}{Esercizio 3.10}
    \framesubtitle{Soluzione 3.10.1}
    \vspace*{-1.2cm}
    \schemaEx3.10
    {\small Studenti che hanno riportato in almeno un esame una votazione pari a 30, mostrando, per ciascuno di essi, nome e cognome e data della prima di tali occasioni:}
    \small
    \begin{gather*}
        \pi_{Nome,Cognome,Data}\\
        ((\pi_{Studente,Data}(\sigma_{Voto=30} (ESAMI))-\\
        {\pi_{Studente,Data} ((\sigma_{Voto=30}(ESAMI)~\bowtie_{(Studente=Studente1)~AND~(Data>Data1)}}\\
        {\rho_{Studente1,Corso1,Voto1,Data1~\leftarrow~Studente,Corso,Voto,Data}~\sigma_{Voto=30}(ESAMI)))}\\
        {\bowtie_{Studente~=~Matricola}(STUDENTE))}
    \end{gather*}
\end{frame}
%
% \begin{frame}{Esercizio 3.10}
%     \framesubtitle{3.10.2}
%     \schemaEx3.10
%     Per ogni insegnamento della facolt\`a di ingegneria, gli studenti che hanno superato l'esame nell'ultima seduta svoltaf.
% \end{frame}
% %
% \begin{frame}{Esercizio 3.10}
%     \framesubtitle{Soluzione 3.10.2}
%     \vspace*{-1.2cm}
%     \schemaEx3.10
%     {\small Per ogni insegnamento della facolt\`a di ingegneria, gli studenti che hanno superato l'esame nell'ultima seduta svolta:}
%     \small
%     \begin{gather*}
%         \sigma_{(Voto~\geq~18)~AND~(Facolta='Ingegneria')}\\
%         (\pi_{Cognome,Nome,Materia,Voto,Data}\\
%         (STUDENTI~\bowtie_{Studente=Matricola} (\pi_{Studente,Materia,Voto,Data}~(\sigma_{(D~\geq~Data)})\\
%         (ESAMI~\bowtie_{Materia=M}~(\rho_{S,M,V,D~\leftarrow~Studente,Materia,Voto,Data}(ESAMI))))))
%     \end{gather*}
% \end{frame}
%
% \begin{frame}{Esercizio 3.10}
%     \framesubtitle{3.10.3}
%     \schemaEx3.10
%     Trovare gli studenti che hanno superato tutti gli esami previsti dal rispettivo piano di studio.
% \end{frame}
% %
% \begin{frame}{Esercizio 3.10}
%     \framesubtitle{Soluzione 3.10.3}
%     \vspace*{-1.2cm}
%     \schemaEx3.10
%     {\small Gli studenti che hanno superato tutti gli esami previsti dal rispettivo piano di studio:}
%     \small
%     \begin{gather*}
%         \pi_{Studente}~(PIANIDISTUDIO)-\\
%         (\pi_{Studente}(\pi_{Studente,Materia}(PIANIDISTUDIO)~-~\pi_{Studente,Materia}(ESAMI)))
%     \end{gather*}
% \end{frame}
%
\begin{frame}{Esercizio 3.10}
    \framesubtitle{3.10.6}
    \schemaEx3.10
    Trovare nome e cognome degli studenti che hanno sostenuto almeno un esame con un professore che ha il
    loro stesso nome proprio.
\end{frame}
%
\begin{frame}{Esercizio 3.10}
    \framesubtitle{Soluzione 3.10.6}
    \vspace*{-1.2cm}
    \schemaEx3.10
    {\small Nome e cognome degli studenti che hanno sostenuto almeno un esame con un professore che ha il
    loro stesso nome proprio:}
    \small
    \begin{gather*}
        \pi_{Nome,Cognome}(\sigma_{Nome=NomeP}(\\
        (ESAMI \bowtie_{Studente=Matricola} STUDENTI) \bowtie_{Materia=Codice}\\
        (\rho_{NomeP,CognomeP \leftarrow Nome,Cognome} (PROFESSORI \bowtie_{Matricola=Professore} MATERIE))))        
    \end{gather*}
\end{frame}