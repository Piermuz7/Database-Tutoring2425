\def\schemaEx3.19{\small \begin{align*}
    CLIENTI &= (\underline{Codice}, Nome, Indirizzo, Citta)\\
    NOLEGGI &= (\underline{Cliente, Auto, DataPrelievo}, DataRestituzione)\\
    AUTOVETTURE &= (\underline{Targa}, Modello, Colore, AnnoImmatricolazione, CostoGiornaliero)
    \end{align*}
    $NOLEGGI$ con vincolo di integrit\`a referenziale fra l'attributo $Auto$ e la relazione $AUTOVETTURE$ e con vincolo di integrit\`a referenziale fra l'attributo $Cliente$ e la relazione $CLIENTI$.}

\def\soloschemaEx3.19{\small \begin{align*}
    CLIENTI &= (\underline{Codice}, Nome, Indirizzo, Citta)\\
    NOLEGGI &= (\underline{Cliente, Auto, DataPrelievo}, DataRestituzione)\\
    AUTOVETTURE &= (\underline{Targa}, Modello, Colore, AnnoImmatricolazione, CostoGiornaliero)
    \end{align*}}

\begin{frame}{Esercizio 3.19}
    \framesubtitle{3.19.1}
    \vspace{-3.cm}
    \schemaEx3.19
    \vspace{.3cm}

    Formulare l'interrogazione che restituisce i dati dei clienti che hanno noleggiato almeno un'autovettura nell'anno 2006.
\end{frame}
%
\begin{frame}{Esercizio 3.19}
    \framesubtitle{Soluzione 3.19.1}
    \vspace*{-2cm}
    \schemaEx3.19
    \vspace{.3cm}

    {\small Formulare l'interrogazione che restituisce i dati dei clienti che hanno noleggiato almeno un'autovettura nell'anno 2006:}
    \small
    \begin{gather*}
        A = \sigma_{(DataPrelievo~\geq~'01/01/2006')~AND~(DataPrelievo~\leq~'31/12/2006')}~(NOLEGGI);\\
        CLIENTI~\bowtie_{Codice~=~Cliente} A
    \end{gather*}
\end{frame}
%
\begin{frame}{Esercizio 3.19}
    \framesubtitle{3.19.2}
    \vspace{-3.cm}
    \schemaEx3.19
    \vspace{.3cm}

    Formulare l'interrogazione che restituisce i clienti che hanno noleggiato pi\`u di un'autovettura.
\end{frame}
%
\begin{frame}{Esercizio 3.19}
    \framesubtitle{Soluzione 3.19.2}
    \vspace*{-2cm}
    \schemaEx3.19
    \vspace{.3cm}

    {\small Formulare l'interrogazione che restituisce i clienti che hanno noleggiato pi\`u di un'autovettura:}
    \small
    \begin{gather*}
        \pi_{Cliente} (\sigma_{Auto~\neq~Auto'} (NOLEGGI~\bowtie_{Cliente=Cliente'}~(\rho_{X'~\leftarrow~X} NOLEGGI)))
    \end{gather*}
\end{frame}
%
\begin{frame}{Esercizio 3.19}
    \framesubtitle{Soluzione 3.19.2 - Step by step}
    \vspace*{-2cm}
    \soloschemaEx3.19
    %\vspace{.3cm}

    {\small Esempio:}

    \small
    \centering
    $NOLEGGI$

    \begin{tabular}{|c|c|c|}
    \hline
    \rowcolor{cyan!30} Cliente  & Auto & DataPrelievo \\
    \hline
    C1 & A1 & 2025-01-10 \\
    \hline
    C1 & A2 & 2025-01-15 \\
    \hline
    C2 & A3 & 2025-02-01 \\
    \hline
    C2 & A3 & 2025-03-01 \\
    \hline
    C3 & A4 & 2025-01-20 \\
    \hline
    \end{tabular}
\end{frame}
%
\begin{frame}{Esercizio 3.19}
    \framesubtitle{Soluzione 3.19.2 - Step by step}
    \vspace*{-1cm}
    \soloschemaEx3.19
    %\vspace{.3cm}

    {\small \textbf{STEP 1: Self-join: confronta ogni noleggio con un altro dello stesso cliente}}
    
    \vspace{.3cm}

    {\small \textbf{STEP 1.1: Ridenominazione di $NOLEGGI$ in $NOLEGGI'$:}}
    
    \small
    \begin{gather*}
        \rho_{X'~\leftarrow~X} (NOLEGGI)
    \end{gather*}
    
    \centering
    
    \begin{tabular}{|c|c|}
    \hline
    \rowcolor{cyan!30} Cliente' & Auto' \\
    \hline
    C1 & A1 \\
    \hline
    C1 & A2 \\
    \hline
    C2 & A3 \\
    \hline
    C2 & A3 \\
    \hline
    C3 & A4 \\
    \hline
    \end{tabular}
\end{frame}
%
\begin{frame}{Esercizio 3.19}
    \framesubtitle{Soluzione 3.19.2 - Step by step}
    \vspace*{-1cm}
    \soloschemaEx3.19
    %\vspace{.3cm}

    {\small \textbf{STEP 1: Self-join: confronta ogni noleggio con un altro dello stesso cliente}}
    
    \vspace{.3cm}

    {\small \textbf{STEP 1.2: Join tra NOLEGGI e NOLEGGI' su Cliente = Cliente' (le date non ci interessano):}}
    
    \vspace{.1cm}

    \small
    
    \centering
    
    \begin{tabular}{|c|c|c|c|}
    \hline
    \rowcolor{cyan!30} Cliente & Auto & Cliente' & Auto' \\
    \hline
    C1 & A1 & C1 & A2 \\
    \hline
    C1 & A1 & C1 & A2 \\
    \hline
    C1 & A2 & C1 & A1 \\
    \hline
    C1 & A2 & C1 & A2 \\
    \hline
    C2 & A3 & C2 & A3 \\
    \hline
    C2 & A3 & C2 & A3 \\
    \hline
    C3 & A4 & C3 & A4 \\
    \hline
    \end{tabular}
\end{frame}
%
\begin{frame}{Esercizio 3.19}
    \framesubtitle{Soluzione 3.19.2 - Step by step}
    \vspace*{-1cm}
    \soloschemaEx3.19
    %\vspace{.3cm}

    {\small \textbf{STEP 2: Selezione di Auto $\neq$ Auto'}}
    
    \vspace{.3cm}

    {\small Manteniamo solo i casi in cui lo stesso cliente ha noleggiato auto diverse:}
    
    \vspace{.1cm}

    \small
    
    \centering
    
    \begin{tabular}{|c|c|c|}
    \hline
    \rowcolor{cyan!30} Cliente & Auto & Auto' \\
    \hline
    C1 & A1 & A2 \\
    \hline
    C1 & A2 & A1 \\
    \hline
    \end{tabular}
\end{frame}
%
\begin{frame}{Esercizio 3.19}
    \framesubtitle{Soluzione 3.19.2 - Step by step}
    \vspace*{-1cm}
    \soloschemaEx3.19
    %\vspace{.3cm}

    {\small \textbf{STEP 3: Proiezione su Cliente}}
    
    \vspace{.3cm}

    {\small Proiettiamo solo l'attributo Cliente:}
    
    \small
    \begin{gather*}
        \pi_{Cliente} (\sigma_{Auto~\neq~Auto'} (NOLEGGI~\bowtie_{Cliente=Cliente'}~(\rho_{X'~\leftarrow~X} NOLEGGI)))
    \end{gather*}

    \vspace{.1cm}

    \small
    
    \centering
    
    \begin{tabular}{|c|}
    \hline
    \rowcolor{cyan!30} Cliente \\
    \hline
    C1 \\
    \hline
    \end{tabular}
\end{frame}
\begin{frame}{Esercizio 3.19}
    \framesubtitle{Soluzione 3.19.2 - Step by step}
    \vspace*{-1cm}
    \soloschemaEx3.19
    %\vspace{.3cm}

    {\small \textbf{RISULTATO FINALE: Clienti che hanno noleggiato pi\`u di un'autovettura}}
    
    \vspace{.3cm}
    
    \small
    \begin{gather*}
        \pi_{Cliente} (\sigma_{Auto~\neq~Auto'} (NOLEGGI~\bowtie_{Cliente=Cliente'}~(\rho_{X'~\leftarrow~X} NOLEGGI)))
    \end{gather*}

    \vspace{.1cm}

    \small
    
    \centering
    
    \begin{tabular}{|c|}
    \hline
    \rowcolor{cyan!30} Cliente \\
    \hline
    C1 \\
    \hline
    \end{tabular}
\end{frame}
%
\begin{frame}{Esercizio 3.19}
    \framesubtitle{3.19.3}
    \vspace{-3.cm}
    \schemaEx3.19
    \vspace{.3cm}

    Formulare l'interrogazione che restituisce i clienti che hanno noleggiato autovetture di un solo modello.
\end{frame}
%
\begin{frame}{Esercizio 3.19}
    \framesubtitle{Soluzione 3.19.3}
    \vspace*{-2cm}
    \schemaEx3.19
    \vspace{.3cm}

    {\small Formulare l'interrogazione che restituisce i clienti che hanno noleggiato autovetture di un solo modello:}
    \small
    \begin{gather*}
        V = NOLEGGI~\bowtie_{Auto~=~Targa}~AUTOVETTURE\\
        V_{1} = V~\bowtie_{Cliente~=~Cliente'} (\rho_{X'~\leftarrow~X} V)\\
        \pi_{Cliente}~NOLEGGI - \pi_{Cliente}~(\sigma_{Modello~\neq~Modello'} V_1)
    \end{gather*}
\end{frame}
%
\begin{frame}{Esercizio 3.19}
    \framesubtitle{Soluzione 3.19.3 - Step by step}
    \vspace*{-2cm}
    \soloschemaEx3.19
    %\vspace{.3cm}

    {\small Esempio:}
    \begin{minipage}{.33\textwidth}
    \small
    \centering
    $CLIENTI$

    \begin{tabular}{|c|c|}
    \hline
    \rowcolor{cyan!30} Codice  & Nome \\
    \hline
    C1 & Mario \\
    \hline
    C2 & Lucia \\
    \hline
    C3 & Andrea \\
    \hline
    \end{tabular}
    \end{minipage}%
    \begin{minipage}{.33\textwidth}
    \small
    \centering
    $AUTOVETTURE$
    \begin{tabular}{|c|c|}
    \hline
    \rowcolor{cyan!30} Targa  & Modello \\
    \hline
    A1 & Fiat500 \\
    \hline
    A2 & Golf \\
    \hline
    A3 & Golf \\
    \hline
    A4 & Panda \\
    \hline
    \end{tabular}
    \end{minipage}%
    \begin{minipage}{.33\textwidth}
    \small
    \centering
    $NOLEGGI$
    \begin{tabular}{|c|c|c|}
    \hline
    \rowcolor{cyan!30} Cliente  & Auto & DataPrelievo \\
    \hline
    C1 & A1 & 2025-01-10 \\
    \hline
    C1 & A4 & 2025-01-15 \\
    \hline
    C2 & A2 & 2025-02-01 \\
    \hline
    C2 & A3 & 2025-03-01 \\
    \hline
    C3 & A4 & 2025-01-20 \\
    \hline
    \end{tabular}
    \end{minipage}
    
\end{frame}
%
\begin{frame}{Esercizio 3.19}
    \framesubtitle{Soluzione 3.19.3 - Step by step}
    \vspace*{-1cm}
    \soloschemaEx3.19
    %\vspace{.3cm}

    {\small \textbf{STEP 1: Join tra NOLEGGI e AUTOVETTURE (chiamiamolo V):}}

    In questo modo, sappiamo che per ogni noleggio abbiamo il modello dell'auto noleggiata.

    \small
    \begin{gather*}
        V = NOLEGGI~\bowtie_{Auto~=~Targa}~AUTOVETTURE
    \end{gather*}
    
    \centering
    
    \begin{tabular}{|c|c|c|}
    \hline
    \rowcolor{cyan!30} Cliente  & Auto & Modello \\
    \hline
    C1 & A1 & Fiat500 \\
    \hline
    C1 & A4 & Panda \\
    \hline
    C2 & A2 & Golf \\
    \hline
    C2 & A3 & Golf \\
    \hline
    C3 & A4 & Panda \\
    \hline
    \end{tabular}
\end{frame}
%
\begin{frame}{Esercizio 3.19}
    \framesubtitle{Soluzione 3.19.3 - Step by step}
    \vspace*{-1cm}
    \soloschemaEx3.19
    %\vspace{.3cm}

    {\small \textbf{STEP 2: Self-join: confronta ogni noleggio con un altro dello stesso cliente}}

    Creiamo una copia di V con attributi rinominati: (Cliente', Auto', Modello') da confrontare con la versione originale.
    
    \begin{minipage}{.45\textwidth}
    \centering
    $V'$
    
    \begin{tabular}{|c|c|c|}
    \hline
    \rowcolor{cyan!30} Cliente' & Auto' & Modello' \\
    \hline
    C1 & A1 & Fiat500 \\
    \hline
    C1 & A4 & Panda \\
    \hline
    C2 & A2 & Golf \\
    \hline
    C2 & A3 & Golf \\
    \hline
    C3 & A4 & Panda \\
    \hline
    \end{tabular}
    \end{minipage}%
    \begin{minipage}{.45\textwidth}
    \centering
    \small

    \centering
    
        $V_{1} = V~\bowtie_{Cliente~=~Cliente'} (\rho_{X'~\leftarrow~X} V)$

    \begin{tabular}{|c|c|c|}
    \hline
    \rowcolor{cyan!30} Cliente & Modello & Modello' \\
    \hline
    C1 & Fiat500 & Fiat500 \\
    \hline
    C1 & Fiat500 & Panda \\
    \hline
    C1 & Panda & Fiat500 \\
    \hline
    C1 & Panda & Panda \\
    \hline
    \ldots & \ldots & \ldots \\
    \hline
    \end{tabular}
    \end{minipage}
\end{frame}
%
\begin{frame}{Esercizio 3.19}
    \framesubtitle{Soluzione 3.19.3 - Step by step}
    \vspace*{-1cm}
    \soloschemaEx3.19
    %\vspace{.3cm}

    {\small \textbf{STEP 3: Filtro le tuple in cui i modelli sono diversi}}

    \small
    \begin{gather*}
        \sigma_{Modello~\neq~Modello'} V_1
    \end{gather*}

    Applico il filtro: voglio vedere solo i clienti che hanno noleggiato due modelli diversi.

    \centering
    \begin{tabular}{|c|c|c|}
    \hline
    \rowcolor{cyan!30} Cliente & Modello & Modello' \\
    \hline
    C1 & Fiat500 & Panda \\
    \hline
    C1 & Panda & Fiat500 \\
    \hline
    \end{tabular}

    \vspace{.3cm}

    Solo C1 sopravvive a questo filtro, ovvero ha noleggiato modelli diversi.

\end{frame}
%
\begin{frame}{Esercizio 3.19}
    \framesubtitle{Soluzione 3.19.3 - Step by step}
    \vspace*{-1cm}
    \soloschemaEx3.19
    %\vspace{.3cm}

    {\small \textbf{STEP 4: Proiezione dei clienti che hanno noleggiato modelli diversi}}

    \small
    \begin{gather*}
        \pi_{Cliente} (\sigma_{Modello~\neq~Modello'} V_1)
    \end{gather*}

    \centering
    \begin{tabular}{|c|c|c|}
    \hline
    \rowcolor{cyan!30} Cliente \\
    \hline
    C1 \\
    \hline
    \end{tabular}

    \vspace{.3cm}

    Questo \`e l'insieme dei clienti che hanno noleggiato almeno due modelli diversi.

\end{frame}
%
\begin{frame}{Esercizio 3.19}
    \framesubtitle{Soluzione 3.19.3 - Step by step}
    \vspace*{-1cm}
    \soloschemaEx3.19
    %\vspace{.3cm}

    {\small \textbf{STEP 5: Tutti i clienti che hanno fatto almeno un noleggio}}

    \small
    \begin{gather*}
        \pi_{Cliente}~NOLEGGI
    \end{gather*}

    \centering
    \begin{tabular}{|c|}
    \hline
    \rowcolor{cyan!30} Cliente \\
    \hline
    C1 \\
    \hline
    C2 \\
    \hline
    C3 \\
    \hline
    \end{tabular}
\end{frame}
%
\begin{frame}{Esercizio 3.19}
    \framesubtitle{Soluzione 3.19.3 - Step by step}
    \vspace*{-1cm}
    \soloschemaEx3.19
    %\vspace{.3cm}

    {\small \textbf{STEP 6: Differenza tra tutti i clienti e quelli che hanno noleggiato modelli diversi}}

    \small
    \begin{gather*}
        \pi_{Cliente}~NOLEGGI - \pi_{Cliente}~(\sigma_{Modello~\neq~Modello'} V_1)
    \end{gather*}

    \centering
    
    \begin{tabular}{|c|}
    \hline
    \rowcolor{cyan!30} Cliente \\
    \hline
    C2 \\
    \hline
    C3 \\
    \hline
    \end{tabular}

    \vspace{.3cm}

    Questo \`e l'insieme dei clienti che hanno noleggiato solo un modello di auto:
    \begin{itemize}
        \item C2 ha noleggiato 2 auto, ma entrambe Golf;
        \item C3 ha noleggiato 1 auto, Panda.
    \end{itemize}

\end{frame}