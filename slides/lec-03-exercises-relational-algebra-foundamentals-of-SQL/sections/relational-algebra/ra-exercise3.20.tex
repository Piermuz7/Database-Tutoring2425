\def\schemaEx3.20{\small \begin{align*}
    FORNITORI &= (\underline{CF}, Nome, Indirizzo, Citta)\\
    PRODOTTI &= (\underline{CP}, Nome, Marca, Modello)\\
    CATALOGO &= (\underline{CF}, \underline{CP}, Costo)
    \end{align*}
    $CATALOGO$ con vincoli di integrit\`a referenziale fra l'attributo $CF$ e la relazione $FORNITORI$ e fra l'attributo $CP$ e la relazione $PRODOTTI$.}
\begin{frame}{Esercizio 3.20}
    \framesubtitle{3.20.1}
    \vspace{-3.cm}
    \schemaEx3.20
    \vspace{.3cm}

    Trovare Nome, Marca e Modello dei prodotti acquistabili con meno di 2000.
\end{frame}
%
\begin{frame}{Esercizio 3.20}
    \framesubtitle{Soluzione 3.20.1}
    \vspace*{-2cm}
    \schemaEx3.20
    \vspace{.3cm}

    {\small Nome, Marca e Modello dei prodotti acquistabili con meno di 2000:}
    \small
    \begin{gather*}
        \pi_{Nome,Marca,Modello} (\sigma_{Costo~<~2000}~PRODOTTI~\bowtie~CATALOGO)
    \end{gather*}
\end{frame}
%
\begin{frame}{Esercizio 3.20}
    \framesubtitle{3.20.2}
    \vspace{-3.cm}
    \schemaEx3.20
    \vspace{.3cm}

    Trovare i nomi dei fornitori che distribuiscono prodotti IBM (IBM \`e la marca di un prodotto).
\end{frame}
%
\begin{frame}{Esercizio 3.20}
    \framesubtitle{Soluzione 3.20.2}
    \vspace*{-2cm}
    \schemaEx3.20
    \vspace{.3cm}

    {\small Nomi dei fornitori che distribuiscono prodotti IBM (IBM \`e la marca di un prodotto):}
    \small
    \begin{gather*}
        R = (FORNITORI~\bowtie~CATALOGO)~\bowtie~(\pi_{CP,Marca}~PRODOTTI)\\
        \pi_{Nome} (\sigma_{Marca='IBM'} (R))
    \end{gather*}
\end{frame}
%
\begin{frame}{Esercizio 3.20}
    \framesubtitle{3.20.3}
    \vspace{-3.cm}
    \schemaEx3.20
    \vspace{.3cm}

    Trovare i codici di tutti i prodotti che sono forniti da almeno due fornitori.
\end{frame}
%
\begin{frame}{Esercizio 3.20}
    \framesubtitle{Soluzione 3.20.3}
    \vspace*{-2cm}
    \schemaEx3.20
    \vspace{.3cm}

    {\small Codici di tutti i prodotti che sono forniti da almeno due fornitori:}
    \small
    \begin{gather*}
        COPIACATALOGO = \rho_{CF'~\leftarrow~CF} (\pi_{CP,CF} CATALOGO)\\
        \pi_{CP} (\sigma_{CF~>~CF'} (CATALOGO~\bowtie~COPIACATALOGO))
    \end{gather*}
\end{frame}
%
\begin{frame}{Esercizio 3.20}
    \framesubtitle{3.20.4}
    \vspace{-3.cm}
    \schemaEx3.20
    \vspace{.3cm}

    Trovare i nomi dei fornitori che distribuiscono tutti i prodotti presenti nel catalogo.
\end{frame}
%
\begin{frame}{Esercizio 3.20}
    \framesubtitle{Soluzione 3.20.4}
    \vspace*{-1cm}
    \schemaEx3.20
    \vspace{.3cm}

    {\small Nomi dei fornitori che distribuiscono tutti i prodotti presenti nel catalogo:}
    \small
    \newline
    {\\Formuliamo innanzitutto l'interrogazione $R$ che trova i fornitori a cui manca almeno un prodotto del catalogo:\\}
    \vspace{-.5cm}
    \begin{gather*}
        R = \pi_{CF}((\pi_{CF} (FORNITORI)~\bowtie~\pi_{CP}(CATALOGO))~-~\pi_{CF,CP}(CATALOGO))\\
        \pi_{Nome}((\pi_{CF}(FORNITORI)~-~R)~\bowtie~FORNITORI)
    \end{gather*}
\end{frame}
%
\begin{frame}{Esercizio 3.20}
    \framesubtitle{3.20.5}
    \vspace{-3.cm}
    \schemaEx3.20
    \vspace{.3cm}

    Trovare i nomi dei fornitori che forniscono tutti i prodotti IBM presenti nel catalogo.
\end{frame}
%
\begin{frame}{Esercizio 3.20}
    \framesubtitle{Soluzione 3.20.5}
    \vspace*{-2cm}
    \schemaEx3.20
    \vspace{.3cm}

    {\small Nomi dei fornitori che forniscono tutti i prodotti IBM presenti nel catalogo:}
    \small
    \begin{gather*}
        CATALOGOIBM = \pi_{CP,CF} (\sigma_{Marca='IBM'} (CATALOGO~\bowtie~PRODOTTI))\\
        R = \pi_{CF}((\pi_{CF} (FORNITORI)~\bowtie~\pi_{CP}(CATALOGOIBM))~-~\pi_{CF,CP}(CATALOGOIBM))\\
        \pi_{Nome}((\pi_{CF}(FORNITORI)~-~R)~\bowtie~FORNITORI)
    \end{gather*}
\end{frame}