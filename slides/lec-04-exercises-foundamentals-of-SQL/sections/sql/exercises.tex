\begin{frame}{Esercizio 4.15.1}
    \texttt{DISCO(\underline{NroSerie}, TitoloAlbum,Anno,Prezzo)\\
    CONTIENE(\underline{NroSerieDisco, CodiceReg} , NroProg)\\
    ESECUZIONE(\underline{CodiceReg}, TitoloCanz, Anno)\\
    AUTORE(\underline{Nome, TitoloCanzone})\\
    CANTANTE(\underline{NomeCantante, CodiceReg})}
    \vspace{1em}
    
    \textbf{I cantautori (persone che hanno cantato e scritto la stessa canzone) il cui nome inizia per `D';}
\end{frame}

\begin{frame}{Esercizio 4.15.1}
    \textbf{Soluzione:}
    \vspace{1em}
    
    \texttt{SELECT NomeCantante\\FROM CANTANTE \\JOIN ESECUZIONE ON CANTANTE.CodiceReg=ESECUZIONE.CodiceReg\\JOIN AUTORE ON ESECUZIONE.TitoloCanz=AUTORE.TitoloCanzone\\WHERE Nome=NomeCantante and Nome like `d\%'}
    \end{frame}
%%%%%%%%%%%%%%%%%%%
\begin{frame}{Esercizio 4.15.2}
    \texttt{DISCO(\underline{NroSerie}, TitoloAlbum,Anno,Prezzo)\\
    CONTIENE(\underline{NroSerieDisco, CodiceReg} , NroProg)\\
    ESECUZIONE(\underline{CodiceReg}, TitoloCanz, Anno)\\
    AUTORE(\underline{Nome, TitoloCanzone})\\
    CANTANTE(\underline{NomeCantante, CodiceReg})}
    \vspace{1em}
    
    \textbf{I titoli dei dischi che contengono canzoni di cui non si conosce l`anno di registrazione;}
\end{frame}

\begin{frame}{Esercizio 4.15.2}
    \textbf{Soluzione:}
    \vspace{1em}
    
    \texttt{SELECT TitoloAlbum\\FROM DISCO\\JOIN CONTIENE ON DISCO.NroSerie=CONTIENE.NroSerieDisco
\\JOIN ESECUZIONE ON
CONTIENE.CodiceReg=ESECUZIONE.CodiceReg \\WHERE ESECUZIONE.anno IS NULL}
    \end{frame}
%%%%%%%%%%%%%%%%%%
\begin{frame}{Esercizio 4.15.4}
    \texttt{DISCO(\underline{NroSerie}, TitoloAlbum,Anno,Prezzo)\\
    CONTIENE(\underline{NroSerieDisco, CodiceReg} , NroProg)\\
    ESECUZIONE(\underline{CodiceReg}, TitoloCanz, Anno)\\
    AUTORE(\underline{Nome, TitoloCanzone})\\
    CANTANTE(\underline{NomeCantante, CodiceReg})}
    \vspace{1em}
    
    \textbf{Gli autori e i cantanti puri, ovvero autori che non hanno mai registrato una canzone e cantanti che non hanno mai scritto una canzone;}
\end{frame}

\begin{frame}{Esercizio 4.15.4}
    \textbf{Soluzione:}
    \vspace{1em}
    
    \texttt{SELECT Nomefrom AUTORE\\WHERE Nome NOT IN (SELECT NomeCantante
FROM CANTANTE)\\
UNION\\
SELECT NomeCantante\\FROM CANTANTE\\WHERE NomeCantante NOT IN (SELECT Nome FROM AUTORE)}
    \end{frame}
%%%%%%%%%%%%%%%%%%
\begin{frame}{Esercizio 4.15.5}
    \texttt{DISCO(\underline{NroSerie}, TitoloAlbum,Anno,Prezzo)\\
    CONTIENE(\underline{NroSerieDisco, CodiceReg} , NroProg)\\
    ESECUZIONE(\underline{CodiceReg}, TitoloCanz, Anno)\\
    AUTORE(\underline{Nome, TitoloCanzone})\\
    CANTANTE(\underline{NomeCantante, CodiceReg})}
    \vspace{1em}
    
    \textbf{I cantanti del disco che contiene il maggior numero di canzoni;}
\end{frame}

\begin{frame}{Esercizio 4.15.5}
    \textbf{Soluzione:}
    \vspace{1em}
    
    \texttt{CREATE VIEW DiscoNumerato AS\\ SELECT NroSerieDisco , COUNT(*) AS NumCanzoni
\\FROM CONTIENE\\GROUP BY NroSerieDisco}
\vspace{1em}

    \texttt{SELECT DISTINCT NomeCantante\\FROM CANTANTE JOIN CONTIENE ON CANTANTE.CodiceReg=CONTIENE.CodiceReg\\JOIN DiscoNumerato ON
CONTIENE.NroSerieDisco=DiscoNumerato.NroSerieDisco\\WHERE NumCanzoni= (SELECT MAX(NumCanzoni)\\ \hspace{9.5em}FROM DiscoNumerato)}
    \end{frame}
%%%%%%%%%%%%%%%%%%
\begin{frame}{Esercizio 4.15.6}
    \texttt{DISCO(\underline{NroSerie}, TitoloAlbum,Anno,Prezzo)\\
    CONTIENE(\underline{NroSerieDisco, CodiceReg} , NroProg)\\
    ESECUZIONE(\underline{CodiceReg}, TitoloCanz, Anno)\\
    AUTORE(\underline{Nome, TitoloCanzone})\\
    CANTANTE(\underline{NomeCantante, CodiceReg})}
    \vspace{1em}
    
    \textbf{Gli autori solisti di ``collezioni di successi'' (dischi in cui tutte le canzoni sono di un solo cantante e in cui almeno tre registrazioni sono di anni precedenti la pubblicazione del disco);}
\end{frame}

\begin{frame}{Esercizio 4.15.6}
    \textbf{Soluzione:}
    \vspace{1em}
    
    \texttt{SELECT NroSerie\\FROM DISCO\\WHERE NroSerie NOT IN\\(SELECT NroSerieDisco\\FROM CONTIENE JOIN CANTANTE AS S1 ON CONTIENE.CodiceReg=S1.CodiceReg\\JOIN CANTANTE AS S2 ON CONTIENE.CodiceReg =S2.CodiceReg\\WHERE S1.NomeCantante<>S2.NomeCantante)\\AND NroSerie IN\\(SELECT NroSerieDisco, COUNT(*)\\FROM CONTIENE JOIN ESECUZIONE ON CodiceReg=CodiceReg JOIN DISCO ON
DISCO.NroSerie=CONTIENE.NroSerieDisco)\\WHERE ESECUZIONE.Anno<DISCO.Anno
\\GROUP BY NroSerieDisco
HAVING COUNT(*) >=3)}
    \end{frame}
%%%%%%%%%%%%%%%%%%
\begin{frame}{Esercizio 4.15.7}
    \texttt{DISCO(\underline{NroSerie}, TitoloAlbum,Anno,Prezzo)\\
    CONTIENE(\underline{NroSerieDisco, CodiceReg} , NroProg)\\
    ESECUZIONE(\underline{CodiceReg}, TitoloCanz, Anno)\\
    AUTORE(\underline{Nome, TitoloCanzone})\\
    CANTANTE(\underline{NomeCantante, CodiceReg})}
    \vspace{1em}
    
    \textbf{I cantanti che non hanno mai registrato una canzone come solisti;}
\end{frame}

\begin{frame}{Esercizio 4.15.7}
    \textbf{Soluzione:}
    \vspace{1em}
    
    \texttt{SELECT DISTINCT NomeCantante\\FROM CANTANTE\\WHERE NomeCantante NOT IN(SELECT S1.NomeCantante\\\hspace{13em}FROM CANTANTE AS S1\\\hspace{13em}WHERE CodiceReg NOT IN(SELECT CodiceReg\\\hspace{24,5em}FROM CANTANTE S2\\\hspace{24,5em}WHERE S2.NomeCantante <> \\\hspace{24,5em}S1.NomeCantante))}
    \end{frame}
%%%%%%%%%%%%%%%%%%
\begin{frame}{Esercizio 4.15.8}
    \texttt{DISCO(\underline{NroSerie}, TitoloAlbum,Anno,Prezzo)\\
    CONTIENE(\underline{NroSerieDisco, CodiceReg} , NroProg)\\
    ESECUZIONE(\underline{CodiceReg}, TitoloCanz, Anno)\\
    AUTORE(\underline{Nome, TitoloCanzone})\\
    CANTANTE(\underline{NomeCantante, CodiceReg})}
    \vspace{1em}
    
    \textbf{I cantanti che non hanno mai inciso un disco in cui comparissero come unici cantanti;}
\end{frame}

\begin{frame}{Esercizio 4.15.8}
    \textbf{Soluzione:}
    \vspace{1em}
    
    \texttt{CREATE VIEW CantantiUnDisco AS
\\SELECT NomeCantante
\\FROM CONTIENE JOIN CANTANTE ONCONTIENE.CodiceReg=CANTANTE.CodiceReg\\WHERE NroSerieDisco NOT IN
\\(select NroSerieDisco
FROM CONTIENE JOIN CANTANTE AS S1 ON
CONTIENE.CodiceReg=S1.CodiceReg\\JOIN CANTANTE AS S2 ON CodiceReg=S2.CodiceReg\\WHERE S1.NomeCantante<>S2.NomeCantante)\\
\vspace{1em}
SELECT NomeCantante\\FROM CANTANTE
WHERE NomeCantante NOT IN (SELECT NomeCantante\\FROM CantantiUnDisco)}
    \end{frame}
%%%%%%%%%%%%%%%%%%
\begin{frame}{Esercizio 4.15.9}
    \texttt{DISCO(\underline{NroSerie}, TitoloAlbum,Anno,Prezzo)\\
    CONTIENE(\underline{NroSerieDisco, CodiceReg} , NroProg)\\
    ESECUZIONE(\underline{CodiceReg}, TitoloCanz, Anno)\\
    AUTORE(\underline{Nome, TitoloCanzone})\\
    CANTANTE(\underline{NomeCantante, CodiceReg})}
    \vspace{1em}
    
    \textbf{I cantanti che hanno sempre registrato canzoni come solisti.}
\end{frame}

\begin{frame}{Esercizio 4.15.9}
    \textbf{Soluzione:}
    \vspace{1em}
    
    \texttt{SELECT NomeCantante\\FROM CANTANTE\\WHERE NomeCantante NOT IN(SELECT S1.NomeCantante\\\hspace{13em}FROM CANTANTE AS S1 JOIN ESECUZIONE ON\\\hspace{12,5em} CodiceReg=S1.CodiceReg JOIN CANTANTE AS S2 ON\\\hspace{12,5em} CodiceReg=S2.CodiceReg)\\\hspace{13em}WHERE S1.NomeCantante<>S2.NomeCantante)}
    \end{frame}
%%%%%%%%%%%%%%%%%%
\begin{frame}{Esercizio 4.19}
    \texttt{PRODOTTI(\underline{Codice}, Nome, Categoria)\\
    VENDITE(\underline{CodiceProdotto}, Data, Incasso)}
    \vspace{1em}
    \begin{table}[h]
\centering
\begin{minipage}{.45\textwidth}
\centering
\begin{tabular}{|c|c|c|}
\hline
\rowcolor{cyan!30} \multicolumn{3}{|c|}{Prodotti} \\
\hline
\rowcolor{cyan!30} Codice  & Nome & Categoria \\
\hline
101 & A & Bevanda \\
102 & B & Bevanda \\
103 & C & Pasta \\
104 & D & Biscotti \\
\hline
\end{tabular}
\end{minipage}%
\begin{minipage}{.45\textwidth}
\centering
\begin{tabular}{|c|c|c|}
\hline
\rowcolor{cyan!30} \multicolumn{3}{|c|}{Vendite} \\
\hline
\rowcolor{cyan!30} CodiceProdotto  & Data & Incasso\\
\hline
101 & 24/11/2008 & 2.000 \\
101 & 25/11/2008 & 1.000 \\
102 & 23/11/2008 & 2.500 \\
102 & 24/11/2008 & 4.000 \\
103 & 25/11/2008 & 1.320 \\
\hline
\end{tabular}
\end{minipage}
\end{table}
\end{frame}

\begin{frame}{Esercizio 4.19.1}
    \textbf{Mostrare il risultato della seguente interrogazione:}\\
    \texttt{SELECT Codice\\FROM Prodotti\\WHERE NOT EXISTS(SELECT *\\\hspace{8,5em}FROM Vendite\\\hspace{8,5em}WHERE CodiceProd=Codice)}
\end{frame}

\begin{frame}{Esercizio 4.19.1}
    \textbf{Mostrare il risultato della seguente interrogazione:}\\
    \texttt{SELECT Codice\\FROM Prodotti\\WHERE NOT EXISTS(SELECT *\\\hspace{8,5em}FROM Vendite\\\hspace{8,5em}WHERE CodiceProd=Codice)}
    \vspace{1em}
    \begin{table}[h]
\centering
\begin{minipage}{.45\textwidth}
\centering
\begin{tabular}{|c|}
\hline
\rowcolor{cyan!30} Codice \\
\hline
104\\
\hline
\end{tabular}
\end{minipage}%
\end{table}
\end{frame}

%%%%%%%%%%%%%%%%%%
\begin{frame}{Esercizio 4.19.2}
    \textbf{Mostrare il risultato della seguente interrogazione:}\\
    \texttt{SELECT Codice\\FROM Prodotti\\WHERE NOT EXISTS(SELECT *\\\hspace{8,5em}FROM Vendite\\\hspace{8,5em}WHERE Data = `2008-11-24' AND CodiceProd=Codice)}
\end{frame}

\begin{frame}{Esercizio 4.19.2}
    \textbf{Mostrare il risultato della seguente interrogazione:}\\
    \texttt{SELECT Codice\\FROM Prodotti\\WHERE NOT EXISTS(SELECT *\\\hspace{8,5em}FROM Vendite\\\hspace{8,5em}WHERE Data = `2008-11-24' AND CodiceProd=Codice)}
    \begin{table}[h]
    \centering
    \begin{minipage}{.45\textwidth}
    \centering
    \begin{tabular}{|c|}
    \hline
    \rowcolor{cyan!30} Codice \\
    \hline
    103\\
    \hline
    104\\
    \hline
    \end{tabular}
    \end{minipage}%
    \end{table}
\end{frame}
%%%%%%%%%%%%%%%%%%
\begin{frame}{Esercizio 4.19.3}
    \textbf{Mostrare il risultato della seguente interrogazione:}\\
    \texttt{SELECT Codice\\FROM Prodotti\\WHERE NOT EXISTS(SELECT *\\\hspace{8,5em}FROM Vendite\\\hspace{8,5em}WHERE Data = `2008-11-24')}
\end{frame}

\begin{frame}{Esercizio 4.19.3}
    \textbf{Mostrare il risultato della seguente interrogazione:}\\
    \texttt{SELECT Codice\\FROM Prodotti\\WHERE NOT EXISTS(SELECT *\\\hspace{8,5em}FROM Vendite\\\hspace{8,5em}WHERE Data = `2008-11-24')}
    \begin{table}[h]
    \centering
    \begin{minipage}{.45\textwidth}
    \centering
    \begin{tabular}{|c|}
    \hline
    \rowcolor{cyan!30} Codice \\
    \hline
    103\\
    \hline
    \end{tabular}
    \end{minipage}%
    \end{table}
\end{frame}
%%%%%%%%%%%%%%%%%%
\begin{frame}{Esercizio 4.25}
    \textbf{Si supponga di avere un sistema di basi di dati non in grado di
eseguire left join espliciti. Si traduca la query SQL seguente in un suo equivalente
supportato da tale dbms.}
    \vspace{1em}
    
    \texttt{SELECT *\\FROM Animali \\LEFT JOIN Padroni ON (padrone = codiceFiscale AND etaPadrone>40)}
\end{frame}

\begin{frame}{Esercizio 4.25}
    \textbf{Si supponga di avere un sistema di basi di dati non in grado di
eseguire left join espliciti. Si traduca la query SQL seguente in un suo equivalente
supportato da tale dbms.}
    \vspace{1em}
    
    \texttt{SELECT *\\FROM Animali \\LEFT JOIN Padroni ON (padrone = codiceFiscale AND etaPadrone>40)}
    \vspace{1em}
    
    \textbf{Soluzione:}\\
    \texttt{SELECT *\\FROM Animali, Padroni\\WHERE (padrone = codiceFiscale AND etaPadrone>40)
OR padrone = NULL.}
\end{frame}
%%%%%%%%%%%%%%%%%%
\begin{frame}{Esercizio 4.26.1} 
    \texttt{AEROPORTI(\underline{Codice}, Citta, Nome)\\
    AEREI(\underline{Codice}, Nome, NumeroPosti)\\
    VOLI(\underline{Destinazione, Aereo}, OraPart, OraArr)}
    \vspace{1em}
    
    \textbf{L`interrogazione che trova le citt\`a raggiungibili con un volo diretto che utilizzi
un aereo con almeno 200 posti;}
\end{frame}

\begin{frame}{Esercizio 4.26.1}
    \textbf{Soluzione:}
    \vspace{1em}
    
    \texttt{SELECT DISTINCT Citta\\FROM Aeroporti JOIN Voli ON Aeroporti.Codice = Voli.Destinazione\\JOIN Aerei on Aerei.Codice = Voli.Aereo\\WHERE Aerei.NumeroPosti >= 200}
    \end{frame}
%%%%%%%%%%%%%%%%%%
\begin{frame}{Esercizio 4.26.2} 
    \texttt{AEROPORTI(\underline{Codice}, Citta, Nome)\\
    AEREI(\underline{Codice}, Nome, NumeroPosti)\\
    VOLI(\underline{Destinazione, Aereo}, OraPart, OraArr)}
    \vspace{1em}
    
    \textbf{L`interrogazione che trova le citt\`a raggiungibili con voli diretti e, per ciascuna,
mostra il numero di tali voli.}
\end{frame}

\begin{frame}{Esercizio 4.26.2}
    \textbf{Soluzione:}
    \vspace{1em}
    
    \texttt{SELECT Citta, COUNT(*)\\FROM Aereoporti JOIN Voli ON Aeroporti.Codice = Voli.Destinazione \\GROUP BY Citta}
    \end{frame}
%%%%%%%%%%%%%%%%%%
\begin{frame}{Esercizio (ASTA)} 
    \texttt{QUADRO(\underline{CODICE\#}, AUTORE, PERIODO)\\
    MOSTRA(\underline{CODICE*}, NOME, ANNO, ORGANIZZATORE)\\
    ASTA(\underline{MOSTRA*, QUADRO\#}, PREZZO)}
    \vspace{1em}
    
    \textbf{La lista degli autori , senza duplicati, che hanno venduto
quadri di prezzo maggiore di 1000;}
\end{frame}

\begin{frame}{Esercizio (ASTA)}
    \textbf{Soluzione:}
    \vspace{1em}
    
    \texttt{SELECT DISTINCT AUTORE\\FROM ASTA JOIN QUADRO ON QUADRO.CODICE=ASTA.QUADRO\\WHERE
PREZZO > 1000}
    \end{frame}
%%%%%%%%%%%%%%%%%%
\begin{frame}{Esercizio (ASTA) } 
    \texttt{QUADRO(\underline{CODICE\#}, AUTORE, PERIODO)\\
    MOSTRA(\underline{CODICE*}, NOME, ANNO, ORGANIZZATORE)\\
    ASTA(\underline{MOSTRA*, QUADRO\#}, PREZZO)}
    \vspace{1em}
    
    \textbf{La lista degli autori che hanno venduto almeno 3 quadri;}
\end{frame}

\begin{frame}{Esercizio (ASTA)}
    \textbf{Soluzione:}
    \vspace{1em}
    
    \texttt{SELECT AUTORE, COUNT(*)\\FROM QUADRO,ASTA\\WHERE QUADRO.CODICE=ASTA.QUADRO\\GROUP BY AUTORE
\\HAVING COUNT(*) >= 3}
    \end{frame}
%%%%%%%%%%%%%%%%%%
\begin{frame}{Esercizio (ASTA) } 
    \texttt{QUADRO(\underline{CODICE\#}, AUTORE, PERIODO)\\
    MOSTRA(\underline{CODICE*}, NOME, ANNO, ORGANIZZATORE)\\
    ASTA(\underline{MOSTRA*, QUADRO\#}, PREZZO)}
    \vspace{1em}
    
    \textbf{I nomi delle mostre dove non si \`e venduto nessun quadro;}
\end{frame}

\begin{frame}{Esercizio (ASTA)}
    \textbf{Soluzione:}
    \vspace{1em}
    
    \texttt{SELECT NOME \\FROM MOSTRA \\WHERE CODICE NOT IN (SELECT MOSTRA FROM ASTA)}
    \end{frame}
%%%%%%%%%%%%%%%%%
\begin{frame}{Esercizio (ASTA) } 
    \texttt{QUADRO(\underline{CODICE\#}, AUTORE, PERIODO)\\
    MOSTRA(\underline{CODICE*}, NOME, ANNO, ORGANIZZATORE)\\
    ASTA(\underline{MOSTRA*, QUADRO\#}, PREZZO)}
    \vspace{1em}
    
    \textbf{I nomi delle mostre dove si sono venduti pi\`u quadri.}
\end{frame}

\begin{frame}{Esercizio (ASTA)}
    \textbf{Soluzione:}
    \vspace{1em}
    
    \texttt{SELECT MOSTRA.NOME, COUNT(*) \\FROM ASTA,MOSTRA\\WHERE MOSTRA.CODICE = ASTA.MOSTRA\\GROUP BY MOSTRA.NOME\\HAVING COUNT(*) = (SELECT MAX(NQUADRI)\\\hspace{9,5em}FROM
(SELECT ASTA.MOSTRA, COUNT(*) AS NQUADRI \\\hspace{12,5em}FROM ASTA \\\hspace{12,5em}GROUP BY ASTA.MOSTRA))}
    \end{frame}
%%%%%%%%%%%%%%%%%
\begin{frame}{Esercizio (IMMOBILI) } 
    \texttt{PERSONE(\underline{CF\#}, Nome, Cognome, DataNascita)\\
    Immobili(\underline{Codice*}, Via, NumeroCivico, Citta, Valore)\\
    Proprieta(\underline{Persona\#, Immobile*}, Percentuale)}
    \vspace{1em}
    
    \textbf{Il Nome e Cognome delle persone di cui non \`e dato sapere la data di nascita;}
\end{frame}

\begin{frame}{Esercizio (IMMOBILI)}
    \textbf{Soluzione:}
    \vspace{1em}
    
    \texttt{SELECT Nome , Cognome\\FROM Persone\\WHERE DataNascita IS NULL}
    \end{frame}
%%%%%%%%%%%%%%%%%
\begin{frame}{Esercizio (IMMOBILI) } 
    \texttt{PERSONE(\underline{CF\#}, Nome, Cognome, DataNascita)\\
    Immobili(\underline{Codice*}, Via, NumeroCivico, Citta, Valore)\\
    Proprieta(\underline{Persona\#, Immobile*}, Percentuale)}
    \vspace{1em}
    
    \textbf{Il Nome e Cognome delle persone che non hanno immobili;}
\end{frame}

\begin{frame}{Esercizio (IMMOBILI)}
    \textbf{Soluzione:}
    \vspace{1em}
    
    \texttt{SELECT Nome , Cognome\\FROM Persone\\WHERE CF NOT IN (SELECT Persona FROM Proprieta)}
\end{frame}
%%%%%%%%%%%%%%%%%
\begin{frame}{Esercizio (IMMOBILI) } 
    \texttt{PERSONE(\underline{CF\#}, Nome, Cognome, DataNascita)\\
    Immobili(\underline{Codice*}, Via, NumeroCivico, Citta, Valore)\\
    Proprieta(\underline{Persona\#, Immobile*}, Percentuale)}
    \vspace{1em}
    
    \textbf{Nome e Cognome dei proprietari che posseggono immobili con percentuali inferioni a 0,5 e con valore compreso tra a 50000 e 100000;}
\end{frame}

\begin{frame}{Esercizio (IMMOBILI)}
    \textbf{Soluzione:}
    \vspace{1em}
    
    \texttt{SELECT Persone.CF , Persone.Nome , Persone.Cognome\\FROM Persone JOIN Proprieta ON Persone.CF = Proprieta.Persona JOIN Immobili ON Proprieta.Immobile = Immobili.Codice\\WHERE Proprieta.Percentuale < 0.5 AND Immobili.Valore BETWEEN 50000 AND 100000}
\end{frame}
%%%%%%%%%%%%%%%%%
\begin{frame}{Esercizio (IMMOBILI) } 
    \texttt{PERSONE(\underline{CF\#}, Nome, Cognome, DataNascita)\\
    Immobili(\underline{Codice*}, Via, NumeroCivico, Citta, Valore)\\
    Proprieta(\underline{Persona\#, Immobile*}, Percentuale)}
    \vspace{1em}
    
    \textbf{La quantit\`a degli immobili collocati nella citt\`a di Milano.}
\end{frame}

\begin{frame}{Esercizio (IMMOBILI)}
    \textbf{Soluzione:}
    \vspace{1em}
    
    \texttt{SELECT COUNT(*) \\FROM Immobili JOIN Proprieta ON Immobili.Codice= Proprieta.Immobile \\WHERE Citta = `Milano'}
\end{frame}
%%%%%%%%%%%%%%%%%
\begin{frame}{Esercizio (CALCIO) } 
    \texttt{STADIO(\underline{Nome\#}, Citta, Capienza)\\
    INCONTRO(\underline{NomeStadio\#,Squadra1*, Squadra2*}, Data, Ora)\\
    NAZIONALE(\underline{Paese*}, Continente)}
    \vspace{1em}
    
    \textbf{Il nome del/degli Stadio/i con capienza minima;}
\end{frame}

\begin{frame}{Esercizio (CALCIO)}
    \textbf{Soluzione:}
    \vspace{1em}
    
    \texttt{SELECT Nome \\FROM STADIO\\WHERE Capienza = (SELECT MIN(Capienza) FROM STADIO)}
\end{frame}
%%%%%%%%%%%%%%%%%
\begin{frame}{Esercizio (CALCIO) } 
    \texttt{STADIO(\underline{Nome\#}, Citta, Capienza)\\
    INCONTRO(\underline{NomeStadio\#,Squadra1*, Squadra2*}, Data, Ora)\\
    NAZIONALE(\underline{Paese*}, Continente)}
    \vspace{1em}
    
    \textbf{La capienza media degli stadi della citt\`a di `Roma';}
\end{frame}

\begin{frame}{Esercizio (CALCIO)}
    \textbf{Soluzione:}
    \vspace{1em}
    
    \texttt{SELECT AVG(Capienza)\\FROM STADIO\\WHERE Citta = `Roma'}
\end{frame}
%%%%%%%%%%%%%%%%%
\begin{frame}{Esercizio (CALCIO) } 
    \texttt{STADIO(\underline{Nome\#}, Citta, Capienza)\\
    INCONTRO(\underline{NomeStadio\#,Squadra1*, Squadra2*}, Data, Ora)\\
    NAZIONALE(\underline{Paese*}, Continente)}
    \vspace{1em}
    
    \textbf{Il numero delle partite giocate nello stadio `Olimpico';}
\end{frame}

\begin{frame}{Esercizio (CALCIO)}
    \textbf{Soluzione:}
    \vspace{1em}
    
    \texttt{SELECT COUNT(*)\\FROM INCONTRO\\WHERE NomeStadio = `Olimpico'}
\end{frame}
%%%%%%%%%%%%%%%%%
\begin{frame}{Esercizio (CALCIO) } 
    \texttt{STADIO(\underline{Nome\#}, Citta, Capienza)\\
    INCONTRO(\underline{NomeStadio\#,Squadra1*, Squadra2*}, Data, Ora)\\
    NAZIONALE(\underline{Paese*}, Continente)}
    \vspace{1em}
    
    \textbf{Il Nome dello Stadio, Data, Ora delle partite dove gioca la `Scozia' ordinato per Data e ora;}
\end{frame}

\begin{frame}{Esercizio (CALCIO)}
    \textbf{Soluzione:}
    \vspace{1em}
    
    \texttt{SELECT NomeStadio, Data, Ora\\FROM INCONTRO\\WHERE (Squadra1 = `Scozia' OR Squadra2 = `Scozia') \\ORDER BY Data, Ora}
\end{frame}
%%%%%%%%%%%%%%%%%
\begin{frame}{Esercizio (CALCIO) } 
    \texttt{STADIO(\underline{Nome\#}, Citta, Capienza)\\
    INCONTRO(\underline{NomeStadio\#,Squadra1*, Squadra2*}, Data, Ora)\\
    NAZIONALE(\underline{Paese*}, Continente)}
    \vspace{1em}
    
    \textbf{La capienza totale degli stadi;}
\end{frame}

\begin{frame}{Esercizio (CALCIO)}
    \textbf{Soluzione:}
    \vspace{1em}
    
    \texttt{SELECT SUM(Capienza)\\FROM STADIO}
\end{frame}
%%%%%%%%%%%%%%%%%
\begin{frame}{Esercizio (CALCIO) } 
    \texttt{STADIO(\underline{Nome\#}, Citta, Capienza)\\
    INCONTRO(\underline{NomeStadio\#,Squadra1*, Squadra2*}, Data, Ora)\\
    NAZIONALE(\underline{Paese*}, Continente)}
    \vspace{1em}
    
    \textbf{Le squadre che giocano al `Meazza';}
\end{frame}

\begin{frame}{Esercizio (CALCIO)}
    \textbf{Soluzione:}
    \vspace{1em}
    
    \texttt{SELECT Squadra1 as squadra\\FROM INCONTRO\\WHERE NomeStadio = `Meazza'
\\UNION\\SELECT Squadra2 as squadra\\FROM INCONTRO\\WHERE NomeStadio = `Meazza'}
\end{frame}
%%%%%%%%%%%%%%%%%
\begin{frame}{Esercizio (CALCIO) } 
    \texttt{STADIO(\underline{Nome\#}, Citta, Capienza)\\
    INCONTRO(\underline{NomeStadio\#,Squadra1*, Squadra2*}, Data, Ora)\\
    NAZIONALE(\underline{Paese*}, Continente)}
    \vspace{1em}
    
    \textbf{Le squadre che non giocano allo stadio `Flaminio';}
\end{frame}

\begin{frame}{Esercizio (CALCIO)}
    \textbf{Soluzione:}
    \vspace{1em}
    
    \texttt{SELECT Paese\\FROM NAZIONALE\\WHERE Paese NOT IN (SELECT Squadra1 \\\hspace{10em}FROM INCONTRO \\\hspace{10em}WHERE NomeStadio = `Flaminio'
\\\hspace{10em}UNION\\\hspace{10em}SELECT Squadra2 \\\hspace{10em}FROM INCONTRO \\\hspace{10em}WHERE NomeStadio = `Flaminio')}
\end{frame}
%%%%%%%%%%%%%%%%%
\begin{frame}{Esercizio (CALCIO) } 
    \texttt{STADIO(\underline{Nome\#}, Citta, Capienza)\\
    INCONTRO(\underline{NomeStadio\#,Squadra1*, Squadra2*}, Data, Ora)\\
    NAZIONALE(\underline{Paese*}, Continente)}
    \vspace{1em}
    
    \textbf{l Nome dello Stadio, Data, Ora delle partite dove giocano squadre di continenti diversi.}
\end{frame}

\begin{frame}{Esercizio (CALCIO)}
    \textbf{Soluzione:}
    \vspace{1em}
    
    \texttt{SELECT NomeStadio, Data, Ora\\FROM INCONTRO, NAZIONALE N1, NAZIONALE N2\\WHERE Squadra1 = N1.Paese AND Squadra2 = N2.Paese AND NOT (N1.Continente = N2.Continente)}
\end{frame}
