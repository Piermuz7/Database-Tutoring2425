\def\schemaPersonaMatrimonio{\small \begin{align*}
    &PERSONA(\underline{CF}, cognome, nome, annoNascita, sesso)\\
    &MATRIMONIO(\underline{CFmoglie}, \underline{CFmarito}, anno, luogo)
    \end{align*}}
\def\schemaNegoziProdottiListino{\small \begin{align*}
    &NEGOZI(\underline{IDnegozio}, nome, citta)\\
    &PRODOTTI(\underline{codProdotto}, nomeProdotto, marca)\\
    &LISTINO(\underline{negozio}, \underline{prodotto}, prezzo)
    \end{align*}}
\def\schemaVenditeClientiDettagliVendite{\small \begin{align*}
    &Vendite(\underline{NumeroScontrino}, Data)\\
    &Clienti(\underline{Codice}, Cognome, Eta)\\
    &DettagliVendite(\underline{NumeroScontrino}, \underline{Riga}, Prodotto, Importo, Cliente)
    \end{align*}}
\def\schemaClientiNoleggiAuto{\small \begin{align*}
    &CLIENTI(\underline{Codice}, Nome, Indirizzo, Citta)\\
    &NOLEGGI(\underline{Cliente}, \underline{Auto}, DataPrelievo, DataRestituzione)\\
    &AUTOVETTURE(\underline{Targa}, Modello, Colore, AnnoImmatricolazione, CostoGiornaliero)
    \end{align*}}
\def\schemaProdottiVendite{\small \begin{align*}
    &PRODOTTI(\underline{Codice}, Descrizione, Marca)\\
    &VENDITE(\underline{Prodotto}, \underline{Mese}, \underline{Anno}, Quantita)  
    \end{align*}}
\def\schemaRivisteArticoli{\small \begin{align*}
    &RIVISTE(\underline{CodR}, NomeR, Editore)\\
    &ARTICOLI(\underline{CodA}, Titolo, Argomento, CodR)
    \end{align*}}
\def\schemaMuseiOpereArtisti{\small \begin{align*}
    &MUSEI (\underline{IdMuseo}, NomeM, Citta, Nazione)\\
    &OPERE (\underline{IdOpera}, Titolo, Tipo, AnnoO, IdMuseo^{*}, NomeA^{*})\\
    &ARTISTI (\underline{NomeA}, Nazione, AnnoN, AnnoM)
    \end{align*}}
\begin{frame}[fragile]{Random: Es. 1}
Dato il seguente schema:
\schemaPersonaMatrimonio

Trovare cognome e Nome delle persone di sesso femminile in ordine alfabetico di cognome.
\pause
\begin{lstlisting}
SELECT Nome, Cognome
FROM Persona
WHERE Sesso = `F'
ORDER BY cognome;
\end{lstlisting}
\end{frame}
%
\begin{frame}[fragile]{Random: Es. 2}
Dato il seguente schema:
\schemaPersonaMatrimonio

Trovare il numero di persone di ogni sesso.
\pause
\begin{lstlisting}
SELECT Sesso, COUNT(*)
FROM Persona
GROUP BY Sesso;
\end{lstlisting}
\end{frame}
%
\begin{frame}[fragile]{Random: Es. 3}
Dato il seguente schema:
\schemaPersonaMatrimonio

Trovare nome e cognome degli uomini pi\`u giovani della loro moglie.
\pause
\begin{lstlisting}
SELECT m.Nome, m.Cognome
FROM Matrimonio JOIN Persona AS m ON CFmarito = m.CF
JOIN Persona AS f ON CFmoglie = f.CF
WHERE m.AnnoNascita > f.AnnoNascita;
\end{lstlisting}
\end{frame}
%
\begin{frame}[fragile]{Random: Es. 4}
Dato il seguente schema:
\schemaPersonaMatrimonio

Trovare il codice fiscale degli uomini che si sono sposati pi\`u volte.
\pause
\begin{lstlisting}
SELECT CFmarito
FROM Matrimonio
GROUP BY CFMarito
HAVING COUNT(*) > 1;
\end{lstlisting}
\end{frame}
%
\begin{frame}[fragile]{Random: Es. 5}
Dato il seguente schema:
\schemaPersonaMatrimonio

Elencare nome e cognome degli sposati nell'anno 2012.
\pause
\begin{lstlisting}
SELECT m.Nome, m.Cognome, f.Nome, f.Cognome
FROM Matrimonio JOIN Persona AS m ON CFmarito = m.CF
JOIN Persona AS f ON CFmoglie = f.CF
WHERE anno = 2012;
\end{lstlisting}
\end{frame}
%
\begin{frame}[fragile]{Random: Es. 6}
Dato il seguente schema:
\schemaNegoziProdottiListino
Con vincoli di integrit\`a referenziale fra l'attributo Negozio di LISTINO e la relazione NEGOZI e fra l'attributo Prodotto di LISTINO e la relazione PRODOTTI.
\newline
\\Trovare nomi e citt\`a dei negozi che vendono i prodotti della marca `Durex'.
\pause
\begin{lstlisting}
SELECT DISTINCT Negozi.Nome, Negozi.Citta
FROM Negozi JOIN Listino ON Negozi.IDNegozio = Listino.Negozio
JOIN Prodotti ON Listino.Prodotto = Prodotti.CodProdotto
WHERE UPPER(Prodotti.Marca) = `DUREX';
\end{lstlisting}
\end{frame}
%
\begin{frame}[fragile]{Random: Es. 7}
Dato il seguente schema:
\schemaNegoziProdottiListino
Con vincoli di integrit\`a referenziale fra l'attributo Negozio di LISTINO e la relazione NEGOZI e fra l'attributo Prodotto di LISTINO e la relazione PRODOTTI.
\newline
\\Quanti negozi vendono il prodotto con codice `A'.
\pause
\begin{lstlisting}
SELECT COUNT(*)
FROM Listino
WHERE Prodotto = `A';
\end{lstlisting}
\end{frame}
%
\begin{frame}[fragile]{Random: Es. 8}
Dato il seguente schema:
\schemaNegoziProdottiListino
Con vincoli di integrit\`a referenziale fra l'attributo Negozio di LISTINO e la relazione NEGOZI e fra l'attributo Prodotto di LISTINO e la relazione PRODOTTI.
\newline
\\I codici dei prodotti che vengono venduti in una sola citt\`a.
\pause
\begin{lstlisting}
SELECT DISTINCT Prodotto
FROM Listino JOIN Negozi ON Negozio=IDnegozio
GROUP BY prodotto
HAVING COUNT(*) = 1;
\end{lstlisting}
\end{frame}
%
\begin{frame}[fragile]{Random: Es. 9}
Dato il seguente schema:
\schemaVenditeClientiDettagliVendite
con valori nulli ammessi sull'attributo Cliente e con vincoli di integrit\`a referenziale fra NumeroScontrino e la relazione Vendite fra Cliente e la relazione Clienti.
\newline
\\Definire in SQL la vista VenditeConTotale(NumeroScontrino, Totale) che riporta, per ogni scontrino, l'importo totale (ottenuto come somma degli importi dei prodotti riportati su tale scontrino).
\pause
\begin{lstlisting}
CREATE VIEW VenditeConTotale AS
SELECT NumeroScontrino, SUM(Importo) AS Totale
FROM DettagliVendite
GROUP BY NumeroScontrino;
\end{lstlisting}
\end{frame}
%
\begin{frame}[fragile]{Random: Es. 10}
Dato il seguente schema:
\schemaClientiNoleggiAuto
con vincoli di integrit\`a referenziale fra l'attributo Auto della relazione Noleggi e la relazione Autovetture e fra l'attributo Cliente della relazione Noleggi e la relazione Clienti.
\newline
\\Formulare in SQL l'interrogazione che restituisce, per ogni noleggio del 2008, il costo globale (ottenuto moltiplicando il costo giornaliero dell'auto noleggiata per la durata del noleggio) e i dati del cliente.
\pause
\begin{lstlisting}
SELECT Auto, DataPrelievo, CLIENTI.*, 
(DataRestituzione-DataPrelievo)*CostoGiornaliero AS CostoGlobale
FROM NOLEGGI JOIN CLIENTI ON Cliente = Codice
JOIN AUTOVETTURE ON Auto = Targa
WHERE YEAR(DataPrelievo) = 2008;    
\end{lstlisting}
\end{frame}
%
\begin{frame}[fragile]{Random: Es. 11}
Dato il seguente schema:
\schemaClientiNoleggiAuto
con vincoli di integrit\`a referenziale fra l'attributo Auto della relazione Noleggi e la relazione Autovetture e fra l'attributo Cliente della relazione Noleggi e la relazione Clienti.
\newline
\\Formulare in SQL l'interrogazione che restituisce i clienti che hanno noleggiato pi\`u di un'autovettura.
\pause
\begin{lstlisting}   
SELECT DISTINCT N1.Cliente
FROM NOLEGGI N1 JOIN NOLEGGI N2 ON N1.Cliente = N2.Cliente
WHERE N1.Auto <> N2.Auto;
\end{lstlisting}
\end{frame}
%
\begin{frame}[fragile]{Random: Es. 12}
Dato il seguente schema:
\schemaClientiNoleggiAuto
con vincoli di integrit\`a referenziale fra l'attributo Auto della relazione Noleggi e la relazione Autovetture e fra l'attributo Cliente della relazione Noleggi e la relazione Clienti.
\newline
\\Formulare in SQL l'interrogazione che restituisce, per ogni autovettura, il numero di clienti che l'hanno noleggiata (che pu\`o essere zero).
\pause
\begin{lstlisting}   
SELECT Targa, COUNT (Cliente)
FROM Autovetture LEFT JOIN NOLEGGI ON Targa = Auto
GROUP BY Targa;    
\end{lstlisting}
\end{frame}
%
\begin{frame}[fragile]{Random: Es. 13}
Dato il seguente schema:
\schemaProdottiVendite
con vincolo di integrit\`a referenziale fra Prodotto e la relazione PRODOTTI.
\newline
\\Formulare in SQL l'interrogazione che trova codice, descrizione e marca per ogni prodotto che abbia almeno una vendita nell'anno 2007.
\pause
\begin{lstlisting}   
SELECT DISTINCT Codice, Descrizione, Marca
FROM PRODOTTI JOIN VENDITE ON Codice = Prodotto
WHERE Anno = 2007;
\end{lstlisting}
\end{frame}
%
\begin{frame}[fragile]{Random: Es. 14}
Dato il seguente schema:
\schemaProdottiVendite
con vincolo di integrit\`a referenziale fra Prodotto e la relazione PRODOTTI.
\newline
\\Formulare in SQL l'interrogazione che trova, per ogni prodotto (basta mostrare il codice), la quantit\`a venduta nel 2007 (cio\`e la somma delle quantit\`a vendute nei mesi del 2007); mostrare due formulazioni, una che ignora i prodotti non venduti nel 2007 e l'altra che li include, ponendo il valore pari a 0.
\pause
Formulazione 1:
\begin{lstlisting}   
SELECT Prodotto AS Codice, SUM(Quantita) AS QTot
FROM VENDITE
WHERE Anno = 2007
GROUP BY Prodotto;
\end{lstlisting}
\end{frame}
%
\begin{frame}[fragile]{Random: Es. 14}
\vspace{-.7cm}
Dato il seguente schema:
\schemaProdottiVendite
con vincolo di integrit\`a referenziale fra Prodotto e la relazione PRODOTTI.

\pause
Formulazione 2: include i prodotti non venduti nel 2007, ponendo il valore pari a 0
\begin{lstlisting}   
SELECT Prodotto AS Codice, SUM(Quantita) AS QTot
FROM VENDITE
WHERE Anno = 2007
GROUP BY Prodotto
UNION
SELECT Codice, 0 AS QTot
FROM PRODOTTI
WHERE Codice NOT IN (
    SELECT Prodotto
    FROM VENDITE
    WHERE Anno = 2007
);
\end{lstlisting}
\end{frame}
%
\begin{frame}[fragile]{Random: Es. 15}
Dato il seguente schema:
\schemaRivisteArticoli
\newline
\\Formulare in SQL l'interrogazione per estrarre il codice e il nome delle riviste che hanno pubblicato almeno un articolo di motociclismo.
\pause

\begin{lstlisting}   
SELECT CodR, NomeR
FROM RIVISTE
WHERE CodR IN (
    SELECT CodR
    FROM ARTICOLI
    WHERE Argoment = `Motociclismo'
);
\end{lstlisting}
\end{frame}
%
\begin{frame}[fragile]{Random: Es. 15}
Dato il seguente schema:
\schemaRivisteArticoli
\newline
\\Formulare in SQL l'interrogazione per estrarre il codice e il nome delle riviste che hanno pubblicato almeno un articolo di motociclismo.

Altra possibile soluzione:
\begin{lstlisting}   
SELECT CodR, NomeR
FROM RIVISTE AS R
WHERE EXISTS (
    SELECT *
    FROM ARTICOLI AS A
    WHERE A.CodR = R.CodR AND Argomento = `Motociclismo'
);   
\end{lstlisting}
\end{frame}
%
\begin{frame}[fragile]{Random: Es. 16}
Dato il seguente schema:
\schemaRivisteArticoli
\newline
\\Formulare in SQL l'interrogazione per estrarre il nome delle riviste che hanno pubblicato solo articoli di motociclismo.
\pause

\begin{lstlisting}   
SELECT DISTINCT NomeR
FROM RIVISTE JOIN ARTICOLI ON RIVISTE.CodR = ARTICOLI.CodR
WHERE RIVISTE.CodR NOT IN (
    SELECT CodR
    FROM ARTICOLI
    WHERE Argomento <> `Motociclismo'
);   
\end{lstlisting}
\end{frame}
%
\begin{frame}[fragile]{Random: Es. 17}
\vspace{-.5cm}
Dato il seguente schema:
\schemaRivisteArticoli
\\Formulare in SQL l'interrogazione per estrarre il nome delle riviste che hanno pubblicato almeno 10 articoli di automobilismo e almeno 25 articoli di motociclismo.
\pause

\begin{lstlisting}   
SELECT NomeR FROM RIVISTE
WHERE CodR IN (
    SELECT CodR FROM ARTICOLI
    WHERE Argomento = `Automobilismo'
    GROUP BY CodR HAVING COUNT(*)>=10
    ) 
AND CodR IN (
    SELECT CodR FROM ARTICOLI
    WHERE Argomento = `Motociclismo'
    GROUP BY CodR HAVING COUNT(*) >= 25
    );
\end{lstlisting}
\end{frame}
%
\begin{frame}[fragile]{Random: Es. 18}
\vspace{-.5cm}
Dato il seguente schema:
\schemaMuseiOpereArtisti
con vincoli di integrit\`a referenziale indicati da $*$.
\newline
\\Si formuli la seguente interrogazione SQL che restituisce senza duplicazione dei risultati:

Per ciascun museo che conserva solo dipinti, il nome del museo ed il numero di dipinti.
\pause

\begin{lstlisting}   
SELECT M.NomeM, COUNT(*) AS NumeroDipinti
FROM MUSEI AS M JOIN OPERE AS O ON M.IdMuseo = O.IdMuseo
WHERE NOT EXISTS (
    SELECT *
    FROM OPERE AS O2
    WHERE O2.IdMuseo = M.IdMuseo AND O2.Tipo <> 'Dipinto'
)
GROUP BY M.IdMuseo, M.NomeM;
\end{lstlisting}
\end{frame}
%
\begin{frame}[fragile]{Random: Es. 19}
\vspace{-.5cm}
Dato il seguente schema:
\schemaMuseiOpereArtisti
con vincoli di integrit\`a referenziale indicati da $*$.
\newline
\\Si formuli la seguente interrogazione SQL che restituisce senza duplicazione dei risultati:

Per ogni museo italiano, il numero di opere di artisti inglesi.
\pause

\begin{lstlisting}   
SELECT M.IdMuseo, COUNT(*) AS numeroOpere
FROM MUSEI AS M JOIN OPERE AS O ON M.IdMuseo = O.IdMuseo
JOIN ARTISTI AS A ON O.NomeA = A.NomeA
WHERE M.Nazione = `Italia' AND A.Nazione = `Inghilterra'
GROUP BY M.IdMuseo;
\end{lstlisting}
\end{frame}
%
\begin{frame}[fragile]{Random: Es. 20}
\vspace{-.5cm}
Dato il seguente schema:
\schemaMuseiOpereArtisti
con vincoli di integrit\`a referenziale indicati da $*$.
\newline
\\Si formuli la seguente interrogazione SQL che restituisce senza duplicazione dei risultati:

Il nome dell'artista ed il titolo delle opere di artisti francesi conservate nei musei di Londra.
\pause

\begin{lstlisting}   
SELECT A.NomeA, O.Titolo
FROM ARTISTI AS A JOIN OPERE AS O ON A.NomeA = O.NomeA
JOIN MUSEI AS M ON O.IdMuseo = M.IdMuseo
WHERE A.Nazionalita = `Francia' AND M.Citta = `Londra';
\end{lstlisting}
\end{frame}
%
\begin{frame}[fragile]{Random: Es. 21}
\vspace{-.5cm}
Dato il seguente schema:
\schemaMuseiOpereArtisti
con vincoli di integrit\`a referenziale indicati da $*$.
\newline
\\Si formuli la seguente interrogazione SQL che restituisce senza duplicazione dei risultati:

Il nome e la citt\`a dei musei che conservano almeno 20 opere di uno stesso artista.
\pause

\begin{lstlisting}   
SELECT DISTINCT M.NomeM, M.Citta
FROM MUSEI AS M JOIN OPERE AS O ON M.IdMuseo = O.IdMuseo
GROUP BY M.IdMuseo, M.NomeM, M.Citta, O.NomeA
HAVING COUNT(*) >= 20;
\end{lstlisting}
\end{frame}
%
\begin{frame}[fragile]{Random: Es. 22}
\vspace{-.5cm}
Dato il seguente schema:
\schemaMuseiOpereArtisti
con vincoli di integrit\`a referenziale indicati da $*$.
\newline
\\Si formuli la seguente interrogazione SQL che restituisce senza duplicazione dei risultati:

Il nome degli artisti italiani che non hanno opere in musei francesi tranne che al Louvres.
\pause

\begin{lstlisting}   
SELECT A.NomeA
FROM ARTISTI AS A
WHERE A.Nazione = `Italia' AND NOT EXISTS (
    SELECT *
    FROM OPERE AS O JOIN MUSEI AS M
    ON (O.IdMuseo = M.IdMuseo AND O.NomeA = A.NomeA)
    WHERE M.Nazione = `Francia' AND M.NomeM <> `Louvres'
);
\end{lstlisting}
\end{frame}