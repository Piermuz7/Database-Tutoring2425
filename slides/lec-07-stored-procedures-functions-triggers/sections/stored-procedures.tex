\begin{frame}[fragile]{Stored Procedures (SPs)}
\vspace{-.7cm}
\begin{minipage}{.85\textwidth}
\begin{block}{Stored Procedure}
Una \textbf{stored procedure (SP)} (procedura memorizzata) \`e un codice SQL preparato che pu\`o essere salvato e riutilizzato pi\`u volte.
\end{block}
\end{minipage}

\vspace{.2cm}

\begin{itemize}[<+->]
    \item SP = insieme di istruzioni SQL raggruppate per eseguire un compito specifico;
    \item Le SPs possono essere richiamate da:
        \begin{itemize}
            \item utenti
            \item applicazioni
            \item o altre procedure
        \end{itemize}
        \item Le SPs sono essenziali per automatizzare le attivit\`a del database, migliorare l'efficienza e ridurre la ridondanza.
        Incapsulando la logica all'interno delle SPs, gli sviluppatori possono semplificare il flusso di lavoro e applicare regole aziendali coerenti tra pi\`u applicazioni e sistemi.
\end{itemize}
\end{frame}
%
\begin{frame}[fragile]{SPs}
\framesubtitle{Sintassi}
\vspace{-1cm}
La sintassi per creare una SP varia a seconda del DBMS, ma in generale segue questo schema:

\begin{lstlisting}
DELIMITER //
CREATE PROCEDURE nome_procedura
(param1 tipoDato, param2 tipoDato, ..., paramN tipoDato)
BEGIN
statement_sql
END //
\end{lstlisting}
\end{frame}
%
\begin{frame}[fragile]{SPs: Osservazioni}
\framesubtitle{Sintassi}
\vspace{-.8cm}
\begin{lstlisting}
|\colorbox{magenta!40}{DELIMITER //}|
CREATE PROCEDURE nome_procedura
(param1 tipoDato, param2 tipoDato, ..., paramN tipoDato)
BEGIN
statement_sql
END //
\end{lstlisting}
\begin{minipage}{\textwidth}
    \begin{block}{Delimitatore}
    Si tratta di un carattere speciale che pu\`o essere utilizzato per terminare un'istruzione SQL.
    Questo carattere speciale pu`o essere $//$, $\$\$$, $@@$, ecc.
    Separa le singole istruzioni all'interno di uno script, di una query o di una stored procedure pi\`u grandi, consentendo a MySQL di interpretarle ed eseguirle correttamente.
    \end{block}
    \end{minipage}
\end{frame}
%
\begin{frame}[fragile]{SPs: Osservazioni}
\framesubtitle{Sintassi}
\vspace{-.8cm}
\begin{lstlisting}
DELIMITER //
|\colorbox{magenta!40}{CREATE PROCEDURE nome\_procedura}|
(param1 tipoDato, param2 tipoDato, ..., paramN tipoDato)
BEGIN
statement_sql
END //
\end{lstlisting}
\begin{minipage}{\textwidth}
    \begin{block}{CREATE PROCEDURE}
    Questa istruzione viene utilizzata per creare una nuova stored procedure nel database.
    Il nome della procedura deve essere un identificatore univoco all'interno del database.
    \end{block}
    \end{minipage}
\end{frame}
%
\begin{frame}[fragile]{SPs: Osservazioni}
\framesubtitle{Sintassi}
\vspace{-.8cm}
\begin{lstlisting}
DELIMITER //
CREATE PROCEDURE nome_procedura
|\colorbox{magenta!40}{(param1 tipoDato, param2 tipoDato, ..., paramN tipoDato)}|
BEGIN
statement_sql
END //
\end{lstlisting}
\begin{minipage}{\textwidth}
    \scriptsize \begin{block}{Parametri}
    I parametri sono opzionali e possono essere utilizzati per passare valori alla procedura.
    
    Possono essere di tipo:
    \begin{itemize}
        \item \textbf{IN}: il valore viene passato alla procedura, ma non pu\`o essere modificato all'interno della procedura;
        \item \textbf{OUT}: il valore pu\`o essere modificato all'interno della procedura e il risultato viene restituito al chiamante;
        \item \textbf{INOUT}: il valore viene passato alla procedura e pu\`o essere modificato all'interno della procedura.
    \end{itemize}
    \end{block}
    \end{minipage}
\end{frame}
%
\begin{frame}[fragile]{SPs: Osservazioni}
\framesubtitle{Sintassi}
\vspace{-.8cm}
\begin{lstlisting}
DELIMITER //
CREATE PROCEDURE nome_procedura
(param1 tipoDato, param2 tipoDato, ..., paramN tipoDato)
|\colorbox{magenta!40}{BEGIN}|
statement_sql
|\colorbox{magenta!40}{END //}|
\end{lstlisting}
\begin{minipage}{\textwidth}
    \begin{block}{BEGIN ... END}
    L'istruzione \texttt{BEGIN} indica l'inizio del blocco di codice della procedura, mentre \texttt{END} indica la fine del blocco.
    \end{block}
    \end{minipage}
\end{frame}
%
\begin{frame}[fragile]{SPs: Osservazioni}
\framesubtitle{Sintassi}
\vspace{-.8cm}
\begin{lstlisting}
DELIMITER //
CREATE PROCEDURE nome_procedura
(param1 tipoDato, param2 tipoDato, ..., paramN tipoDato)
BEGIN
|\colorbox{magenta!40}{statement\_sql}|
END //
\end{lstlisting}
\begin{minipage}{\textwidth}
    \begin{block}{Statement SQL}
    All'interno del blocco \texttt{BEGIN ... END}, si possono inserire una o pi\`u istruzioni SQL che definiscono la logica della procedura.
    
    Queste istruzioni possono includere operazioni di selezione, inserimento, aggiornamento o cancellazione di dati, nonch\'e altre operazioni di gestione del database.
    \end{block}
    \end{minipage}
\end{frame}
%
\begin{frame}[fragile]{SPs: Vantaggi}
    \begin{itemize}[<+->]
        \item \textbf{Riutilizzabilit\`a}: le SPs possono essere richiamate pi\`u volte, riducendo la necessit\`a di riscrivere lo stesso codice SQL in diverse parti dell'applicazione.
        \item \textbf{Prestazioni}: le SPs possono migliorare le prestazioni del database, poich\'e il codice SQL viene compilato e ottimizzato una sola volta, riducendo il tempo di esecuzione.
        \item \textbf{Sicurezza}: le SPs possono essere utilizzate per limitare l'accesso ai dati, consentendo agli utenti di eseguire solo determinate operazioni senza concedere loro accesso diretto alle tabelle del database.
        \item \textbf{Manutenibilit\`a}: le SPs possono semplificare la manutenzione del codice, poich\'e le modifiche alla logica della procedura possono essere apportate in un unico punto, senza dover aggiornare pi\`u parti dell'applicazione.
        \item \ldots
    \end{itemize}
\end{frame}
%
\begin{frame}[fragile]{SPs: Esempio Sakila con Parametro di Input}
\framesubtitle{Esempio di creazione di una SP}
La seguente SP restituisce tutti i film noleggiati da un cliente avente ID specificato:
\small 
\begin{lstlisting}
DELIMITER \\

CREATE PROCEDURE getCustomerRentals(IN cust_id INT)
BEGIN
    SELECT f.title, r.rental_date
    FROM rental r
    JOIN inventory i ON r.inventory_id = i.inventory_id
    JOIN film f ON i.film_id = f.film_id
    WHERE r.customer_id = cust_id
    ORDER BY r.rental_date DESC;
END \\

DELIMITER ;
\end{lstlisting}
\end{frame}
%
\begin{frame}[fragile]{SPs: Esempio Sakila con Parametro di Input}
\framesubtitle{Esempio di chiamata di una SP}
Per richiamare la SP appena creata, si utilizza la seguente sintassi:
\small 
\begin{lstlisting}
CALL getCustomerRentals(1);

oppure

SET @cust_id = 1;
CALL getCustomerRentals(@cust_id);
\end{lstlisting}
\end{frame}
\begin{frame}[fragile]{SPs: Esempio Sakila con Parametro di Input e Output}
\framesubtitle{Esempio di creazione di una SP}
La seguente SP restituisce il numero di noleggi effettuati da un cliente avente ID specificato:
\small
\begin{lstlisting}
DELIMITER \\
CREATE PROCEDURE getCustomerRentalCount(IN cust_id INT,
OUT rental_count INT)
BEGIN
    SELECT COUNT(*)
    INTO rental_count
    FROM rental
    WHERE customer_id = cust_id;
END \\
DELIMITER ;
\end{lstlisting}
\end{frame}
%
\begin{frame}[fragile]{SPs: Esempio Sakila con Parametro di Input e Output}
\framesubtitle{Esempio di chiamata di una SP}
Per richiamare la SP appena creata, si utilizza la seguente sintassi:
\small 
\begin{lstlisting}
SET @cust_id = 1;
CALL getCustomerRentalCount(@cust_id, @rental_count);

SELECT @cust_id AS 'ID cliente',
@rental_count AS 'numero di noleggi effettuati';
\end{lstlisting}
\end{frame}