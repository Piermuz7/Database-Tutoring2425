\def\consegnaSPsExOne{
Creare una stored procedure \textbf{get\_customer\_total\_payment} che riceve un \textbf{customer\_id} e restituisce l'importo totale pagato da quel cliente.
}
\def\consegnaSPsExTwo{
Creare una stored procedure \textbf{insert\_rental} che inserisce una nuova riga nella tabella \textbf{rental} dato:
    \begin{itemize}
        \item \textbf{customer\_id}
        \item \textbf{inventory\_id}
        \item \textbf{staff\_id}
    \end{itemize}
}
\def\consegnaSPsExThree{
Creare una stored procedure \textbf{deactivate\_inactive\_customers} che imposta \textbf{active = 0} per tutti i clienti che non hanno effettuato un pagamento negli ultimi 12 mesi.
}
\def\consegnaFunctionsExOne{
Creare una funzione \textbf{get\_customer\_full\_name} che prende un \textbf{customer\_id} e restituisce il nome completo (ad esempio, 'John Smith') come stringa unica.
}
\def\consegnaFunctionsExTwo{
Creare una funzione \textbf{get\_rental\_count} che restituisce il numero di noleggi effettuati da un determinato \textbf{customer\_id}.
}
\def\consegnaFunctionsExThree{
Creare una funzione \textbf{is\_film\_available} che prende un \textbf{film\_id} e restituisce 1 se c'\`e almeno una copia di quel film nell'inventario, altrimenti 0.
}
\def\consegnaTriggersExOne{
Creare un trigger che memorizzi il \textbf{titolo} e il \textbf{timestamp} di cancellazione di qualsiasi film che viene eliminato dalla \textbf{tabella film}.

Utilizzare una nuova tabella chiamata \textbf{deleted\_films\_log}.
}
\def\consegnaTriggersExTwo{
Creare un trigger che aggiorni una tabella \textbf{customer\_payment\_count} ogni volta che viene inserito un pagamento.
Se il cliente esiste gi\`a nella tabella, incrementare il suo conteggio; altrimenti, inserirlo con count = 1.

Prima di tutto, creare una tabella di riepilogo \textbf{customer\_payment\_count(customer\_id, payment\_count)}.
}
\def\consegnaTriggersExThree{
Creare un trigger che, prima di cancellare un cliente, copia i dati del cliente in una tabella \textbf{archived\_customers} per scopi di registrazione.
}
%
\begin{frame}[fragile]{Stored Procedures}
\framesubtitle{Esercizio 1: Calcolo del Totale Pagato da un Cliente}
\consegnaSPsExOne
\end{frame}
%
\begin{frame}[fragile]{Stored Procedures}
\framesubtitle{Soluzione Esercizio 1: Calcolo del Totale Pagato da un Cliente}
\consegnaSPsExOne

\vspace{.2cm}

\textbf{Creazione della stored procedure:}
\begin{lstlisting}
DELIMITER \\
CREATE PROCEDURE get_customer_total_payment(IN cust_id INT, OUT total DECIMAL(10,2))
BEGIN
    SELECT SUM(amount)
    INTO total
    FROM payment
    WHERE customer_id = cust_id;
END\\
DELIMITER ;
\end{lstlisting}
\end{frame}
%
\begin{frame}[fragile]{Stored Procedures}
\framesubtitle{Soluzione Esercizio 1: Calcolo del Totale Pagato da un Cliente}
\consegnaSPsExOne

\vspace{.2cm}

\textbf{Chiamata della stored procedure:}
\begin{lstlisting}
CALL get_customer_total_payment(1, @total);
SELECT @total;
\end{lstlisting}
\end{frame}
%
\begin{frame}[fragile]{Stored Procedures}
\framesubtitle{Esercizio 2: Inserimento di un Noleggio}
\consegnaSPsExTwo
\end{frame}
%
\begin{frame}[fragile]{Stored Procedures}
\framesubtitle{Soluzione Esercizio 2: Inserimento di un Noleggio}

\vspace{-.5cm}

\textbf{Creazione della stored procedure:}
\begin{lstlisting}
DELIMITER \\

CREATE PROCEDURE insert_rental(
    IN p_customer_id INT,
    IN p_inventory_id INT,
    IN p_staff_id INT
)
BEGIN
    INSERT INTO rental (rental_date, inventory_id, customer_id, staff_id)
    VALUES (NOW(), p_inventory_id, p_customer_id, p_staff_id);
END\\

DELIMITER ;
\end{lstlisting}
\end{frame}
%
\begin{frame}[fragile]{Stored Procedures}
\framesubtitle{Soluzione Esercizio 2: Inserimento di un Noleggio}
\consegnaSPsExOne

\vspace{.2cm}

\textbf{Chiamata della stored procedure:}
\begin{lstlisting}
CALL insert_rental(1, 1000, 2);
\end{lstlisting}
\end{frame}
%
\begin{frame}[fragile]{Stored Procedures}
\framesubtitle{Esercizio 3: Disattivazione Clienti Inattivi}
\consegnaSPsExThree
\end{frame}
%
\begin{frame}[fragile]{Stored Procedures}
\framesubtitle{Soluzione Esercizio 3: Disattivazione Clienti Inattivi}

\vspace{-.5cm}
\small
\textbf{Creazione della stored procedure:}
\begin{lstlisting}
DELIMITER \\

CREATE PROCEDURE deactivate_inactive_customers()
BEGIN
    UPDATE customer
    SET active = 0
    WHERE customer_id IN (
        SELECT c.customer_id
        FROM customer c
        LEFT JOIN payment p ON c.customer_id = p.customer_id
        GROUP BY c.customer_id
        HAVING MAX(p.payment_date) < DATE_SUB(NOW(), INTERVAL 12 MONTH)
    );
END\\
DELIMITER ;
\end{lstlisting}
\end{frame}
%
\begin{frame}[fragile]{Stored Procedures}
\framesubtitle{Soluzione Esercizio 3: Disattivazione Clienti Inattivi}
\consegnaSPsExOne

\vspace{.2cm}

\textbf{Chiamata della stored procedure:}
\begin{lstlisting}
CALL deactivate_inactive_customers();
\end{lstlisting}
\end{frame}
%
\begin{frame}[fragile]{Functions}
\framesubtitle{Esercizio 1: Nome Completo del Cliente}
\consegnaFunctionsExOne
\end{frame}
%
\begin{frame}[fragile]{Functions}
\framesubtitle{Soluzione Esercizio 1: Nome Completo del Cliente}

\vspace{-.5cm}
\small
\textbf{Creazione della funzione:}
\begin{lstlisting}
DELIMITER \\

CREATE FUNCTION get_customer_full_name(cust_id INT)
RETURNS VARCHAR(100)
DETERMINISTIC
BEGIN
    DECLARE full_name VARCHAR(100);
    SELECT CONCAT(first_name, ' ', last_name)
    INTO full_name
    FROM customer
    WHERE customer_id = cust_id;
    RETURN full_name;
END\\

DELIMITER ;
\end{lstlisting}
\end{frame}
%
\begin{frame}[fragile]{Functions}
\framesubtitle{Soluzione Esercizio 1: Nome Completo del Cliente}
\consegnaFunctionsExOne

\vspace{.2cm}

\textbf{Chiamata della funzione:}
\begin{lstlisting}
SELECT get_customer_full_name(1);
\end{lstlisting}
\end{frame}
%
\begin{frame}[fragile]{Functions}
\framesubtitle{Esercizio 2: Conteggio dei Noleggi per Cliente}
\consegnaFunctionsExTwo
\end{frame}
%
\begin{frame}[fragile]{Functions}
\framesubtitle{Soluzione Esercizio 2: Conteggio dei Noleggi per Cliente}

\vspace{-.5cm}
\small
\textbf{Creazione della funzione:}
\begin{lstlisting}
DELIMITER \\

CREATE FUNCTION get_rental_count(cust_id INT)
RETURNS INT
DETERMINISTIC
BEGIN
    DECLARE rental_total INT;
    SELECT COUNT(*) INTO rental_total
    FROM rental
    WHERE customer_id = cust_id;
    RETURN rental_total;
END\\

DELIMITER ;
\end{lstlisting}
\end{frame}
%
\begin{frame}[fragile]{Functions}
\framesubtitle{Soluzione Esercizio 2: Conteggio dei Noleggi per Cliente}
\consegnaFunctionsExTwo

\vspace{.2cm}

\textbf{Chiamata della funzione:}
\begin{lstlisting}
SELECT get_rental_count(1);
\end{lstlisting}
\end{frame}
%
\begin{frame}[fragile]{Functions}
\framesubtitle{Esercizio 3: Disponibilit\`a del Film}
\consegnaFunctionsExThree
\end{frame}
%
\begin{frame}[fragile]{Functions}
\framesubtitle{Soluzione Esercizio 3: Disponibilit\`a del Film}

\vspace{-.5cm}
\scriptsize
\textbf{Creazione della funzione:}
\begin{lstlisting}
DELIMITER \\

CREATE FUNCTION is_film_available(film_id_input INT)
RETURNS TINYINT
DETERMINISTIC
BEGIN
    DECLARE count_inventory INT;
    SELECT COUNT(*) INTO count_inventory
    FROM inventory
    WHERE film_id = film_id_input;

    IF count_inventory > 0 THEN
        RETURN 1;
    ELSE
        RETURN 0;
    END IF;
END\\

DELIMITER ;
\end{lstlisting}
\end{frame}
%
\begin{frame}[fragile]{Functions}
\framesubtitle{Soluzione Esercizio 3: Disponibilit\`a del Film}
\consegnaFunctionsExThree

\vspace{.2cm}

\textbf{Chiamata della funzione:}
\begin{lstlisting}
SELECT is_film_available(100);
\end{lstlisting}
\end{frame}
%
\begin{frame}[fragile]{Triggers}
\framesubtitle{Esercizio 1: Log di Cancellazione dei Film}
\consegnaTriggersExOne
\end{frame}
%
\begin{frame}[fragile]{Triggers}
\framesubtitle{Soluzione Esercizio 1: Log di Cancellazione dei Film}
\consegnaTriggersExOne

\vspace{.5cm}

\textbf{Step 1: Creazione della tabella di log}
\begin{lstlisting}
CREATE TABLE IF NOT EXISTS deleted_films_log (
    film_id INT,
    title VARCHAR(255),
    deleted_at DATETIME
);

\end{lstlisting}

\end{frame}
%
\begin{frame}[fragile]{Triggers}
\framesubtitle{Soluzione Esercizio 1: Log di Cancellazione dei Film}
\consegnaTriggersExOne

\vspace{.2cm}

\textbf{Step 2: Creazione del trigger}
\begin{lstlisting}
DELIMITER \\
CREATE TRIGGER after_delete_film_log
AFTER DELETE ON film
FOR EACH ROW
BEGIN
    INSERT INTO deleted_films_log(film_id, title, deleted_at)
    VALUES (OLD.film_id, OLD.title, NOW());
END\\
DELIMITER ;
\end{lstlisting}

\end{frame}
%
\begin{frame}[fragile]{Triggers}
\framesubtitle{Esercizio 2: Conteggio dei Pagamenti per Cliente}
\consegnaTriggersExTwo
\end{frame}
%
\begin{frame}[fragile]{Triggers}
\framesubtitle{Soluzione Esercizio 2: Conteggio dei Pagamenti per Cliente}

\consegnaTriggersExTwo

\vspace{.5cm}

\textbf{Step 1: Creazione della tabella di riepilogo}
\begin{lstlisting}
CREATE TABLE IF NOT EXISTS customer_payment_count (
    customer_id INT PRIMARY KEY,
    payment_count INT DEFAULT 0
);

\end{lstlisting}

\end{frame}
%
\begin{frame}[fragile]{Triggers}
\framesubtitle{Soluzione Esercizio 2: Conteggio dei Pagamenti per Cliente}

\scriptsize

\vspace{-1cm}

\textbf{Step 2: Creazione del trigger}
\begin{lstlisting}
DELIMITER \\

CREATE TRIGGER after_insert_payment_count
AFTER INSERT ON payment
FOR EACH ROW
BEGIN
    IF EXISTS (SELECT * FROM customer_payment_count WHERE customer_id = NEW.customer_id) THEN
        UPDATE customer_payment_count
        SET payment_count = payment_count + 1
        WHERE customer_id = NEW.customer_id;
    ELSE
        INSERT INTO customer_payment_count(customer_id, payment_count)
        VALUES (NEW.customer_id, 1);
    END IF;
END\\

DELIMITER ;
\end{lstlisting}
\end{frame}
%
\begin{frame}[fragile]{Triggers}
\framesubtitle{Esercizio 3: Impedire la Cancellazione di Staff Attivi}
\consegnaTriggersExThree
\end{frame}
%
\begin{frame}[fragile]{Triggers}
\framesubtitle{Soluzione Esercizio 3: Impedire la Cancellazione di Staff Attivi}

\textbf{Step 1: Creazione della tabella di archiviazione}
\begin{lstlisting}
CREATE TABLE IF NOT EXISTS archived_customers (
    customer_id INT,
    store_id TINYINT,
    first_name VARCHAR(45),
    last_name VARCHAR(45),
    email VARCHAR(50),
    address_id SMALLINT,
    active BOOLEAN,
    create_date DATETIME,
    archive_date DATETIME
);

\end{lstlisting}

\end{frame}
%
\begin{frame}[fragile]{Triggers}
\framesubtitle{Soluzione Esercizio 3: Impedire la Cancellazione di Staff Attivi}

\vspace{-.5cm}
\small
\textbf{Step 2: Creazione del trigger}
\begin{lstlisting}
DELIMITER \\
CREATE TRIGGER before_delete_customer_archive
BEFORE DELETE ON customer
FOR EACH ROW
BEGIN
    INSERT INTO archived_customers (
        customer_id, store_id, first_name, last_name, email,
        address_id, active, create_date, archive_date
    )
    VALUES (
        OLD.customer_id, OLD.store_id, OLD.first_name, OLD.last_name, OLD.email,
        OLD.address_id, OLD.active, OLD.create_date, NOW()
    );
END\\
DELIMITER ;
\end{lstlisting}
\end{frame}