\begin{frame}[fragile]{Triggers}
\vspace{-.7cm}
\begin{minipage}{.85\textwidth}
\begin{block}{Trigger}
Un \textbf{trigger} \`e un oggetto del database che viene automaticamente eseguito in risposta a determinati eventi che si verificano su una tabella o una vista.

In altre parole, un trigger \`e una SP che viene attivata automaticamente quando si verificano eventi specifici nel database.
\end{block}
\end{minipage}

\vspace{.2cm}

I trigger:
\begin{itemize}[<+->]
    \item consentono l'esecuzione di istruzioni SQL quando si verificano eventi specifici del database, ad esempio quando qualcuno aggiunge, modifica o rimuove dati.
    \item sono comunemente utilizzati per mantenere l'integrit\`a dei dati, tracciare le modifiche e applicare le regole di business automaticamente, senza bisogno di input manuali.
    \item permettono agli sviluppatori di automatizzare le attivit\`a, garantire la coerenza dei dati e mantenere una registrazione accurata delle attivit\`a del database.
\end{itemize}
\end{frame}
%
\begin{frame}{Triggers}
    \begin{itemize}
        \item I DBMS che supportano la programmazione basata su regole possono creare basi di dati che sono dette \textbf{basi di dati attive};
        \item Le istruzioni vengono dette ``triggers'' (o regole attive) e vengono interpretate dal processore delle regole del DBMS;
        \item I trigger possono essere:
            \begin{itemize}
                \item \textbf{esterni}: definiti dal Database Administrator (DBA) o dagli sviluppatori applicativi;
                \item \textbf{interni}: definiti dal DBMS stesso (non visibili all'utente).
            \end{itemize}
    \end{itemize}
\end{frame}
%
\begin{frame}{Triggers: Evento-Condizione-Azione}
    \begin{itemize}
        \item Paradigma Evento-Condizione-Azione (ECA):
            \begin{itemize}
                \item \textbf{Evento}: un evento che si verifica nel database, come l'inserimento, l'aggiornamento o la cancellazione di dati.
                Quando si verifica un evento, il trigger viene \textbf{attivato}.
                \item \textbf{Condizione}: un predicato che deve essere soddisfatta affinch\'e il trigger venga attivato.
                Se il predicato \`e vero, il trigger entra nello stato \textbf{valutato}. Se il predicato \`e falso, il trigger resta nello stato \textbf{non valutato}.
                \item \textbf{Azione}: l'azione da eseguire quando l'evento si verifica e la condizione \`e soddisfatta.
                Il trigger entra nello stato \textbf{eseguito} e l'azione viene eseguita.
            \end{itemize}
    \end{itemize}
\end{frame}
%
\begin{frame}{Triggers: Regole Relazionali}
    Esistono due modalit\`a:
    \begin{itemize}
        \item \textbf{Immediata}: l'azione viene eseguita immediatamente dopo che l'evento si verifica e la condizione \`e soddisfatta;
        \begin{itemize}
            \item \textbf{BEFORE}: l'azione viene eseguita prima che l'evento venga completato;
            \item \textbf{AFTER}: l'azione viene eseguita dopo che l'evento \`e stato completato.
        \end{itemize}
        \item \textbf{Differita}: Il trigger si attiva solo alla fine della transazione, cio\`e quando viene dato il comando COMMIT.
    \end{itemize}
    e due livelli di granularit\`a:
    \begin{itemize}
        \item \textbf{row-level}: l'attivazione avviene per ogni tupla coinvolta nell'evento; \textbf{MySQL} \checkmark
        \item \textbf{statement-level}: l'attivazione avviene per ogni primitiva SQL. \textbf{MySQL} \xmark
    \end{itemize}

    \begin{block}{Nota bene}
        MySQL supporta solo trigger a livello \textbf{row-level}.
        I trigger a livello \textbf{statement-level} non sono supportati.
        I trigger possono essere definiti per gli eventi \textbf{INSERT}, \textbf{UPDATE} e \textbf{DELETE}.
    \end{block}
\end{frame}
%
\begin{frame}[fragile]{Triggers: Sintassi SQL}
\framesubtitle{Sintassi}
\vspace{-.5cm}
\begin{lstlisting}
CREATE TRIGGER nome_trigger
    { BEFORE |\textbar| AFTER } {INSERT |\textbar| DELETE |\textbar| UPDATE}

    ON nome_tabella

    [ REFERENCING reference ]

    [ FOR EACH ROW |\textbar| STATEMENT ]

    [ WHEN (condizione) ]

    azione;
\end{lstlisting}

\end{frame}
%
\begin{frame}[fragile]{Triggers: Esempio Sakila BEFORE INSERT}
\framesubtitle{Esempio creazione trigger}
\vspace{-.5cm}
\textbf{Obiettivo}: prima di inserire un nuovo cliente, convertiamo automaticamente il \\last\_name in maiuscolo.

\begin{lstlisting}
    DELIMITER \\
    CREATE TRIGGER before_insert_customer
    BEFORE INSERT ON customer
    FOR EACH ROW
    BEGIN
        SET NEW.last_name = UPPER(NEW.last_name);
    END\\
    DELIMITER ;
\end{lstlisting}

\vspace{-.2cm}

\begin{block}{Osservazioni}
    \small \textbf{NEW} \`e una parola chiave che rappresenta la nuova riga che verr\`a inserita nella tabella.
    
    \textbf{UPPER()} \`e una funzione SQL che converte il testo in maiuscolo.
    Il trigger viene attivato prima dell'inserimento di una nuova riga nella tabella \texttt{customer}.
\end{block}
\end{frame}
%
\begin{frame}[fragile]{Triggers: Esempio Sakila BEFORE INSERT}
\framesubtitle{Esempio inserimento cliente}

Inseriamo un nuovo cliente nella tabella customer:
\begin{lstlisting}
INSERT INTO customer (store_id,first_name,last_name,email,address_id,active,create_date)
VALUES (1,'Mario','rossi','mario.rossi@example.com',5,1,NOW());
\end{lstlisting}

Verifichiamo che il trigger abbia convertito il last\_name in maiuscolo:
\begin{lstlisting}
SELECT * FROM customer
WHERE email = 'mario.rossi@example.com';

\end{lstlisting}

\end{frame}
%
\begin{frame}[fragile]{Triggers: Esempio Sakila AFTER UPDATE}
\framesubtitle{Esempio creazione trigger}
\vspace{-.5cm}
\textbf{Obiettivo}: registrare ogni modifica all'importo (amount) di un pagamento, salvando le informazioni nella tabella payment\_audit.

\begin{lstlisting}
CREATE TABLE IF NOT EXISTS payment_audit (
    payment_id INT,
    old_amount DECIMAL(5,2),
    new_amount DECIMAL(5,2),
    changed_at DATETIME
);
\end{lstlisting}
\end{frame}
%
\begin{frame}[fragile]{Triggers: Esempio Sakila AFTER UPDATE}
\framesubtitle{Esempio creazione trigger}
\vspace{-.5cm}

\begin{lstlisting}
DELIMITER \\

CREATE TRIGGER after_update_payment
AFTER UPDATE ON payment
FOR EACH ROW
BEGIN
    IF OLD.amount <> NEW.amount THEN
        INSERT INTO payment_audit(payment_id, old_amount, new_amount, changed_at)
        VALUES (OLD.payment_id, OLD.amount, NEW.amount, NOW());
    END IF;
END\\

DELIMITER ;
\end{lstlisting}
\end{frame}
%
\begin{frame}[fragile]{Triggers: Esempio Sakila AFTER UPDATE}
\framesubtitle{Esempio aggiornamento pagamento}
\vspace{-.5cm}

Controlliamo se esiste un pagamento da aggiornare:
\begin{lstlisting}
    SELECT payment_id, amount
    FROM payment
    WHERE amount > 5
    LIMIT 1;
\end{lstlisting}

Se esiste (supponiamo payment\_id = 3), aggiorniamo l'importo del pagamento:
\begin{lstlisting}
    UPDATE payment
    SET amount = 9.99
    WHERE payment_id = 3;
\end{lstlisting}

Verifichiamo che il trigger abbia registrato la modifica:
\begin{lstlisting}
    SELECT * FROM payment_audit
    WHERE payment_id = 3;
\end{lstlisting}

\end{frame}
%
\begin{frame}[fragile]{Triggers: Comandi utili}

Per visualizzare i \textbf{trigger definiti} nel database:
\begin{lstlisting}
    SHOW TRIGGERS;
\end{lstlisting}

Per \textbf{eliminare} un trigger:
\begin{lstlisting}
    DROP TRIGGER IF EXISTS nome_trigger;
\end{lstlisting}

Per visualizzare il \textbf{codice sorgente} di un trigger:
\begin{lstlisting}
    SHOW CREATE TRIGGER nome_trigger;
\end{lstlisting}

\end{frame}
%
