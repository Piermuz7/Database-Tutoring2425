\begin{frame}{Esercizio 1}
\begin{table}[h]
    \centering
    \scriptsize
    \begin{tabular}{| l | l | l | l | l | l | l |}
        \hline
        \rowcolor{cyan!30} \textbf{REGISTA} & \textbf{NAZIONE} & \textbf{REG} & \textbf{FILM} & \textbf{GENERE} & \textbf{DURATA} & \textbf{METRAGGIO} \\ \hline
        Tarantino & Stati Uniti & Pulp Fiction & gangster & 154 & lungo & 14/10/1994 \\ \hline
        Park Chan-wook & Corea del Sud & Old Boy & thriller & 120 & lungo & 21/11/2003 \\ \hline
        Spike Lee & Stati Uniti & Old Boy & azione & 105 & lungo & 27/11/2013 \\ \hline
        Christopher Nolan & Inghilterra & Memento & thriller & 113 & lungo & 5/9/2000 \\ \hline
        Christopher Nolan & Inghilterra & Inception & fantascienza & 148 & lungo & 8/7/2010 \\ \hline
        Mario Monicelli & Italia & Il marchese del Grillo & commedia & 139 & lungo & 16/08/1981 \\ \hline
    \end{tabular}
\end{table}
\vspace{1cm}
    \begin{itemize}[<+->]
    \item Capire in quale forma normale si trova la tabella e motivarne la risposta;
    \item Passare gradualmente dalla forma normale attuale fino alla 3NF passando passo passo per tutte le normalizzazioni intermedie, motivandone le modifiche.
\end{itemize}

\end{frame}
%
\begin{frame}{Esercizio 2}
    \begin{table}[h]
    \centering
    \begin{tabular}{| l | l | l | l | l |}
        \hline
        \rowcolor{cyan!30} \textbf{\underline{MATRICOLA}} & \textbf{\underline{CODICE ESAME}} & \textbf{EMAIL} & \textbf{VOTO} \\ \hline
        607080 & ST020 & pippo.calippo@studenti.unicam.it & 18 \\ \hline
        607085 & ST040 & ciccio.pasticcio@studenti.unicam.it & 20 \\ \hline
        607087 & ST050 & cacio.cappella@studenti.unicam.it & 22 \\ \hline
        607080 & ST040 & pippo.calippo@studenti.unicam.it & 24 \\ \hline
        607085 & ST040 & ciccio.pasticcio@studenti.unicam.it & 26 \\ \hline
        607085 & ST050 & ciccio.pasticcio@studenti.unicam.it & 28 \\ \hline
    \end{tabular}
\end{table}
\vspace{1cm}
    \begin{itemize}[<+->]
    \item Capire in quale forma normale si trova la tabella e motivarne la risposta;
    \item Passare dalla forma normale attuale alla BCNF motivandone le modifiche.
\end{itemize}
\end{frame}